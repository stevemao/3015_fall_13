\documentclass[11pt]{article}

% \usepackage{amssymb}
\usepackage{remark}
% \usepackage{html}


\usepackage{/home/faculty/jlc/papers/tex-stuff/bussproofs}



\setlength{\textwidth}{6.5in}
\setlength{\textheight}{8.8in}
\setlength{\oddsidemargin}{0.0in}
\setlength{\evensidemargin}{0.0in}


 \newremark{theorem}{Theorem}[section]
 \newremark{Rule}{Rule}[section]
 \newremark{conjecture}{Conjecture}[section]
 \newremark{corollary}{Corollary}[section]
 \newremark{example}{Example}[section]
 \newremark{fact}{Fact}[section]
 \newremark{lemma}{Lemma}[section]
 \newremark{definition}{Definition}[section]
 \newremark{claim}{Claim}[section]
 \newremark{remark}{Remark}[section]
 \newremark{slogan}{Slogan}[section]
 \newremark{axiom}{Axiom}[section]
 \newremark{note}{Note}[]
 \newremark{problem}{Problem}[section]
 \newremark{exercise}{Exercise}[section]

\newcommand{\Proof}{{\goodbreak\noindent\bf{Proof:\ }}}
\newcommand{\qed}{\goodbreak\noindent$\Box$}

\newcommand{\byDef}[1]{\stackrel{\langle\langle{{\rm{def.\ of\ }}{#1}}\rangle\rangle}{=}}
\newcommand{\byEq}[1]{\stackrel{\langle\langle{{#1}}\rangle\rangle}{=}}
\newcommand{\by}[1]{{\small{\langle\langle{\mathit{by.\;def.\;of\;}}{#1}\rangle\rangle}}}


\newcommand{\dotcup}{{\cup\hspace{-.5em}\cdot}}
\newcommand{\barcup}{{\cup\hspace{-.68em}-}}


\newcommand{\arrow}{$\rightarrow$}
\newcommand{\nat}{{\mathbb{N}}}
\newcommand{\bool}{{\mathbb{B}}}
\newcommand{\Int}{{\mathbb{Z}}}

\newcommand{\nand}{{\overline{\wedge}}}
\newcommand{\nor}{{\overline{\vee}}}


\newcommand{\Meaning}[1]{{[{\hspace{-.13em}}[{#1}]{\hspace{-.13em}}]}}
\newcommand{\Meaningof}[2]{{[{\hspace{-.13em}}[\,{#2:#1}\,]{\hspace{-.13em}}]}}
\newcommand{\Skip}{{\bf{skip}}}
\newcommand{\Let}[2]{{\bf{Let\ }} {#1}{\bf{\ in\ }}{#2}{\bf{\ end}}}
\newcommand{\A}{{\tt{a}}}
% \newcommand{\F}{{\tt{F}}}
\newcommand{\ua}[1]{{\tt{(#1 under a)}}}
\newcommand{\Fua}{\ua{F}}
\newcommand{\aF}{{\tt{(a F)}}}
\newcommand{\Btt}{{\tt{Btt}}}
\newcommand{\Bff}{{\tt{Bff}}}
%\newcommand{\T3}{{\tt{T3}}}
%\newcommand{\F3}{{\tt{F3}}}
%\newcommand{\?3}{{\tt{?3}}}
\newcommand{\V}[1]{{\mkleeneopen{}{\tt{#1}}\mkleeneclose{}}}
\newcommand{\Neg}[1]{{\tt{N(#1)}}}
\newcommand{\I}[2]{{\tt{(#1)I(#2)}}}
\newcommand{\ie}{{\em{i.e.}}}
\newcommand{\eg}{{\em{e.g.}}}
\newcommand{\sat}[2]{{\tt{#1 \mbar{}= #2}}}
\newcommand{\nsat}[2]{{\tt{#1 \mbar{}\mneq{} #2}}}
\newcommand{\Three}{{\mBbbN{}\msubthree{}}}
%\newcommand{\Three}{{\bf{3}}}
\newcommand{\Tzero}{${\tt{0}}_{\tt{3}}$}
\newcommand{\Tone}{${\tt{1}}_{\tt{3}}$}
\newcommand{\Ttwo}{${\tt{2}}_{\tt{3}}$}
\newcommand{\msubst}{{\tt{/}}}
\newcommand{\nsequent}{{\tt{>>}}}

\newcommand{\kleene}[1]{{\mkleeneopen{}#1\mkleeneclose{}}}
\newcommand{\mfvar}[1]{{\kleene{{\tt{#1}}}}}
\newcommand{\mfnot}{{\kleene{\msim}}}
\newcommand{\mfand}{{\kleene{\mwedge}}}
\newcommand{\mfor}{{\kleene{\mvee}}}
\newcommand{\mfimp}{{\kleene{\mRightarrow}}}
\newcommand{\Knot}{$\sim_K\,$}
\newcommand{\Kand}{$\wedge_K$}
\newcommand{\Kor}{$\vee_K$}
\newcommand{\Kimp}{$\Rightarrow_K$}
\newcommand{\miff}{$\Leftrightarrow{}$}
\newcommand{\mlangle}{{$\langle$}}
\newcommand{\mrangle}{{$\rangle$}}
\newcommand{\mldots}{{$\ldots$}}
\newcommand{\mcdots}{{$\cdots$}}
%\newcommand{\rhd}{\triangleright}
\newcommand{\mrhd}{{$\triangleright$}}
\newcommand{\mRHD}{{$\triangleright_{\!R}\,$}}
\newcommand{\Sequent}[2]{{${#1}\,\vdash{}\,{#2}$}}

\newcommand{\typingRuleThree}[4]{{\begin{tabular}{c}{${#1}$} \hspace{2em} {${#2}$} \hspace{2em} {${#3}$}\\\hline{${#4}$}\end{tabular}}}
\newcommand{\typingRuleTwo}[3]{{\begin{tabular}{c}{${#1}$} \hspace{2em} {${#2}$}\\\hline{${#3}$}\end{tabular}}}
\newcommand{\typingRule}[2]{{\begin{tabular}{c}{${#1}$}{}\\\hline{${#2}$}{}\end{tabular}}}


\newcommand{\newSequentRule}[2]{{\begin{tabular}{c}{${#1}$}{}\\\hline{${#2}$}{}\end{tabular}}}

\newcommand{\sequentRule}[4]{\mbox{\raisebox{-.4ex}[4.25ex]{\begin{tabular}{rcl} \rule{0mm}{2.85ex}{$#1$} & {$\vdash$} & ${#2}$ \\\hline {\rule{0mm}{2.65ex}$#3$} & {$\vdash$} & {$#4$}\end{tabular}\\}}}
\newcommand{\AxiomRule}[3]{{\begin{tabular}{rcl} $\,$ \\\hline {\rule{0mm}{2.65ex}$#1$} & {$\vdash$} & {$#2$}\end{tabular}}(#3)}

\newcommand{\startProof}[1]{{\begin{tabular}{c} $\,$ \\\hline {\rule{0mm}{2.65ex}{#1}} \end{tabular}}}

\newcommand{\SequentRule}[5]{{\begin{tabular}{rcl} \rule{0mm}{2.85ex}{$#3$} & {$\vdash$} & ${#4}$ \\\hline {\rule{0mm}{2.65ex}$#1$} & {$\vdash$} & {$#2$}\end{tabular}}(#5)}

\newcommand{\SequentRuleTwo}[7]{{\begin{tabular}{lr} \Sequent{#3}{#4}\ &\ \Sequent{#5}{#6} \\\hline%
\multicolumn{2}{c}{{\rule{0mm}{2.65ex}{\Sequent{#1}{#2}}}} \end{tabular}}({#7})}


\newcommand{\cSequentRuleTwo}[7]{{\begin{tabular}{lr} \Sequent{${#1}$}{${#2}$}\ &\ \Sequent{${#3}$}{${#4}$} \\\hline%
\multicolumn{2}{c}{{\rule{0mm}{2.65ex}{\Sequent{${#5}$}{${#6}$}}}} \end{tabular}}({#7})}

\newcommand{\false}{\bf{false}}
\newcommand{\definedAs}{\;{\stackrel{\rm def}{=}}\;}
\newcommand{\pair}[1]{{\langle{#1}\rangle}}

\newcommand{\ttrue}{{\bf{T}}}
\newcommand{\ffalse}{{\bf{F}}}


% \renewcommand{\inr}[1]{{\tt{inr}({#1})}}
% \renewcommand{\inl}[1]{{\tt{inl}({#1})}}
% \renewcommand{\spread}[1]{{\tt{spread}}{#1}}
% \renewcommand{\decide}[1]{{\tt{decide}}({#1})}
% \renewcommand{\void}{{\tt{void}}}
% \renewcommand{\any}[1]{{\tt{any}{#1}}}
\newcommand{\Zero}{{\mathit{Zero}}}
\newcommand{\Succ}{{\mathit{Succ\,}}}
% \renewcommand{\ind}[1]{{\tt{ind}}{#1}}
% \renewcommand{\N}{\bf{N}}
% \renewcommand{\xxx}{\mbox{\hspace{.125in}}}
% \renewcommand{\Tpp}{${\bf{T}}^{++}$}
% \renewcommand{\TPP}{{\bf{T}}^{++}}
% \renewcommand{\GT}{{\bf{T}}}
% \newcommand{\abit}{$\,$}

\newcommand{\mysubtitle}[1]{{\ \\*[-.75em]{\bf{{#1}}}}}

{\obeyspaces\global\let =\ }

\newenvironment{bogustabbing}{\begin{tabbing}\={\mbox{\hspace{10em}}}\=\=\=\kill}%
{\end{tabbing}}


\newenvironment{program}{\tt\obeyspaces\begin{bogustabbing}\+\kill}{\end{bogustabbing}}
\newenvironment{program*}{\tt\obeyspaces\begin{bogustabbing}}{\end{bogustabbing}}
\newenvironment{program**}{\it\obeyspaces\begin{bogustabbing}}{\end{bogustabbing}}
\newenvironment{smallprogram*}{\hspace{2.5em}\small \it\obeyspaces\begin{bogustabbing}}{\end{bogustabbing}\vspace{-.0625in}}

\newcommand{\CASE}{\noindent{\bf{Case}}}
\newcommand{\expr}{{{\cal{E}}}}
\newcommand{\comm}{{{\cal{C}}}}
\newcommand{\mylet}[3]{{\tt{let\ }}{#1}{\tt{\ =\ }}{#2}{\tt{\ in\ }}{#3}}
\newcommand{\myfun}[2]{{\tt{fun\ }}{#1}{\tt{\ =\ }}{#2}}
\newcommand{\semwhile}[2]{{\tt{while\ }}{#1}{\tt{\ do\ }}{#2}{\tt{\ od:\,}}comm}
\newcommand{\while}[2]{{\tt{while\ }}{#1}{\tt{\ do\ }}{#2}{\tt{\ od}}}
\newcommand{\Seq}[2]{{#1}{\bf{\,;\,}}{#2}{\tt{:\,}}comm}
\newcommand{\semif}[4]{{\tt{if(}}{\Meaning{#1}({#4})}{\tt{,\ }}{\Meaning{#2}({#4})}{\tt{,\ }}{#3}{\tt{)}}}
\newcommand{\wif}[3]{{\tt{if}}{{#1}}{\tt{\ then\ }}{{#2}}{\tt{\ else\ }}{#3}{\tt{\ fi}}:\,comm}
\newcommand{\wnot}[1]{\neg{#1}}
\newcommand{\wassign}[2]{{\tt{loc}}_{#1} := {#2}}
\newcommand{\wderef}[1]{{\tt{@loc}}_{#1} }
\newcommand{\loc}[1]{${\tt{loc}}_{#1}$}
% \newcommand{\semif1}[4]{{\tt{if(}}{\Meaning{#1}({#4})}{\tt{,\ }}{\Meaning{#2}({#4})}{\tt{,\ }}{\Meaning{#3}({#4})}{\tt{)}}}

\newcommand{\mkleeneopen}{{\boldmath{^{\lceil}}}}
\newcommand{\mkleeneclose}{{\boldmath{^{\rceil}}}}
\newcommand{\kquote}[1]{{\mkleeneopen{}{{#1}}\mkleeneclose}}

\newcommand{\homework}[2]{\ \\\vspace{-1.25in}\\%
{\bf{HW {#1}}} \hfill {\bf{Prof. Caldwell}} \\%
{\bf{Due:}} {#2} 2013 \hfill {\bf{COSC 3015}}\ \\}

\newcommand{\ite}[3]{{\bf{if\ }}{#1}{\bf{\ then\ }}{#2}{\bf{\ else\ }}{#3}{\bf{\ fi}}}
\newcommand{\fun}{{\bf{fun\, }}}
\newcommand{\prog}[3]{{#1}{\bf{\ in \ }}{#2}}



\newcommand{\store}[1]{\langle{}{#1}\rangle}

\newcommand{\alphaeq}{\,{=_\alpha}\,}
\newcommand{\xleaf}{{\rm{Leaf}}}
\newcommand{\xnode}{{\rm{Node}}}
\newcommand{\vbar}{{\;\;{\bf{|}}\;\;}}
\newcommand{\abit}{{\mbox{\hspace{1em}}}}


\newcommand{\assign}[2]{{{#1} {\bf{:=}} {#2}}}
\newcommand{\sequence}[2]{{#1}\,{\bf{;}}\,{#2}}

\newcommand{\F}[1]{{\cal{F}}_{#1}}
\newcommand{\harrow}{{\tt{-\!\!\!>}}}
\newcommand{\smallsection}{{\goodbreak\[ \star {\hspace{.35in}} \star {\hspace{.35in}} \star \] \ \\}}
\newcommand{\append}{\,{\texttt{++}}\,}
\newcommand{\nil}{{\texttt{[{\hspace{.125em}}]}}}

\newcommand{\defof}[1]{\stackrel{<\!\!<{\textrm{def. of }}{#1}>\!\!>}{=}}
\newcommand{\pnote}[1]{{$\langle\!\langle$ {#1} $\rangle\!\rangle$}}

\newcommand{\spread}[3]{{\mathit{spread}({#1};{#2}.{#3})}}


\newcommand{\sfa}{{\sf{a}}}
\newcommand{\sfb}{{\sf{b}}}
\newcommand{\sfc}{{\sf{c}}}
\newcommand{\sfd}{{\sf{d}}}
\newcommand{\sfe}{{\sf{e}}}
\newcommand{\sff}{{\sf{f}}}


\newcommand{\append}{\,{\texttt{++}}\,}
\newcommand{\nil}{{\texttt{[{\hspace{.125em}}]}}}
\usepackage{remark}
 \newremark{theorem}{Theorem}[section]
 \newremark{conjecture}{Conjecture}[section]
 \newremark{corollary}{Corollary}[section]
 \newremark{example}{Example}[section]
 \newremark{fact}{Fact}[section]
 \newremark{lemma}{Lemma}[section]
 \newremark{definition}{Definition}[section]
 \newremark{claim}{Claim}[section]
 \newremark{remark}{Remark}[section]
 \newremark{slogan}{Slogan}[section]
 \newremark{axiom}{Axiom}[section]
 \newremark{note}{Note}[]
 \newremark{problem}{Problem}[section]
 \newremark{exercise}{Exercise}[chapter]


\title{Remarks on the failed proof of \\$take\; k\; xs \append drop\; k\; xs = xs$}
\author{James Caldwell}
\begin{document}
\maketitle

In Hudak's text {\em{The Haskell School of Expression: Learning Functional
Programming through Multimedia}} \cite{hudak} the following property is claimed
to hold for all infinite lists.

\[take\, k\, xs \append{}drop\, k\, xs\,  = xs\]

The definitions of append ($\append$)\cite[pp.330]{hudak}, {\it{take}}
\cite[pp. 324]{hudak} and {\it{drop}} \cite[pp.324]{hudak} are given as
follows:


\[\begin{array}{lcl}
 \append{} &  :: & [a] \rightarrow [a] \rightarrow [a] \\
 \nil \append{} ys  & = &  ys \\
 (x:xs) \append{} ys  & = &  x:(xs\append{}ys) \\
\ \\
 take &  :: & Int \rightarrow [a] \rightarrow [a] \\
 take\; 0\; \_ & = &  \nil \\
 take\; \_ \;\nil & = &  \nil \\
 take\, k\, (x:xs)\, |\, k > 0 & = &  x:take\, (k-1)\, xs \\
 take\; \_ \;\_ & = &  error  \\
\ \\
 drop &  :: & Int \rightarrow [a] \rightarrow [a] \\
 drop\; 0\; \_ & = &  \nil \\
 drop\; \_ \;\nil & = &  \nil \\
 drop\, k\, (\_:xs)\, |\, k > 0 & = &  drop\, (k-1)\, xs \\
 drop\; \_ \;\_ & = &  error  \\
\end{array}\]

On the errata sheet, the definition of {\it{drop}} is modified as follows to
make the base case of the proof work when $k=0$.
\[\begin{array}{lcl}
 drop &  :: & Int \rightarrow [a] \rightarrow [a] \\
 drop\; 0\; xs & = &  xs \\
 drop\; \_ \;\nil & = &  \nil \\
 drop\, k\, (\_:xs)\, |\, k > 0 & = &  drop\, (k-1)\, xs \\

\end{array}\]


\section{In the book}

The method of proof by induction for infinite lists as presented in chapter 14
says \em
\begin{quote}
``... to prove a property $P$, expressed as an equation in Haskell, is true of
all infinite lists, we proceed in two steps.
\begin{description}
\item{1.} Prove $P(\bot)$ (base case)
\item{2.} Assume that $P(xs)$ is true (the induction hypothesis), and prove that $P(x:xs)$ is true (the induction step).''

\end{description}

\end{quote}
\rm

Without elaboration with quantifiers, the base case of the proof of the
following equality is presented in the book (pp. 204) and is corrected on the
errata sheet \cite{hudak_errata}.  The claim is made in the text that the proof
of the induction step is straightforward (and thus omitted).

\begin{theorem}
$take\; k\; xs \append drop\; k\; xs = xs$
\end{theorem}
\paragraph{Proof:} By induction on the infinite list $xs$. There are two cases.
\\
\noindent{\bf{(base case:)}} Show  $take\; k\; \bot \append drop\; k\; \bot = \bot$
Now, either $k=0$ or not. \\
If $k=0$ then:
\[\begin{array}{l}
take \; 0 \; \bot \append drop\; 0\; \bot   \\
\;\;\Longrightarrow \nil \append{} drop \; 0 \; \bot \\
\;\;\Longrightarrow drop \; 0 \; \bot \\
\;\;\Longrightarrow \bot
\end{array}\]
If $k \not= 0$ then we have the following computation.
\[\begin{array}{l}
take \; k \; \bot \append drop\; k\; \bot \\
\;\;\Longrightarrow \bot \append{} drop \; k \; \bot \\
\;\;\Longrightarrow \bot
\end{array}\]

\noindent{\bf{(Induction step:)}} Assume  the induction hypothesis:
\[{\bf{(I.H.)}} \;\; take\; k\; xs \append drop\; k\; xs = xs \]
and show
\[take\; k\; (x:xs) \append drop\; k\; (x:xs) = (x:xs) \]
Now, since the type of {\it{take}} and {\it{drop}} are $Int \rightarrow [a]
\rightarrow [a]$ we know $k::Int$ We proceed by cases on $k$: either $k<0$ or
$k=0$ or $k>0$.\\
\noindent{\bf{case 1.}}($k = 0$)
\[\begin{array}{l}
take\; 0\; (x:xs) \append drop\; 0\; (x:xs)\\
\;\;\Rightarrow [] \append drop \; 0\; (x:xs) \\
\;\;\Rightarrow drop \; 0\; (x:xs) \\
\;\;\Rightarrow (x:xs) \\
\end{array}\]


\noindent{\bf{case 2.}}($k > 0$)
\[\begin{array}{l}
take\; k\; (x:xs) \append drop\; k\; (x:xs) \\
\;\;\Rightarrow x:(take\; (k-1)\; (xs)) \append drop \; k\; (x:xs) \\
\;\;\Rightarrow x:(take\; (k-1)\; (xs)) \append drop \; (k-1)\; xs \\
\;\;\Rightarrow x:(take\; (k-1)\; (xs) \append drop \; (k-1)\; xs) \\
\end{array}\]
At this point in the proof we would like to invoke the induction hypothesis
{\bf{(I.H.)}} bu we can not match $k$ with $k-1$ and so the substitution fails.
We are stuck.  We need to state the theorem so that we are claiming this is
true for all $k$.

\noindent{\bf{case 3.}}($k < 0$)
\[\begin{array}{l}
take\; k\; (x:xs) \append drop\; k\; (x:xs) \\
\;\;\Rightarrow \bot \append drop \; k\; (x:xs) \\
\;\;\Rightarrow \bot
\end{array}\]
So this branch of the proof has failed as well.  Interestingly,
in Hugs, negative values of $k$ return the empty list.

\ \\ 

The inductive step of the proof has failed for two reasons. First: the
unquantified induction hypothesis can not be instantiated with $(k -1)$ and;
second: when $k < 0$ the proof fails with {\it{take}} (and {\it{drop}}) as
defined in the book -- but not as defined in the Haskell-98 prelude.



\section{Restatement of the theorem}

We restate the theorem by universally quantifying over the free variables in
the equation.  Instead of restricting the properties provable by induction on
infinite lists to simply be an equation in Haskell, we restrict the properties
provable in this way to be properties of the form $\forall{}\bar{x}:\bar{T}.E$
where $E$ is an equation in Haskell.

 Then, to prove $\forall{}xs:[a].P(xs)$, where $xs$ is an infinite list and $P$ is a property of the appropriate form, prove two things.
\begin{description}
\item{{\bf{Base case:}}} $P(\bot)$
\item{{\bf{Induction step:}}} Assume $P(xs)$ for arbitrary list $xs$ and show
$P(x:xs)$.

\end{description} 



We redefine {\it{take}} and {\it{drop}} so that they return the empty list if
the argument is negative.

\[\begin{array}{lcl}
 take &  :: & Int \rightarrow [a] \rightarrow [a] \\
 take\; k\; \_ \, |\, k \le 0 & = &  \nil \\
 take\; \_ \;\nil & = &  \nil \\
 take\, k\, (x:xs)\, |\, k > 0 & = &  x:take\, (k-1)\, xs \\
\ \\
 drop &  :: & Int \rightarrow [a] \rightarrow [a] \\
 drop\; k\; xs \,|\, k \le 0  & = &  xs \\
 drop\; \_ \;\nil & = &  \nil \\
 drop\, k\, (\_:xs)\, |\, k > 0 & = &  drop\, (k-1)\, xs \\

\end{array}\]
Note that these are the defintions in the Haskell prelude.

\ \\

Here is the restated theorem using the universal quantifiers.

\begin{theorem}
$\forall{}xs:[a].\,\forall{}k:Int.\;\;take\; k\; xs \append drop\; k\; xs = xs$
\end{theorem}

\exercise{Prove this theorem by induction on infinite lists.}


\bibliographystyle{plain}
\bibliography{./note}

\end{document}
% Local Variables:
% mode:latex
% comment-column:0
% comment-start: "% "
% compile-command: "latex hudak_error; bibtex hudak_error; dvips hudak_error -o hudak_error.ps; pdflatex hudak_error"
% fill-column:79
% End:

