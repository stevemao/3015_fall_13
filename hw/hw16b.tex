\documentclass[11pt]{article}

\usepackage{amssymb}
\usepackage{remark}
% \usepackage{html}


\usepackage{/home/faculty/jlc/papers/tex-stuff/bussproofs}



\setlength{\textwidth}{6.5in}
\setlength{\textheight}{8.8in}
\setlength{\oddsidemargin}{0.0in}
\setlength{\evensidemargin}{0.0in}


 \newremark{theorem}{Theorem}[section]
 \newremark{Rule}{Rule}[section]
 \newremark{conjecture}{Conjecture}[section]
 \newremark{corollary}{Corollary}[section]
 \newremark{example}{Example}[section]
 \newremark{fact}{Fact}[section]
 \newremark{lemma}{Lemma}[section]
 \newremark{definition}{Definition}[section]
 \newremark{claim}{Claim}[section]
 \newremark{remark}{Remark}[section]
 \newremark{slogan}{Slogan}[section]
 \newremark{axiom}{Axiom}[section]
 \newremark{note}{Note}[]
 \newremark{problem}{Problem}[section]
 \newremark{exercise}{Exercise}[section]

\newcommand{\Proof}{{\goodbreak\noindent\bf{Proof:\ }}}
\newcommand{\qed}{\goodbreak\noindent$\Box$}

\newcommand{\byDef}[1]{\stackrel{\langle\langle{{\rm{def.\ of\ }}{#1}}\rangle\rangle}{=}}
\newcommand{\byEq}[1]{\stackrel{\langle\langle{{#1}}\rangle\rangle}{=}}
\newcommand{\by}[1]{{\small{\langle\langle{\mathit{by.\;def.\;of\;}}{#1}\rangle\rangle}}}


\newcommand{\dotcup}{{\cup\hspace{-.5em}\cdot}}
\newcommand{\barcup}{{\cup\hspace{-.68em}-}}


\newcommand{\arrow}{$\rightarrow$}
\newcommand{\nat}{{\mathbb{N}}}
\newcommand{\bool}{{\mathbb{B}}}
\newcommand{\Int}{{\mathbb{Z}}}

\newcommand{\nand}{{\overline{\wedge}}}
\newcommand{\nor}{{\overline{\vee}}}


\newcommand{\Meaning}[1]{{[{\hspace{-.13em}}[{#1}]{\hspace{-.13em}}]}}
\newcommand{\Meaningof}[2]{{[{\hspace{-.13em}}[\,{#2:#1}\,]{\hspace{-.13em}}]}}
\newcommand{\Skip}{{\bf{skip}}}
\newcommand{\Let}[2]{{\bf{Let\ }} {#1}{\bf{\ in\ }}{#2}{\bf{\ end}}}
\newcommand{\A}{{\tt{a}}}
% \newcommand{\F}{{\tt{F}}}
\newcommand{\ua}[1]{{\tt{(#1 under a)}}}
\newcommand{\Fua}{\ua{F}}
\newcommand{\aF}{{\tt{(a F)}}}
\newcommand{\Btt}{{\tt{Btt}}}
\newcommand{\Bff}{{\tt{Bff}}}
%\newcommand{\T3}{{\tt{T3}}}
%\newcommand{\F3}{{\tt{F3}}}
%\newcommand{\?3}{{\tt{?3}}}
\newcommand{\V}[1]{{\mkleeneopen{}{\tt{#1}}\mkleeneclose{}}}
\newcommand{\Neg}[1]{{\tt{N(#1)}}}
\newcommand{\I}[2]{{\tt{(#1)I(#2)}}}
\newcommand{\ie}{{\em{i.e.}}}
\newcommand{\eg}{{\em{e.g.}}}
\newcommand{\sat}[2]{{\tt{#1 \mbar{}= #2}}}
\newcommand{\nsat}[2]{{\tt{#1 \mbar{}\mneq{} #2}}}
\newcommand{\Three}{{\mBbbN{}\msubthree{}}}
%\newcommand{\Three}{{\bf{3}}}
\newcommand{\Tzero}{${\tt{0}}_{\tt{3}}$}
\newcommand{\Tone}{${\tt{1}}_{\tt{3}}$}
\newcommand{\Ttwo}{${\tt{2}}_{\tt{3}}$}
\newcommand{\msubst}{{\tt{/}}}
\newcommand{\nsequent}{{\tt{>>}}}

\newcommand{\kleene}[1]{{\mkleeneopen{}#1\mkleeneclose{}}}
\newcommand{\mfvar}[1]{{\kleene{{\tt{#1}}}}}
\newcommand{\mfnot}{{\kleene{\msim}}}
\newcommand{\mfand}{{\kleene{\mwedge}}}
\newcommand{\mfor}{{\kleene{\mvee}}}
\newcommand{\mfimp}{{\kleene{\mRightarrow}}}
\newcommand{\Knot}{$\sim_K\,$}
\newcommand{\Kand}{$\wedge_K$}
\newcommand{\Kor}{$\vee_K$}
\newcommand{\Kimp}{$\Rightarrow_K$}
\newcommand{\miff}{$\Leftrightarrow{}$}
\newcommand{\mlangle}{{$\langle$}}
\newcommand{\mrangle}{{$\rangle$}}
\newcommand{\mldots}{{$\ldots$}}
\newcommand{\mcdots}{{$\cdots$}}
%\newcommand{\rhd}{\triangleright}
\newcommand{\mrhd}{{$\triangleright$}}
\newcommand{\mRHD}{{$\triangleright_{\!R}\,$}}
\newcommand{\Sequent}[2]{{${#1}\,\vdash{}\,{#2}$}}

\newcommand{\typingRuleThree}[4]{{\begin{tabular}{c}{${#1}$} \hspace{2em} {${#2}$} \hspace{2em} {${#3}$}\\\hline{${#4}$}\end{tabular}}}
\newcommand{\typingRuleTwo}[3]{{\begin{tabular}{c}{${#1}$} \hspace{2em} {${#2}$}\\\hline{${#3}$}\end{tabular}}}
\newcommand{\typingRule}[2]{{\begin{tabular}{c}{${#1}$}{}\\\hline{${#2}$}{}\end{tabular}}}


\newcommand{\newSequentRule}[2]{{\begin{tabular}{c}{${#1}$}{}\\\hline{${#2}$}{}\end{tabular}}}

\newcommand{\sequentRule}[4]{\mbox{\raisebox{-.4ex}[4.25ex]{\begin{tabular}{rcl} \rule{0mm}{2.85ex}{$#1$} & {$\vdash$} & ${#2}$ \\\hline {\rule{0mm}{2.65ex}$#3$} & {$\vdash$} & {$#4$}\end{tabular}\\}}}
\newcommand{\AxiomRule}[3]{{\begin{tabular}{rcl} $\,$ \\\hline {\rule{0mm}{2.65ex}$#1$} & {$\vdash$} & {$#2$}\end{tabular}}(#3)}

\newcommand{\startProof}[1]{{\begin{tabular}{c} $\,$ \\\hline {\rule{0mm}{2.65ex}{#1}} \end{tabular}}}

\newcommand{\SequentRule}[5]{{\begin{tabular}{rcl} \rule{0mm}{2.85ex}{$#3$} & {$\vdash$} & ${#4}$ \\\hline {\rule{0mm}{2.65ex}$#1$} & {$\vdash$} & {$#2$}\end{tabular}}(#5)}

\newcommand{\SequentRuleTwo}[7]{{\begin{tabular}{lr} \Sequent{#3}{#4}\ &\ \Sequent{#5}{#6} \\\hline%
\multicolumn{2}{c}{{\rule{0mm}{2.65ex}{\Sequent{#1}{#2}}}} \end{tabular}}({#7})}


\newcommand{\cSequentRuleTwo}[7]{{\begin{tabular}{lr} \Sequent{${#1}$}{${#2}$}\ &\ \Sequent{${#3}$}{${#4}$} \\\hline%
\multicolumn{2}{c}{{\rule{0mm}{2.65ex}{\Sequent{${#5}$}{${#6}$}}}} \end{tabular}}({#7})}

\newcommand{\false}{\bf{false}}
\newcommand{\definedAs}{\;{\stackrel{\rm def}{=}}\;}
\newcommand{\pair}[1]{{\langle{#1}\rangle}}

\newcommand{\ttrue}{{\bf{T}}}
\newcommand{\ffalse}{{\bf{F}}}


% \renewcommand{\inr}[1]{{\tt{inr}({#1})}}
% \renewcommand{\inl}[1]{{\tt{inl}({#1})}}
% \renewcommand{\spread}[1]{{\tt{spread}}{#1}}
% \renewcommand{\decide}[1]{{\tt{decide}}({#1})}
% \renewcommand{\void}{{\tt{void}}}
% \renewcommand{\any}[1]{{\tt{any}{#1}}}
\newcommand{\Zero}{{\mathit{Zero}}}
\newcommand{\Succ}{{\mathit{Succ\,}}}
% \renewcommand{\ind}[1]{{\tt{ind}}{#1}}
% \renewcommand{\N}{\bf{N}}
% \renewcommand{\xxx}{\mbox{\hspace{.125in}}}
% \renewcommand{\Tpp}{${\bf{T}}^{++}$}
% \renewcommand{\TPP}{{\bf{T}}^{++}}
% \renewcommand{\GT}{{\bf{T}}}
% \newcommand{\abit}{$\,$}

\newcommand{\mysubtitle}[1]{{\ \\*[-.75em]{\bf{{#1}}}}}

{\obeyspaces\global\let =\ }

\newenvironment{bogustabbing}{\begin{tabbing}\={\mbox{\hspace{10em}}}\=\=\=\kill}%
{\end{tabbing}}


\newenvironment{program}{\tt\obeyspaces\begin{bogustabbing}\+\kill}{\end{bogustabbing}}
\newenvironment{program*}{\tt\obeyspaces\begin{bogustabbing}}{\end{bogustabbing}}
\newenvironment{program**}{\it\obeyspaces\begin{bogustabbing}}{\end{bogustabbing}}
\newenvironment{smallprogram*}{\hspace{2.5em}\small \it\obeyspaces\begin{bogustabbing}}{\end{bogustabbing}\vspace{-.0625in}}

\newcommand{\CASE}{\noindent{\bf{Case}}}
\newcommand{\expr}{{{\cal{E}}}}
\newcommand{\comm}{{{\cal{C}}}}
\newcommand{\mylet}[3]{{\tt{let\ }}{#1}{\tt{\ =\ }}{#2}{\tt{\ in\ }}{#3}}
\newcommand{\myfun}[2]{{\tt{fun\ }}{#1}{\tt{\ =\ }}{#2}}
\newcommand{\semwhile}[2]{{\tt{while\ }}{#1}{\tt{\ do\ }}{#2}{\tt{\ od:\,}}comm}
\newcommand{\while}[2]{{\tt{while\ }}{#1}{\tt{\ do\ }}{#2}{\tt{\ od}}}
\newcommand{\Seq}[2]{{#1}{\bf{\,;\,}}{#2}{\tt{:\,}}comm}
\newcommand{\semif}[4]{{\tt{if(}}{\Meaning{#1}({#4})}{\tt{,\ }}{\Meaning{#2}({#4})}{\tt{,\ }}{#3}{\tt{)}}}
\newcommand{\wif}[3]{{\tt{if}}{{#1}}{\tt{\ then\ }}{{#2}}{\tt{\ else\ }}{#3}{\tt{\ fi}}:\,comm}
\newcommand{\wnot}[1]{\neg{#1}}
\newcommand{\wassign}[2]{{\tt{loc}}_{#1} := {#2}}
\newcommand{\wderef}[1]{{\tt{@loc}}_{#1} }
\newcommand{\loc}[1]{${\tt{loc}}_{#1}$}
% \newcommand{\semif1}[4]{{\tt{if(}}{\Meaning{#1}({#4})}{\tt{,\ }}{\Meaning{#2}({#4})}{\tt{,\ }}{\Meaning{#3}({#4})}{\tt{)}}}

\newcommand{\mkleeneopen}{{\boldmath{^{\lceil}}}}
\newcommand{\mkleeneclose}{{\boldmath{^{\rceil}}}}
\newcommand{\kquote}[1]{{\mkleeneopen{}{{#1}}\mkleeneclose}}

\newcommand{\homework}[2]{\ \\\vspace{-1.25in}\\%
{\bf{HW {#1}}} \hfill {\bf{Prof. Caldwell}} \\%
{\bf{Due:}} {#2} 2013 \hfill {\bf{COSC 3015}}\ \\}

\newcommand{\ite}[3]{{\bf{if\ }}{#1}{\bf{\ then\ }}{#2}{\bf{\ else\ }}{#3}{\bf{\ fi}}}
\newcommand{\fun}{{\bf{fun\, }}}
\newcommand{\prog}[3]{{#1}{\bf{\ in \ }}{#2}}



\newcommand{\store}[1]{\langle{}{#1}\rangle}

\newcommand{\alphaeq}{\,{=_\alpha}\,}
\newcommand{\xleaf}{{\rm{Leaf}}}
\newcommand{\xnode}{{\rm{Node}}}
\newcommand{\vbar}{{\;\;{\bf{|}}\;\;}}
\newcommand{\abit}{{\mbox{\hspace{1em}}}}


\newcommand{\assign}[2]{{{#1} {\bf{:=}} {#2}}}
\newcommand{\sequence}[2]{{#1}\,{\bf{;}}\,{#2}}

\newcommand{\F}[1]{{\cal{F}}_{#1}}
\newcommand{\harrow}{{\tt{-\!\!\!>}}}
\newcommand{\smallsection}{{\goodbreak\[ \star {\hspace{.35in}} \star {\hspace{.35in}} \star \] \ \\}}
\newcommand{\append}{\,{\texttt{++}}\,}
\newcommand{\nil}{{\texttt{[{\hspace{.125em}}]}}}

\newcommand{\defof}[1]{\stackrel{<\!\!<{\textrm{def. of }}{#1}>\!\!>}{=}}
\newcommand{\pnote}[1]{{$\langle\!\langle$ {#1} $\rangle\!\rangle$}}

\newcommand{\spread}[3]{{\mathit{spread}({#1};{#2}.{#3})}}


\newcommand{\sfa}{{\sf{a}}}
\newcommand{\sfb}{{\sf{b}}}
\newcommand{\sfc}{{\sf{c}}}
\newcommand{\sfd}{{\sf{d}}}
\newcommand{\sfe}{{\sf{e}}}
\newcommand{\sff}{{\sf{f}}}


\begin{document}
\homework{16}{6 November}

\section{Substitution}

Here is  some of the code given in class.

\begin{program*}
\>  data Term = V String  \\
\>           | Ap Term Term  \\
\>           | Abs String Term  \\
\>           | Pair Term Term \\
\>           | Spread Term (String,String) Term \\
\>    deriving (Eq) \\
\> \\
\>  fv (V s) = [s] \\
\>  fv (Ap m n) = fv m ++ fv n \\
\>  fv (Abs x m) =   filter (/=x) (fv m) \\
\>  fv (Pair m n) = fv m ++ fv n \\
\>  fv (Spread m (x,y) n) = fv m ++ fv (Abs x (Abs y n))
\> \\
\>  fresh x m =  \\
\>     if x `elem` m then \\
\>       fresh (x ++ x) m \\
\>     else \\
\>       x \\
\> \\
\>  subst (y, n) (V s) = if y == s then n else (V s) \\
\>  subst (x, n) (Ap m k) = Ap (subst (x, n) m) (subst (x, n) k) \\
\>  subst (x, n) (Abs y m) =  \\
\>      if x == y  then \\
\>         Abs y m \\
\>      else \\
\>         if y `elem` (fv n) then \\
\>            Abs z (subst (x,n) (subst (y, V z) m)) \\
\>         else \\
\>            Abs y (subst (x,n) m) \\
\>    where z = fresh "z" (x : y : ((fv n) ++ (fv m))) \\
\>  subst (x,n) (Pair m1 m2) = Pair (subst (x,n) m1) (subst (x,n) m2) \\
\>  subst (x,n) (Spread m (z,y) m1) =  \\
\>       let m' = subst (x,n) m in \\
\>       let t = (Abs z (Abs y m1)) in \\
\>       let t' = subst (x,n) t in \\
\>          case t' of \\
\>             (Abs z' (Abs y' m1')) -> Spread m' (z,y) m1' \\
\>             \_ -> error "subst: impossible case!" \\
\> \\
\>  test f t = show t ++ " ---> " ++ show (f t)
\> \\ 
\end{program*}

The rest of the code is in the file linked to from the homework page.

There is problem with substitution on spread terms.  Consider the following evaluation.
\begin{program*}
\> Main> test (subst ("x", V "y")) (Spread (Pair (V "M") (V "N")) ("y","z") (V "x")) \\
\> "subst (x, y) spread(<M,N>;y,z.x)  -->  spread(<M,N>;y,z.y)"  
\end{program*}
In this substitution, when the {\tt{x}} is replaced by {\tt{y}}, {\tt{y}} got captured.  This is what should happen.
\begin{program*}
\>  Main> test (subst ("x", V "y")) (Spread (Pair (V "M") (V "N")) ("y","z") (V "x")) \\
\>  "subst (x, y) spread(<M,N>;y,z.x)  -->  spread(<M,N>;yy,z.y)" 
\end{program*}
In this case, the binding variable {\tt{y}} is renamed to {\tt{yy}} before the
substitution is done and changing {\tt{x}} to {\tt{y}} is now safe.  

Here's another case.
\begin{program*}
\>  Main> test (subst ("x", V "y")) \\
\>             (Spread (Pair (V "M") (V "N")) ("y","z") (Ap (V "x")(V "y"))) \\
\>  "subst (x, y) spread(<M,N>;y,z.(x y))  -->  spread(<M,N>;y,z.(y yy))" \\
\end{program*} 
It should read as follows:
\begin{program*} 
\>  Main> test (subst ("x", V "y")) \\
\>             (Spread (Pair (V "M") (V "N")) ("y","z") (Ap (V "x")(V "y"))) \\
\>  "subst (x, y) spread(<M,N>;y,z.(x y))  -->  spread(<M,N>;yy,z.(y yy))" 
\end{program*}

\exercise{Fix the {\tt{subst}} function to have the proper behavior\footnote{This is a really easy fix after you think about it for a few minutes, a very minor modification.}}
\ \\
\section{Evaluating Spread}

In class I suggested the following semantics for evaluating the spread
operator:

\begin{program*}
\>    spread'($\pair{{\tt{M1}},{\tt{M2}}}$;(x,y).N) $\longrightarrow$ (N[x:=M1])[y:=M2] 
\end{program*}

\noindent{}But under these semantics, the following bad behavior occurs.

\begin{program*}
\>  Main> test spread'  (Spread (Pair(V "y") (V "M")) ("x","y") (Ap (V "x") (V "y")))  \\
\>  "spread(<y,M>;x,y.(x y)) ---> (M M)" 
\end{program*}

\noindent{}The problem is shown in more detail with the following step-by-step evaluation:
\begin{program*}
\>  spread'  (Spread (Pair(V "y") (V "M")) ("x","y") (Ap (V "x") (V "y"))) \\
\>  $\longrightarrow$ subst ("y",V "M") (subst ("x", V"y") (Ap (V "x") (V "y")))  \\
\>  $\longrightarrow$ subst ("y",V "M") (Ap (V "y") (V "y")))  \\
\>  $\longrightarrow$ Ap (V "M") (V "M") 
\end{program*}
After substituting {\tt{y}} for {\tt{x}} it is essentially captured. This can
be fixed by writing the code that explicitly tests for the appropriate cases,
but instead we can use the code we've already written for the abstraction case
and selective application of {\tt{beta}} reduction to take care of it for us.

Here are the alternative semantics:

\begin{program*}
\>    spread($\pair{{\tt{M1}},{\tt{M2}}}$;(x,y).N) $\longrightarrow$ beta (beta ((($\lambda{}$x.$(\lambda$y.N))) M1) M2) 
\end{program*}

Defining it this way avoids having a {\tt{y}} that occurs in the free variables
of {\tt{M1}} getting substituted for when {\tt{y}} is replaced by {\tt{M2}}.  So, for example,

\begin{program*}
\>  Main> test spread  (Spread (Pair(V "y") (V "M")) ("x","y") (Ap (V "x") (V "y")))  \\
\>  "spread(<y,M>;x,y.(x y)) ---> (y M)" 
\end{program*}

\exercise{Define the {\tt{spread}} function (as we did in class for
{\tt{beta}}) using the alternative semantics and test your code to show that it
behaves properly in a number of situations.}


\end{document}


% Local Variables:
% mode:latex
% comment-column:0
% comment-start: "\\>  "
% comment-end: "\\\\"
% compile-command: "pdflatex hw16b"
% fill-column:79
% End:




