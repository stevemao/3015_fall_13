\documentclass[11pt]{article}

\usepackage{amssymb}
\usepackage{remark}
% \usepackage{html}


\usepackage{/home/faculty/jlc/papers/tex-stuff/bussproofs}



\setlength{\textwidth}{6.5in}
\setlength{\textheight}{8.8in}
\setlength{\oddsidemargin}{0.0in}
\setlength{\evensidemargin}{0.0in}


 \newremark{theorem}{Theorem}[section]
 \newremark{Rule}{Rule}[section]
 \newremark{conjecture}{Conjecture}[section]
 \newremark{corollary}{Corollary}[section]
 \newremark{example}{Example}[section]
 \newremark{fact}{Fact}[section]
 \newremark{lemma}{Lemma}[section]
 \newremark{definition}{Definition}[section]
 \newremark{claim}{Claim}[section]
 \newremark{remark}{Remark}[section]
 \newremark{slogan}{Slogan}[section]
 \newremark{axiom}{Axiom}[section]
 \newremark{note}{Note}[]
 \newremark{problem}{Problem}[section]
 \newremark{exercise}{Exercise}[section]

\newcommand{\Proof}{{\goodbreak\noindent\bf{Proof:\ }}}
\newcommand{\qed}{\goodbreak\noindent$\Box$}

\newcommand{\byDef}[1]{\stackrel{\langle\langle{{\rm{def.\ of\ }}{#1}}\rangle\rangle}{=}}
\newcommand{\byEq}[1]{\stackrel{\langle\langle{{#1}}\rangle\rangle}{=}}
\newcommand{\by}[1]{{\small{\langle\langle{\mathit{by.\;def.\;of\;}}{#1}\rangle\rangle}}}


\newcommand{\dotcup}{{\cup\hspace{-.5em}\cdot}}
\newcommand{\barcup}{{\cup\hspace{-.68em}-}}


\newcommand{\arrow}{$\rightarrow$}
\newcommand{\nat}{{\mathbb{N}}}
\newcommand{\bool}{{\mathbb{B}}}
\newcommand{\Int}{{\mathbb{Z}}}

\newcommand{\nand}{{\overline{\wedge}}}
\newcommand{\nor}{{\overline{\vee}}}


\newcommand{\Meaning}[1]{{[{\hspace{-.13em}}[{#1}]{\hspace{-.13em}}]}}
\newcommand{\Meaningof}[2]{{[{\hspace{-.13em}}[\,{#2:#1}\,]{\hspace{-.13em}}]}}
\newcommand{\Skip}{{\bf{skip}}}
\newcommand{\Let}[2]{{\bf{Let\ }} {#1}{\bf{\ in\ }}{#2}{\bf{\ end}}}
\newcommand{\A}{{\tt{a}}}
% \newcommand{\F}{{\tt{F}}}
\newcommand{\ua}[1]{{\tt{(#1 under a)}}}
\newcommand{\Fua}{\ua{F}}
\newcommand{\aF}{{\tt{(a F)}}}
\newcommand{\Btt}{{\tt{Btt}}}
\newcommand{\Bff}{{\tt{Bff}}}
%\newcommand{\T3}{{\tt{T3}}}
%\newcommand{\F3}{{\tt{F3}}}
%\newcommand{\?3}{{\tt{?3}}}
\newcommand{\V}[1]{{\mkleeneopen{}{\tt{#1}}\mkleeneclose{}}}
\newcommand{\Neg}[1]{{\tt{N(#1)}}}
\newcommand{\I}[2]{{\tt{(#1)I(#2)}}}
\newcommand{\ie}{{\em{i.e.}}}
\newcommand{\eg}{{\em{e.g.}}}
\newcommand{\sat}[2]{{\tt{#1 \mbar{}= #2}}}
\newcommand{\nsat}[2]{{\tt{#1 \mbar{}\mneq{} #2}}}
\newcommand{\Three}{{\mBbbN{}\msubthree{}}}
%\newcommand{\Three}{{\bf{3}}}
\newcommand{\Tzero}{${\tt{0}}_{\tt{3}}$}
\newcommand{\Tone}{${\tt{1}}_{\tt{3}}$}
\newcommand{\Ttwo}{${\tt{2}}_{\tt{3}}$}
\newcommand{\msubst}{{\tt{/}}}
\newcommand{\nsequent}{{\tt{>>}}}

\newcommand{\kleene}[1]{{\mkleeneopen{}#1\mkleeneclose{}}}
\newcommand{\mfvar}[1]{{\kleene{{\tt{#1}}}}}
\newcommand{\mfnot}{{\kleene{\msim}}}
\newcommand{\mfand}{{\kleene{\mwedge}}}
\newcommand{\mfor}{{\kleene{\mvee}}}
\newcommand{\mfimp}{{\kleene{\mRightarrow}}}
\newcommand{\Knot}{$\sim_K\,$}
\newcommand{\Kand}{$\wedge_K$}
\newcommand{\Kor}{$\vee_K$}
\newcommand{\Kimp}{$\Rightarrow_K$}
\newcommand{\miff}{$\Leftrightarrow{}$}
\newcommand{\mlangle}{{$\langle$}}
\newcommand{\mrangle}{{$\rangle$}}
\newcommand{\mldots}{{$\ldots$}}
\newcommand{\mcdots}{{$\cdots$}}
%\newcommand{\rhd}{\triangleright}
\newcommand{\mrhd}{{$\triangleright$}}
\newcommand{\mRHD}{{$\triangleright_{\!R}\,$}}
\newcommand{\Sequent}[2]{{${#1}\,\vdash{}\,{#2}$}}

\newcommand{\typingRuleThree}[4]{{\begin{tabular}{c}{${#1}$} \hspace{2em} {${#2}$} \hspace{2em} {${#3}$}\\\hline{${#4}$}\end{tabular}}}
\newcommand{\typingRuleTwo}[3]{{\begin{tabular}{c}{${#1}$} \hspace{2em} {${#2}$}\\\hline{${#3}$}\end{tabular}}}
\newcommand{\typingRule}[2]{{\begin{tabular}{c}{${#1}$}{}\\\hline{${#2}$}{}\end{tabular}}}


\newcommand{\newSequentRule}[2]{{\begin{tabular}{c}{${#1}$}{}\\\hline{${#2}$}{}\end{tabular}}}

\newcommand{\sequentRule}[4]{\mbox{\raisebox{-.4ex}[4.25ex]{\begin{tabular}{rcl} \rule{0mm}{2.85ex}{$#1$} & {$\vdash$} & ${#2}$ \\\hline {\rule{0mm}{2.65ex}$#3$} & {$\vdash$} & {$#4$}\end{tabular}\\}}}
\newcommand{\AxiomRule}[3]{{\begin{tabular}{rcl} $\,$ \\\hline {\rule{0mm}{2.65ex}$#1$} & {$\vdash$} & {$#2$}\end{tabular}}(#3)}

\newcommand{\startProof}[1]{{\begin{tabular}{c} $\,$ \\\hline {\rule{0mm}{2.65ex}{#1}} \end{tabular}}}

\newcommand{\SequentRule}[5]{{\begin{tabular}{rcl} \rule{0mm}{2.85ex}{$#3$} & {$\vdash$} & ${#4}$ \\\hline {\rule{0mm}{2.65ex}$#1$} & {$\vdash$} & {$#2$}\end{tabular}}(#5)}

\newcommand{\SequentRuleTwo}[7]{{\begin{tabular}{lr} \Sequent{#3}{#4}\ &\ \Sequent{#5}{#6} \\\hline%
\multicolumn{2}{c}{{\rule{0mm}{2.65ex}{\Sequent{#1}{#2}}}} \end{tabular}}({#7})}


\newcommand{\cSequentRuleTwo}[7]{{\begin{tabular}{lr} \Sequent{${#1}$}{${#2}$}\ &\ \Sequent{${#3}$}{${#4}$} \\\hline%
\multicolumn{2}{c}{{\rule{0mm}{2.65ex}{\Sequent{${#5}$}{${#6}$}}}} \end{tabular}}({#7})}

\newcommand{\false}{\bf{false}}
\newcommand{\definedAs}{\;{\stackrel{\rm def}{=}}\;}
\newcommand{\pair}[1]{{\langle{#1}\rangle}}

\newcommand{\ttrue}{{\bf{T}}}
\newcommand{\ffalse}{{\bf{F}}}


% \renewcommand{\inr}[1]{{\tt{inr}({#1})}}
% \renewcommand{\inl}[1]{{\tt{inl}({#1})}}
% \renewcommand{\spread}[1]{{\tt{spread}}{#1}}
% \renewcommand{\decide}[1]{{\tt{decide}}({#1})}
% \renewcommand{\void}{{\tt{void}}}
% \renewcommand{\any}[1]{{\tt{any}{#1}}}
\newcommand{\Zero}{{\mathit{Zero}}}
\newcommand{\Succ}{{\mathit{Succ\,}}}
% \renewcommand{\ind}[1]{{\tt{ind}}{#1}}
% \renewcommand{\N}{\bf{N}}
% \renewcommand{\xxx}{\mbox{\hspace{.125in}}}
% \renewcommand{\Tpp}{${\bf{T}}^{++}$}
% \renewcommand{\TPP}{{\bf{T}}^{++}}
% \renewcommand{\GT}{{\bf{T}}}
% \newcommand{\abit}{$\,$}

\newcommand{\mysubtitle}[1]{{\ \\*[-.75em]{\bf{{#1}}}}}

{\obeyspaces\global\let =\ }

\newenvironment{bogustabbing}{\begin{tabbing}\={\mbox{\hspace{10em}}}\=\=\=\kill}%
{\end{tabbing}}


\newenvironment{program}{\tt\obeyspaces\begin{bogustabbing}\+\kill}{\end{bogustabbing}}
\newenvironment{program*}{\tt\obeyspaces\begin{bogustabbing}}{\end{bogustabbing}}
\newenvironment{program**}{\it\obeyspaces\begin{bogustabbing}}{\end{bogustabbing}}
\newenvironment{smallprogram*}{\hspace{2.5em}\small \it\obeyspaces\begin{bogustabbing}}{\end{bogustabbing}\vspace{-.0625in}}

\newcommand{\CASE}{\noindent{\bf{Case}}}
\newcommand{\expr}{{{\cal{E}}}}
\newcommand{\comm}{{{\cal{C}}}}
\newcommand{\mylet}[3]{{\tt{let\ }}{#1}{\tt{\ =\ }}{#2}{\tt{\ in\ }}{#3}}
\newcommand{\myfun}[2]{{\tt{fun\ }}{#1}{\tt{\ =\ }}{#2}}
\newcommand{\semwhile}[2]{{\tt{while\ }}{#1}{\tt{\ do\ }}{#2}{\tt{\ od:\,}}comm}
\newcommand{\while}[2]{{\tt{while\ }}{#1}{\tt{\ do\ }}{#2}{\tt{\ od}}}
\newcommand{\Seq}[2]{{#1}{\bf{\,;\,}}{#2}{\tt{:\,}}comm}
\newcommand{\semif}[4]{{\tt{if(}}{\Meaning{#1}({#4})}{\tt{,\ }}{\Meaning{#2}({#4})}{\tt{,\ }}{#3}{\tt{)}}}
\newcommand{\wif}[3]{{\tt{if}}{{#1}}{\tt{\ then\ }}{{#2}}{\tt{\ else\ }}{#3}{\tt{\ fi}}:\,comm}
\newcommand{\wnot}[1]{\neg{#1}}
\newcommand{\wassign}[2]{{\tt{loc}}_{#1} := {#2}}
\newcommand{\wderef}[1]{{\tt{@loc}}_{#1} }
\newcommand{\loc}[1]{${\tt{loc}}_{#1}$}
% \newcommand{\semif1}[4]{{\tt{if(}}{\Meaning{#1}({#4})}{\tt{,\ }}{\Meaning{#2}({#4})}{\tt{,\ }}{\Meaning{#3}({#4})}{\tt{)}}}

\newcommand{\mkleeneopen}{{\boldmath{^{\lceil}}}}
\newcommand{\mkleeneclose}{{\boldmath{^{\rceil}}}}
\newcommand{\kquote}[1]{{\mkleeneopen{}{{#1}}\mkleeneclose}}

\newcommand{\homework}[2]{\ \\\vspace{-1.25in}\\%
{\bf{HW {#1}}} \hfill {\bf{Prof. Caldwell}} \\%
{\bf{Due:}} {#2} 2013 \hfill {\bf{COSC 3015}}\ \\}

\newcommand{\ite}[3]{{\bf{if\ }}{#1}{\bf{\ then\ }}{#2}{\bf{\ else\ }}{#3}{\bf{\ fi}}}
\newcommand{\fun}{{\bf{fun\, }}}
\newcommand{\prog}[3]{{#1}{\bf{\ in \ }}{#2}}



\newcommand{\store}[1]{\langle{}{#1}\rangle}

\newcommand{\alphaeq}{\,{=_\alpha}\,}
\newcommand{\xleaf}{{\rm{Leaf}}}
\newcommand{\xnode}{{\rm{Node}}}
\newcommand{\vbar}{{\;\;{\bf{|}}\;\;}}
\newcommand{\abit}{{\mbox{\hspace{1em}}}}


\newcommand{\assign}[2]{{{#1} {\bf{:=}} {#2}}}
\newcommand{\sequence}[2]{{#1}\,{\bf{;}}\,{#2}}

\newcommand{\F}[1]{{\cal{F}}_{#1}}
\newcommand{\harrow}{{\tt{-\!\!\!>}}}
\newcommand{\smallsection}{{\goodbreak\[ \star {\hspace{.35in}} \star {\hspace{.35in}} \star \] \ \\}}
\newcommand{\append}{\,{\texttt{++}}\,}
\newcommand{\nil}{{\texttt{[{\hspace{.125em}}]}}}

\newcommand{\defof}[1]{\stackrel{<\!\!<{\textrm{def. of }}{#1}>\!\!>}{=}}
\newcommand{\pnote}[1]{{$\langle\!\langle$ {#1} $\rangle\!\rangle$}}

\newcommand{\spread}[3]{{\mathit{spread}({#1};{#2}.{#3})}}


\newcommand{\sfa}{{\sf{a}}}
\newcommand{\sfb}{{\sf{b}}}
\newcommand{\sfc}{{\sf{c}}}
\newcommand{\sfd}{{\sf{d}}}
\newcommand{\sfe}{{\sf{e}}}
\newcommand{\sff}{{\sf{f}}}


\begin{document}
\homework{13}{20 October}


\section{Abstract Data Types in Haskell}

In class we talked about how to use the Haskell module system to implement an abstract data type.

An {\em{abstract data type}} (ADT) provides users with the type
signatures of the operations supported on the type together with an
abstract specification of the expected behavior of the type.  This
specification takes the form of a list of axioms that relate the
behaviors of the interactions of the operators with one another.  By
only allowing users to access the interface it is possible to change
an underlying implementation without having to change code that uses
the ADT.

Haskell's module system allows programmers to implement ADT's by
providing a mechanism for hiding underlying representations and
implementations and simply exporting the names and signatures of the
interface.

\begin{exercise}
Read about Modules in Bird (pp.263-264). \\
Read about modules in LYAHFGG {\tt{http://learnyouahaskell.com/modules}}.
\end{exercise}



\subsection{An Abstract Data Type of Trees}

For example, as in class we defined a data type of trees.

\subsubsection{Type Signature}

The type signatures of the tree operations was given as follows:

\begin{program**}
\>                           Tree a             \=-- the type name  \\
\>                           leaf                    \>:: a $\rightarrow$ Tree a \\
\>                           branch                  \>:: Tree a $\rightarrow$ Tree a $\rightarrow$ Tree a \\
\>                           cell                    \>:: Tree a $\rightarrow$ a \\
\>                           left, right             \>:: Tree a $\rightarrow$ Tree a \\
\>                           isLeaf                  \>:: Tree a $\rightarrow$ Bool \\
\end{program**}



\subsubsection{The Tree Axioms}

If $x::a$ and $t_1,t_2 :: (Tree\; a)$ then the following axioms specify the
behaviors of the operators.

\[\begin{array}{ll}
T1.) & cell (leaf\; x) = x\\
T2.) & cell (branch \;t_1\; t_2) = \bot\\
T3.) & left(leaf\; x) = \bot\\
T4.) & left (branch \;t_1\; t_2) = t_1\\
T5.) & right(leaf\; x) = \bot\\
T6.) & right (branch \;t_1\; t_2) = t_2\\
T7.) & isLeaf (leaf\; x) = True\\
T8.) & isLeaf (branch \;t_1\; t_2) = False\\
\end{array}\]

\subsubsection{Haskell Implementation}

A Haskell implementation is given by the following module.

\begin{program**}
\>             module TreeADT (Tree, leaf, branch, cell, left, right, isLeaf) where \\
\>             data Tree a             \== Leaf a $\mid$ Branch (Tree a) (Tree a)  deriving (Eq,Show) \\
\>             leaf                    \>= Leaf \\
\>             branch                  \>= Branch \\
\>             cell  (Leaf a)          \>= a \\
\>             left  (Branch l r)      \>= l \\
\>             right (Branch l r)      \>= r \\
\>             isLeaf   (Leaf \_)      \>= True \\
\>             isLeaf   \_             \>= False \\
\end{program**}

Note that the constructors {\it{Leaf}} and {\it{Branch}} are not in export list
following the modules name.  In the absence of an export list, all declarations
in the module are exported.  Since there is an export list in the
{\em{TreeADT}} module, definitions that are not mentioned in the list are
private.  Programs that import this module do not have access to {\it{Leaf}}
and {\it{Branch}}.  Note that if you load this module into GHCI, you
{\em{will}} have access to the {\em{Leaf}} and {\em{Branch}} constructors --
but those constructors are not in the name-space of modules that import
{\em{TreeADT}}.  Also, the deriving-clause which include the type {\it{Tree a}}
in the {\it{Eq}} and {\it{Show}} type classes make it possible to test elements
of the data type for equality and to display them.  Displaying them reveals the
names of the underlying constructors, but because they were not exported, users
of the module can not use them.

\subsubsection{Unit Tests}

Axioms can be used to design unit tests for an implementation.  This is
especially easy for axioms that are stated in the form given above -- where we
assume the varialbes $x$ and $T_1$ and $t_2$ are universally quantified in each
axiom and the axiom takes the form of an equation.  If the axioms contain
existential quantifiers the problem is more difficult.  In the {\em{TreeADT}}
axioms {\it{T1}} through {\it{T8}}, the variables $x$, $t_1$ and $t_2$ that
occur in each axiom become arguments to the test function for the behavior
specified by that axiom.  If the behavior is specified to be undefined
{\em{i.e.}} it is equal to $\bot$, then the test can be run to see if it
results in an error.  Note that we have used $\bot$ to denote both run-time
errors and looping forever. Obviously, testing if a program loops forever can
not be done unless you are extremely patient and have unlimited time on your
hands.

Here is a Haskell module implementing unit tests for the {\em{TreeADT}} module.

\begin{program**}
\>            module TestTreeADT where  \\
\>               \\
\>            import TreeADT  \\
\>              \\
\>            test\_T1 x = cell (leaf x) == x  \\
\>            -- the following test should raise an error on all inputs t1 and t2 \\
\>            test\_T2 t1 t2 = cell (branch t1 t2)   \\
\>            -- the following test should raise an error on all inputs x\\
\>            test\_T3 x = left(leaf x)   \\
\>            test\_T4 t1 t2 = left (branch t1 t2) == t1  \\
\>            -- the following test should raise an error on all inputs x\\
\>            test\_T5 x = right(leaf x)   \\
\>            test\_T6 t1 t2 = right (branch t1 t2) == t2  \\
\>            test\_T7 x = isLeaf (leaf x) == True  \\
\>            test\_T8 t1 t2 = isLeaf (branch t1 t2) == False  \\
\end{program**}

\subsection{An Abstract Data Type for Sets}

Here is a specification of a Set ADT.  Note that our sets are
{\em{monomorphic}} in the sense that they can contain elements of a single
type. 

\subsubsection{Type Signature}

\begin{program**}
\>                           data Set a           \=-- the type name  \\
\>                           empty                \>:: Set a\\
\>                           ismem                \>:: a $\rightarrow$ Set a $\rightarrow$ Bool\\
\>                           size                 \>:: Set a $\rightarrow$ Int\\
\>                           insert               \>:: a $\rightarrow$ Set a $\rightarrow$ Set a\\
\>                           delete               \>:: a $\rightarrow$ Set a $\rightarrow$ Set a\\
\>                           union                \>:: Set a $\rightarrow$ Set a $\rightarrow$ Set a\\
\>                           intersection         \>:: Set a $\rightarrow$ Set a $\rightarrow$ Set a\\
\end{program**}

\subsubsection{The Set Axioms}

If {\it{s,t :: (Set a)}} and $x,y :: a$ the following axioms specify the behavior for the Set data type.

\[\begin{array}{lrcl}
S1.) & ismem\;x\; empty &=&False\\
S2.) & ismem\;x \;(insert\; y \; s) &=& (x = y) \vee ismem\; x \; s\\
S3.) & size \; empty &=& 0 \\
S4.) & size (insert\; x \;s) &=&
    \left\{ \begin{array}{ll} 
                 size \; s &{\mathrm{if\ }} ismem\; x\;s\\
                 (size\; s) + 1 & {\mathrm{otherwise}} \\
           \end{array}\right.\vspace{.06125in}\\
S5.) & (insert\; x\; s = empty) &=& False\\
S6.) & insert\; x\; (insert\; x\; s) &=& insert\; x\; s\\
S7.) & insert\; x\; (insert\; y\; s) &=& insert\; y\; (insert\; x\; s)\\
S8.) & ismem\;x \;(delete \; y \;s ) &=& 
    \left\{ \begin{array}{ll} 
                 False &{\mathrm{if\ }} x = y\\
                 ismem\; x\; s & {\mathrm{otherwise}} \\
           \end{array}\right.\vspace{.06125in}\\
S9.) & ismem\;x \;(union \; s \; t) &=& (ismem\;x \; s\; \vee \; ismem\;x \; t)  \\
S10.) & ismem\;x \;(intersection \; s \; t) &=& (ismem\;x \; s\; \wedge \; ismem\;x \; t)  \\
\end{array}\]



\begin{exercise}
Your assignment is build on the base code provided on the course web-page to
finish an implementation of the {\it{Set}} data type using lists.  You also
need to make a module {\em{TestSet}} which implements unit tests for each set
axiom.

Note that you may need to use the list function {\it{nub :: (Eq a) $\Rightarrow$ [a] $\rightarrow$
[a]}} which eliminates duplicate elements of a list.  It requires the elements
of the list to be members of the Eq type class.

\end{exercise}



\end{document}
% Local Variables:
% mode:latex
% comment-column:0
% comment-start: "% "
% compile-command: "pdflatex hw13c"
% fill-column:79
% End:




