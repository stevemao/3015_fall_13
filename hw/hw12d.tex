\documentclass[11pt]{article}

\usepackage{amssymb}
\usepackage{remark}
% \usepackage{html}


\usepackage{/home/faculty/jlc/papers/tex-stuff/bussproofs}



\setlength{\textwidth}{6.5in}
\setlength{\textheight}{8.8in}
\setlength{\oddsidemargin}{0.0in}
\setlength{\evensidemargin}{0.0in}


 \newremark{theorem}{Theorem}[section]
 \newremark{Rule}{Rule}[section]
 \newremark{conjecture}{Conjecture}[section]
 \newremark{corollary}{Corollary}[section]
 \newremark{example}{Example}[section]
 \newremark{fact}{Fact}[section]
 \newremark{lemma}{Lemma}[section]
 \newremark{definition}{Definition}[section]
 \newremark{claim}{Claim}[section]
 \newremark{remark}{Remark}[section]
 \newremark{slogan}{Slogan}[section]
 \newremark{axiom}{Axiom}[section]
 \newremark{note}{Note}[]
 \newremark{problem}{Problem}[section]
 \newremark{exercise}{Exercise}[section]

\newcommand{\Proof}{{\goodbreak\noindent\bf{Proof:\ }}}
\newcommand{\qed}{\goodbreak\noindent$\Box$}

\newcommand{\byDef}[1]{\stackrel{\langle\langle{{\rm{def.\ of\ }}{#1}}\rangle\rangle}{=}}
\newcommand{\byEq}[1]{\stackrel{\langle\langle{{#1}}\rangle\rangle}{=}}
\newcommand{\by}[1]{{\small{\langle\langle{\mathit{by.\;def.\;of\;}}{#1}\rangle\rangle}}}


\newcommand{\dotcup}{{\cup\hspace{-.5em}\cdot}}
\newcommand{\barcup}{{\cup\hspace{-.68em}-}}


\newcommand{\arrow}{$\rightarrow$}
\newcommand{\nat}{{\mathbb{N}}}
\newcommand{\bool}{{\mathbb{B}}}
\newcommand{\Int}{{\mathbb{Z}}}

\newcommand{\nand}{{\overline{\wedge}}}
\newcommand{\nor}{{\overline{\vee}}}


\newcommand{\Meaning}[1]{{[{\hspace{-.13em}}[{#1}]{\hspace{-.13em}}]}}
\newcommand{\Meaningof}[2]{{[{\hspace{-.13em}}[\,{#2:#1}\,]{\hspace{-.13em}}]}}
\newcommand{\Skip}{{\bf{skip}}}
\newcommand{\Let}[2]{{\bf{Let\ }} {#1}{\bf{\ in\ }}{#2}{\bf{\ end}}}
\newcommand{\A}{{\tt{a}}}
% \newcommand{\F}{{\tt{F}}}
\newcommand{\ua}[1]{{\tt{(#1 under a)}}}
\newcommand{\Fua}{\ua{F}}
\newcommand{\aF}{{\tt{(a F)}}}
\newcommand{\Btt}{{\tt{Btt}}}
\newcommand{\Bff}{{\tt{Bff}}}
%\newcommand{\T3}{{\tt{T3}}}
%\newcommand{\F3}{{\tt{F3}}}
%\newcommand{\?3}{{\tt{?3}}}
\newcommand{\V}[1]{{\mkleeneopen{}{\tt{#1}}\mkleeneclose{}}}
\newcommand{\Neg}[1]{{\tt{N(#1)}}}
\newcommand{\I}[2]{{\tt{(#1)I(#2)}}}
\newcommand{\ie}{{\em{i.e.}}}
\newcommand{\eg}{{\em{e.g.}}}
\newcommand{\sat}[2]{{\tt{#1 \mbar{}= #2}}}
\newcommand{\nsat}[2]{{\tt{#1 \mbar{}\mneq{} #2}}}
\newcommand{\Three}{{\mBbbN{}\msubthree{}}}
%\newcommand{\Three}{{\bf{3}}}
\newcommand{\Tzero}{${\tt{0}}_{\tt{3}}$}
\newcommand{\Tone}{${\tt{1}}_{\tt{3}}$}
\newcommand{\Ttwo}{${\tt{2}}_{\tt{3}}$}
\newcommand{\msubst}{{\tt{/}}}
\newcommand{\nsequent}{{\tt{>>}}}

\newcommand{\kleene}[1]{{\mkleeneopen{}#1\mkleeneclose{}}}
\newcommand{\mfvar}[1]{{\kleene{{\tt{#1}}}}}
\newcommand{\mfnot}{{\kleene{\msim}}}
\newcommand{\mfand}{{\kleene{\mwedge}}}
\newcommand{\mfor}{{\kleene{\mvee}}}
\newcommand{\mfimp}{{\kleene{\mRightarrow}}}
\newcommand{\Knot}{$\sim_K\,$}
\newcommand{\Kand}{$\wedge_K$}
\newcommand{\Kor}{$\vee_K$}
\newcommand{\Kimp}{$\Rightarrow_K$}
\newcommand{\miff}{$\Leftrightarrow{}$}
\newcommand{\mlangle}{{$\langle$}}
\newcommand{\mrangle}{{$\rangle$}}
\newcommand{\mldots}{{$\ldots$}}
\newcommand{\mcdots}{{$\cdots$}}
%\newcommand{\rhd}{\triangleright}
\newcommand{\mrhd}{{$\triangleright$}}
\newcommand{\mRHD}{{$\triangleright_{\!R}\,$}}
\newcommand{\Sequent}[2]{{${#1}\,\vdash{}\,{#2}$}}

\newcommand{\typingRuleThree}[4]{{\begin{tabular}{c}{${#1}$} \hspace{2em} {${#2}$} \hspace{2em} {${#3}$}\\\hline{${#4}$}\end{tabular}}}
\newcommand{\typingRuleTwo}[3]{{\begin{tabular}{c}{${#1}$} \hspace{2em} {${#2}$}\\\hline{${#3}$}\end{tabular}}}
\newcommand{\typingRule}[2]{{\begin{tabular}{c}{${#1}$}{}\\\hline{${#2}$}{}\end{tabular}}}


\newcommand{\newSequentRule}[2]{{\begin{tabular}{c}{${#1}$}{}\\\hline{${#2}$}{}\end{tabular}}}

\newcommand{\sequentRule}[4]{\mbox{\raisebox{-.4ex}[4.25ex]{\begin{tabular}{rcl} \rule{0mm}{2.85ex}{$#1$} & {$\vdash$} & ${#2}$ \\\hline {\rule{0mm}{2.65ex}$#3$} & {$\vdash$} & {$#4$}\end{tabular}\\}}}
\newcommand{\AxiomRule}[3]{{\begin{tabular}{rcl} $\,$ \\\hline {\rule{0mm}{2.65ex}$#1$} & {$\vdash$} & {$#2$}\end{tabular}}(#3)}

\newcommand{\startProof}[1]{{\begin{tabular}{c} $\,$ \\\hline {\rule{0mm}{2.65ex}{#1}} \end{tabular}}}

\newcommand{\SequentRule}[5]{{\begin{tabular}{rcl} \rule{0mm}{2.85ex}{$#3$} & {$\vdash$} & ${#4}$ \\\hline {\rule{0mm}{2.65ex}$#1$} & {$\vdash$} & {$#2$}\end{tabular}}(#5)}

\newcommand{\SequentRuleTwo}[7]{{\begin{tabular}{lr} \Sequent{#3}{#4}\ &\ \Sequent{#5}{#6} \\\hline%
\multicolumn{2}{c}{{\rule{0mm}{2.65ex}{\Sequent{#1}{#2}}}} \end{tabular}}({#7})}


\newcommand{\cSequentRuleTwo}[7]{{\begin{tabular}{lr} \Sequent{${#1}$}{${#2}$}\ &\ \Sequent{${#3}$}{${#4}$} \\\hline%
\multicolumn{2}{c}{{\rule{0mm}{2.65ex}{\Sequent{${#5}$}{${#6}$}}}} \end{tabular}}({#7})}

\newcommand{\false}{\bf{false}}
\newcommand{\definedAs}{\;{\stackrel{\rm def}{=}}\;}
\newcommand{\pair}[1]{{\langle{#1}\rangle}}

\newcommand{\ttrue}{{\bf{T}}}
\newcommand{\ffalse}{{\bf{F}}}


% \renewcommand{\inr}[1]{{\tt{inr}({#1})}}
% \renewcommand{\inl}[1]{{\tt{inl}({#1})}}
% \renewcommand{\spread}[1]{{\tt{spread}}{#1}}
% \renewcommand{\decide}[1]{{\tt{decide}}({#1})}
% \renewcommand{\void}{{\tt{void}}}
% \renewcommand{\any}[1]{{\tt{any}{#1}}}
\newcommand{\Zero}{{\mathit{Zero}}}
\newcommand{\Succ}{{\mathit{Succ\,}}}
% \renewcommand{\ind}[1]{{\tt{ind}}{#1}}
% \renewcommand{\N}{\bf{N}}
% \renewcommand{\xxx}{\mbox{\hspace{.125in}}}
% \renewcommand{\Tpp}{${\bf{T}}^{++}$}
% \renewcommand{\TPP}{{\bf{T}}^{++}}
% \renewcommand{\GT}{{\bf{T}}}
% \newcommand{\abit}{$\,$}

\newcommand{\mysubtitle}[1]{{\ \\*[-.75em]{\bf{{#1}}}}}

{\obeyspaces\global\let =\ }

\newenvironment{bogustabbing}{\begin{tabbing}\={\mbox{\hspace{10em}}}\=\=\=\kill}%
{\end{tabbing}}


\newenvironment{program}{\tt\obeyspaces\begin{bogustabbing}\+\kill}{\end{bogustabbing}}
\newenvironment{program*}{\tt\obeyspaces\begin{bogustabbing}}{\end{bogustabbing}}
\newenvironment{program**}{\it\obeyspaces\begin{bogustabbing}}{\end{bogustabbing}}
\newenvironment{smallprogram*}{\hspace{2.5em}\small \it\obeyspaces\begin{bogustabbing}}{\end{bogustabbing}\vspace{-.0625in}}

\newcommand{\CASE}{\noindent{\bf{Case}}}
\newcommand{\expr}{{{\cal{E}}}}
\newcommand{\comm}{{{\cal{C}}}}
\newcommand{\mylet}[3]{{\tt{let\ }}{#1}{\tt{\ =\ }}{#2}{\tt{\ in\ }}{#3}}
\newcommand{\myfun}[2]{{\tt{fun\ }}{#1}{\tt{\ =\ }}{#2}}
\newcommand{\semwhile}[2]{{\tt{while\ }}{#1}{\tt{\ do\ }}{#2}{\tt{\ od:\,}}comm}
\newcommand{\while}[2]{{\tt{while\ }}{#1}{\tt{\ do\ }}{#2}{\tt{\ od}}}
\newcommand{\Seq}[2]{{#1}{\bf{\,;\,}}{#2}{\tt{:\,}}comm}
\newcommand{\semif}[4]{{\tt{if(}}{\Meaning{#1}({#4})}{\tt{,\ }}{\Meaning{#2}({#4})}{\tt{,\ }}{#3}{\tt{)}}}
\newcommand{\wif}[3]{{\tt{if}}{{#1}}{\tt{\ then\ }}{{#2}}{\tt{\ else\ }}{#3}{\tt{\ fi}}:\,comm}
\newcommand{\wnot}[1]{\neg{#1}}
\newcommand{\wassign}[2]{{\tt{loc}}_{#1} := {#2}}
\newcommand{\wderef}[1]{{\tt{@loc}}_{#1} }
\newcommand{\loc}[1]{${\tt{loc}}_{#1}$}
% \newcommand{\semif1}[4]{{\tt{if(}}{\Meaning{#1}({#4})}{\tt{,\ }}{\Meaning{#2}({#4})}{\tt{,\ }}{\Meaning{#3}({#4})}{\tt{)}}}

\newcommand{\mkleeneopen}{{\boldmath{^{\lceil}}}}
\newcommand{\mkleeneclose}{{\boldmath{^{\rceil}}}}
\newcommand{\kquote}[1]{{\mkleeneopen{}{{#1}}\mkleeneclose}}

\newcommand{\homework}[2]{\ \\\vspace{-1.25in}\\%
{\bf{HW {#1}}} \hfill {\bf{Prof. Caldwell}} \\%
{\bf{Due:}} {#2} 2013 \hfill {\bf{COSC 3015}}\ \\}

\newcommand{\ite}[3]{{\bf{if\ }}{#1}{\bf{\ then\ }}{#2}{\bf{\ else\ }}{#3}{\bf{\ fi}}}
\newcommand{\fun}{{\bf{fun\, }}}
\newcommand{\prog}[3]{{#1}{\bf{\ in \ }}{#2}}



\newcommand{\store}[1]{\langle{}{#1}\rangle}

\newcommand{\alphaeq}{\,{=_\alpha}\,}
\newcommand{\xleaf}{{\rm{Leaf}}}
\newcommand{\xnode}{{\rm{Node}}}
\newcommand{\vbar}{{\;\;{\bf{|}}\;\;}}
\newcommand{\abit}{{\mbox{\hspace{1em}}}}


\newcommand{\assign}[2]{{{#1} {\bf{:=}} {#2}}}
\newcommand{\sequence}[2]{{#1}\,{\bf{;}}\,{#2}}

\newcommand{\F}[1]{{\cal{F}}_{#1}}
\newcommand{\harrow}{{\tt{-\!\!\!>}}}
\newcommand{\smallsection}{{\goodbreak\[ \star {\hspace{.35in}} \star {\hspace{.35in}} \star \] \ \\}}
\newcommand{\append}{\,{\texttt{++}}\,}
\newcommand{\nil}{{\texttt{[{\hspace{.125em}}]}}}

\newcommand{\defof}[1]{\stackrel{<\!\!<{\textrm{def. of }}{#1}>\!\!>}{=}}
\newcommand{\pnote}[1]{{$\langle\!\langle$ {#1} $\rangle\!\rangle$}}

\newcommand{\spread}[3]{{\mathit{spread}({#1};{#2}.{#3})}}


\newcommand{\sfa}{{\sf{a}}}
\newcommand{\sfb}{{\sf{b}}}
\newcommand{\sfc}{{\sf{c}}}
\newcommand{\sfd}{{\sf{d}}}
\newcommand{\sfe}{{\sf{e}}}
\newcommand{\sff}{{\sf{f}}}


\begin{document}
\homework{12}{11 October}

\section{Unifying a list of constraints}

We represent constraints in Haskell as pair of types.

\begin{program**}
\> type Constraint = (Type,Type)
\end{program**}
We will write $\tau_1\stackrel{c}{=}\tau_2$ to denote the constraint pair
$(\tau_1,\tau_2)$.

\begin{definition}[Satisfiability of a constraint list]
Given a list of constraints of the the form $[\tau_1\stackrel{c}{=}\tau_1',
\cdots ,\tau_k\stackrel{c}{=}\tau_k')]$ we say the list is {\em{satisfiable}}
if there is a single substitution $s$ that unifies all constraints in the list.
\end{definition}


Recall that the function {\it{unify}} has type {\it{Type $\rightarrow$ Type
$\rightarrow$ Substitution}}.  building on the definitions from the previous
assignment we give the definition for a function to unify a list constraints.

\begin{program**}
\> unifyL :: [Constraint] $\rightarrow$ Substitution \\
\> unifyL xs = unifyAll xs idSubst \\
\>    where unifyAll [] ans = ans \\
\>            unifyAll ((t1,t2):cs) ans = unifyAll (map substpair cs) ((subst s) . ans) \\
\>                where s = unify t1 t2 \\
\>                        substpair (t1,t2) = (subst s t1, subst s t2) \\
\end{program**}



\section{Type Inference}

Given a term (a program) the type inference algorithm determines if there is a
type for that term and if so, what the type is. 

There are two stages.  The first stage is to build a set of constraints and the
second is to solve them (if possible) yielding a substitution which is then
applied to determine the type of the term.  The constraint building phase is
defined by a set of proof rules used to build a type derivation.  These rules
decompose the term and build a set of constraints which, if satisfiable, yields
a substitution that can be used to find the type of the term.

We will present the set of proof rules (due to Michell Wand) to build a
derivation (if one exists) which yields a list of constraints.  We define a
Haskell function which ``implements'' these rules.  We then use the function
{\it{unifyL}} to find a solution, if one exists.

\subsection{The Proof Rules}

The state of a type derivation is recorded in a structure of the following
form:

\[\Gamma,C \vdash M : T\]
In this structure, $\Gamma$ is a {\em{context}} representing a state of
knowledge about the types of some variables.  Contexts have the form:
 \[\Gamma = [x_1:\tau_1,\cdots{},x_k:\tau_k]\]
where the $x_i$'s are variables and $\tau_i$'s are types.

$C$ is a list of constraints between pairs
of types and in the rules is presented as follows:
\[C=\{\tau_1 \stackrel{c}{=}\tau_1', \cdots , \tau_k \stackrel{c}{=} \tau_k'\}\]
where $\tau_i$'s are types.  

We write $(x:\tau):\Gamma$ to denote the list obtained from $\Gamma$ by consing
the pair $(x:\tau)$ on the left end.  Haskell's {\tt{lookup}} function can be
used to find the type $\tau$ paired with the leftmost occurrence of a pair
having $x$ as its first element.

The proof rules for Wand's type inference system are given as follows:


\AxiomRule{\Gamma,\{\tau=\tau'\}}{x:\tau}{Var} {\hspace{.25in}} if $lookup \;x \;\Gamma = Just(\tau')$ 
\vspace{.125in}\\


\SequentRuleTwo{\Gamma,C_1\cup{}C_2}{MN:\tau}{\Gamma,C_1}{M:\alpha\rightarrow\tau}{\Gamma,C_2}{N:\alpha}{App}
{\hspace{.25in}} where $\alpha$ is fresh.
\vspace{.125in}\\

\SequentRule{\Gamma,\{\tau=\alpha\rightarrow\beta\}\cup{}C}{\lambda{}x.M : \tau}{(x:\alpha):\Gamma,\,C}{M:\beta}{Abs}
{\hspace{.25in}} where $\alpha$ and $\beta$ are fresh.
\vspace{.125in}\\



A type derivation in this system is a tree of instances of these rules where
the leaves of the tree are all instances of the (Ax) rule.  To construct a
derivation that a closed term (no free variables) (say $M$) has a type, we
postulate that $M$ has some type (say $\alpha$) and proceed by recursion on the
structure of $M$ to show:

\[ \exists{}C.[(Type,Type)]. {\rm{\ such \ that\ the \ sequent\ }}[],C\vdash M:\alpha {\rm{\  is \ derivable.}}\]

To find $C$, we use the proof rules above to try to construct a derivation
(leaving the $C$'s blank to start) and then propagate the constraints in the
$C$'s back down through the derivation tree from the leaves.

\begin{example}
Here is an example of a derivation that $\lambda{}x.x$ has a type by starting
with the sequent of the form  $[],\{??\}\vdash (\lambda{}x.x) :\tau$.  The term is an abstraction
so we apply the rule (Abs).

\begin{center}
\AxiomC{$[x:\alpha],C\vdash x : \beta$}
\LeftLabel{}\RightLabel{(Abs)}
\UnaryInfC{$[],\{\tau=\alpha \rightarrow \beta\}\cup{}C\vdash (\lambda{}x.x) :\tau$}
\DisplayProof
\end{center}

But if we fill in the set $C$ with the constraint $\tau=\alpha$, we have an
instance of the Axiom rule.


\begin{center}
\AxiomC{$C=\{\beta=\alpha\}$}
\LeftLabel{}\RightLabel{(Ax)}
\UnaryInfC{$[x:\alpha],C\vdash x : \beta$}
\LeftLabel{}\RightLabel{(Abs)}
\UnaryInfC{$[],\{\tau=\alpha \rightarrow \beta\}\cup{}C\vdash (\lambda{}x.x) :\tau$}
\DisplayProof
\end{center}

\noindent{}If we completely instantiate the sets $C$ we get the following complete derivation.

\begin{center}
\AxiomC{}
\LeftLabel{}\RightLabel{(Ax)}
\UnaryInfC{$[x:\alpha],\{\beta=\alpha\}\vdash x : \beta$}
\LeftLabel{}\RightLabel{(Abs)}
\UnaryInfC{$[],\{\tau=\alpha \rightarrow \beta,\beta=\alpha\}\vdash (\lambda{}x.x) :\tau$}
\DisplayProof
\end{center}

\end{example}


The fact that there is a derivation indicates that the term $(\lambda{}x.x)$
may have a type. We use the constraint set $C$ to actually determine the type
of $\lambda{}x.x$.  To do this, we unify the set $C$ and apply the resulting
substitution to the type $\tau$.  For this case, when we unify $C$ we get the
substitution $[\tau \mapsto \alpha\rightarrow\alpha, \beta \mapsto \alpha]$.  Applying this
substitution to $\tau$ we determine that
$(\lambda{}x.x):\alpha\rightarrow\alpha$.



We can also do type derivations for terms containing free variables if we
assume those free variables do have types.

\begin{example}
Consider the term $y(\lambda{}x.x)$. This should have a type if
$y:(\alpha\rightarrow\alpha)\rightarrow\beta$.

We start by trying to show there is some $C$ such that there is a derivation of the sequent

\[[y:(\alpha\rightarrow\alpha)\rightarrow\beta],C\vdash y(\lambda{}x.x):\tau\]

\noindent{}Since the term is an application, we use the (App) rule.

\begin{center}
\AxiomC{$[y:(\alpha\rightarrow\alpha)\rightarrow\beta],C_1\vdash y :\alpha'\rightarrow\tau$}
\AxiomC{$[y:(\alpha\rightarrow\alpha)\rightarrow\beta],C_2\vdash (\lambda{}x.x):\alpha'$}
\LeftLabel{}\RightLabel{(App)}
\BinaryInfC{$[y:(\alpha\rightarrow\alpha)\rightarrow\beta],C_1\cup{}C_2\vdash y(\lambda{}x.x):\tau$}
\DisplayProof
\end{center}

\noindent{}The left branch is an instance of an axiom because there is an entry for the variable $y$ in the context.

\small{
\begin{center}
\AxiomC{$C_1 = \{\alpha'\rightarrow\tau=(\alpha\rightarrow\alpha)\rightarrow\beta\}$}
\LeftLabel{}\RightLabel{(Var)}
\UnaryInfC{$[y:(\alpha\rightarrow\alpha)\rightarrow\beta],C_1\vdash y :\alpha'\rightarrow\tau$}
\AxiomC{$[y:(\alpha\rightarrow\alpha)\rightarrow\beta],C_2\vdash (\lambda{}x.x):\alpha'$}
\LeftLabel{}\RightLabel{(App)}
\BinaryInfC{$[y:(\alpha\rightarrow\alpha)\rightarrow\beta],C_1\cup{}C_2\vdash y(\lambda{}x.x):\tau$}
\DisplayProof
\end{center}
}

\noindent{}On the right branch we rebuild the proof given above.

\small{
\begin{center}
\AxiomC{$C_1 = \{\alpha'\rightarrow\tau=(\alpha\rightarrow\alpha)\rightarrow\beta\}$}
\LeftLabel{}\RightLabel{(Var)}
\UnaryInfC{$[y:(\alpha\rightarrow\alpha)\rightarrow\beta],C_1\vdash y :\alpha'\rightarrow\tau$}
% \AxiomC{$[y:(\alpha\rightarrow\alpha)\rightarrow\beta],C_2\vdash (\lambda{}x.x):\alpha'$}
\AxiomC{$C_3=\{\beta'=\alpha''\}$}
\LeftLabel{}\RightLabel{(Var)}
\UnaryInfC{$[x:\alpha'',y:(\alpha\rightarrow\alpha)\rightarrow\beta],C_3\vdash x : \beta'$}
\LeftLabel{}\RightLabel{(Abs)}
\UnaryInfC{$[y:(\alpha\rightarrow\alpha)\rightarrow\beta],C_2=(\{\alpha'=\alpha'' \rightarrow \beta'\}\cup{}C_3)\vdash (\lambda{}x.x) :\alpha'$}
\LeftLabel{}\RightLabel{(App)}
\BinaryInfC{$[y:(\alpha\rightarrow\alpha)\rightarrow\beta],C=(C_1\cup{}C_2)\vdash y(\lambda{}x.x):\tau$}
\DisplayProof
\end{center}
}

Putting together the constraints, we get the following set:

\[\begin{array}{lcl}
C & = & C_1 \cup C_2 \\
  & = & \{\alpha'\rightarrow\tau=(\alpha\rightarrow\alpha)\rightarrow\beta\} \cup (\{\alpha'=\alpha'' \rightarrow \beta'\}\cup{}C_3)\\
  & = & \{\alpha'\rightarrow\tau=(\alpha\rightarrow\alpha)\rightarrow\beta\} \cup (\{\alpha'=\alpha'' \rightarrow \beta'\}\cup{}\{\beta'=\alpha''\})\\
  & = & \{\alpha'\rightarrow\tau=(\alpha\rightarrow\alpha)\rightarrow\beta,\alpha'=\alpha'' \rightarrow \beta', \beta'=\alpha''\})\\
\end{array}\]

Unification of this results in the substitution:

\[s = [a' \mapsto (b' \rightarrow b'),\tau \mapsto b,a \mapsto b',a'' \mapsto b']\]

When $s$ is applied to $\tau$ we get the type $\beta$, as expected.

\end{example}



\subsection{Implementation}


We implemented the {\it{infer}} function in class.  Recall that we encoded
the contexts  by the following type.
\begin{program**}
\> type Context = [(String, Type)]\\
\end{program**}

The {\it{infer}} function essentially implements a derivation.  To be able to
choose {\em{fresh}} variables we need to keep track of the all the variable
names used in building the derivation.  To do this we will pass a list of
variables used so far (a list of strings) and return them together with the
list of constraints.  In this way, we thread a list of the variables used so
far through the computation.


The type of the {\tt{infer}} function is as follows:
\begin{program**}
\> infer :: Context $\rightarrow$ Term $\rightarrow$ Type $\rightarrow$ [String] $\rightarrow$ ([Constraint], [String]) \\ 
\end{program**}
This function takes a context (denoted $\Gamma$ in the rules above and
represented by the type {\it{Context}} here), a term to infer the type of, a
type (denoted $\tau$ in the rules above and initially a type variable not
occurring anywhere in the context), and a string list containing the names of
all the type variables used so far.

The following function is used to generate fresh variables.  It is a bit more
elegant than the quick and dirty version we sued in class.

\begin{program**}
\> fresh :: String $\rightarrow$ [String] $\rightarrow$ String\\
\> fresh x xs = if not(x `elem` xs) then x else fresh' x 0   \\
\>    where  fresh' x i  = if not(x' `elem` xs) then x' else fresh' x (i + 1)  \\
\>        where x' = x ++ (show i) \\
\end{program**}


Here is the infer function we implemented in class.

\begin{program**}
\> infer ctx  (Var x) ty vars =  \\
\>    case (lookup x ctx) of \\
\>       Just t $\rightarrow$ ([(ty,t)],vars) \\
\>       Nothing $\rightarrow$ error "infer: Var-case  failure" \\
\>   \\
\> infer ctx (Ap t1 t2) ty vars = (c1 ++ c2, vars2) \\
\>     where (c1,vars1) = infer ctx t1 (BinType Arrow (TVar a) ty) (a : vars) \\
\>           (c2,vars2) = infer ctx t2 (TVar a) vars1 \\
\>           a =  fresh "a" vars \\
\>   \\
\> infer ctx (Abs x t) ty vars = ((BinType Arrow (TVar a) (TVar b), ty) : c, vars') \\
\>    where (c,vars') = infer ((x, TVar a) : ctx) t (TVar b) (a : b : vars) \\
\>          a = fresh "a" vars \\
\>          b = fresh "b" vars \\
\end{program**}

The function is defined by recursion on the structure of the term whose type is
being inferred.  The case {\it{(Var x)}} implements the Var rule, the case
labeled {\it{(Ap m n)}} implements the (Ap) rule and the case labeled {\it{(Abs
x m)}} implements the (Abs) rule.


\subsection{Adding product types.}

We already included product types (pairs) in the type {\it{Type}} but we did
not include any mends for forming pairs or destructing pairs in the programming
language.  To do so we extend the language as follows:

\begin{program**}
\>  data Term = Var String  \\
\>            $\mid$ Ap Term Term \\
\>            $\mid$ Abs String Term \\
\>            $\mid$ Pair Term Term\\
\>            $\mid$ Fst Term \\
\>            $\mid$ Snd Term\\
\end{program**}

Here is a comparison of the mathematical notation and the corresponding Haskell notation.
\ \\
\begin{center}
\begin{tabular}{|c|c|}
\hline
Mathematical & Haskell \\
Notation & Notation \\ \hline
$x$  & {\it{(Var x)}}\\
(M\,N) & {\it{(Ap m n)}}\\
$\lambda{}x.M$ & {\it{(Abs x m)}}\\
$\langle{}M,N\rangle$  & {\it{(Pair m n)}}\\
{\it{fst M}} & {\it{(Fst m)}}\\
{\it{snd M}} & {\it{(Snd m)}}\\
\hline
\end{tabular}
\end{center}

\ \\
Recall the computation rules for {\it{fst}} and {\it{snd}}.
\[\begin{array}{ccc} {\mathit{fst}} \,\pair{m,n} = m  && {\mathit{snd}} \,\pair{m,n} = n \end{array}\]

Here are the proof rules keyed to the new kinds of terms.\ \\

\SequentRuleTwo{\Gamma,C_1\cup{}C_2\cup{}\{\tau=\alpha\times\beta\}}{\pair{M,N} : \tau}{\Gamma,C_1}{M:\alpha}{\Gamma,C_2}{N:\beta}{Pair}
{\hspace{.25in}} where $\alpha$ and $\beta$ are fresh.
\ \vspace{.125in}\\

\SequentRule{\Gamma,C}{fst\,{M}:\tau}{\Gamma,C}{M:\tau\times\alpha}{fst}{\hspace{.25in}}
where $\alpha$ is fresh. \vspace{.5em}

\SequentRule{\Gamma,C}{snd\,{M}:\tau}{\Gamma,C}{M:\alpha\times\tau}{snd}{\hspace{.25in}}
where $\alpha$ is fresh. \vspace{.5em}


\begin{exercise}
Using the base code provided extend the {\tt{infer}} function to implement
these additional type inference rules.
\end{exercise}




\end{document}
% Local Variables:
% mode:latex
% comment-column:0
% comment-start: "% "
% compile-command: "pdflatex hw12d"
% fill-column:79
% End:




