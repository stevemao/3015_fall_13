\documentclass[11pt]{article}

\usepackage{amssymb}
\usepackage{remark}
% \usepackage{html}


\usepackage{/home/faculty/jlc/papers/tex-stuff/bussproofs}



\setlength{\textwidth}{6.5in}
\setlength{\textheight}{8.8in}
\setlength{\oddsidemargin}{0.0in}
\setlength{\evensidemargin}{0.0in}


 \newremark{theorem}{Theorem}[section]
 \newremark{Rule}{Rule}[section]
 \newremark{conjecture}{Conjecture}[section]
 \newremark{corollary}{Corollary}[section]
 \newremark{example}{Example}[section]
 \newremark{fact}{Fact}[section]
 \newremark{lemma}{Lemma}[section]
 \newremark{definition}{Definition}[section]
 \newremark{claim}{Claim}[section]
 \newremark{remark}{Remark}[section]
 \newremark{slogan}{Slogan}[section]
 \newremark{axiom}{Axiom}[section]
 \newremark{note}{Note}[]
 \newremark{problem}{Problem}[section]
 \newremark{exercise}{Exercise}[section]

\newcommand{\Proof}{{\goodbreak\noindent\bf{Proof:\ }}}
\newcommand{\qed}{\goodbreak\noindent$\Box$}

\newcommand{\byDef}[1]{\stackrel{\langle\langle{{\rm{def.\ of\ }}{#1}}\rangle\rangle}{=}}
\newcommand{\byEq}[1]{\stackrel{\langle\langle{{#1}}\rangle\rangle}{=}}
\newcommand{\by}[1]{{\small{\langle\langle{\mathit{by.\;def.\;of\;}}{#1}\rangle\rangle}}}


\newcommand{\dotcup}{{\cup\hspace{-.5em}\cdot}}
\newcommand{\barcup}{{\cup\hspace{-.68em}-}}


\newcommand{\arrow}{$\rightarrow$}
\newcommand{\nat}{{\mathbb{N}}}
\newcommand{\bool}{{\mathbb{B}}}
\newcommand{\Int}{{\mathbb{Z}}}

\newcommand{\nand}{{\overline{\wedge}}}
\newcommand{\nor}{{\overline{\vee}}}


\newcommand{\Meaning}[1]{{[{\hspace{-.13em}}[{#1}]{\hspace{-.13em}}]}}
\newcommand{\Meaningof}[2]{{[{\hspace{-.13em}}[\,{#2:#1}\,]{\hspace{-.13em}}]}}
\newcommand{\Skip}{{\bf{skip}}}
\newcommand{\Let}[2]{{\bf{Let\ }} {#1}{\bf{\ in\ }}{#2}{\bf{\ end}}}
\newcommand{\A}{{\tt{a}}}
% \newcommand{\F}{{\tt{F}}}
\newcommand{\ua}[1]{{\tt{(#1 under a)}}}
\newcommand{\Fua}{\ua{F}}
\newcommand{\aF}{{\tt{(a F)}}}
\newcommand{\Btt}{{\tt{Btt}}}
\newcommand{\Bff}{{\tt{Bff}}}
%\newcommand{\T3}{{\tt{T3}}}
%\newcommand{\F3}{{\tt{F3}}}
%\newcommand{\?3}{{\tt{?3}}}
\newcommand{\V}[1]{{\mkleeneopen{}{\tt{#1}}\mkleeneclose{}}}
\newcommand{\Neg}[1]{{\tt{N(#1)}}}
\newcommand{\I}[2]{{\tt{(#1)I(#2)}}}
\newcommand{\ie}{{\em{i.e.}}}
\newcommand{\eg}{{\em{e.g.}}}
\newcommand{\sat}[2]{{\tt{#1 \mbar{}= #2}}}
\newcommand{\nsat}[2]{{\tt{#1 \mbar{}\mneq{} #2}}}
\newcommand{\Three}{{\mBbbN{}\msubthree{}}}
%\newcommand{\Three}{{\bf{3}}}
\newcommand{\Tzero}{${\tt{0}}_{\tt{3}}$}
\newcommand{\Tone}{${\tt{1}}_{\tt{3}}$}
\newcommand{\Ttwo}{${\tt{2}}_{\tt{3}}$}
\newcommand{\msubst}{{\tt{/}}}
\newcommand{\nsequent}{{\tt{>>}}}

\newcommand{\kleene}[1]{{\mkleeneopen{}#1\mkleeneclose{}}}
\newcommand{\mfvar}[1]{{\kleene{{\tt{#1}}}}}
\newcommand{\mfnot}{{\kleene{\msim}}}
\newcommand{\mfand}{{\kleene{\mwedge}}}
\newcommand{\mfor}{{\kleene{\mvee}}}
\newcommand{\mfimp}{{\kleene{\mRightarrow}}}
\newcommand{\Knot}{$\sim_K\,$}
\newcommand{\Kand}{$\wedge_K$}
\newcommand{\Kor}{$\vee_K$}
\newcommand{\Kimp}{$\Rightarrow_K$}
\newcommand{\miff}{$\Leftrightarrow{}$}
\newcommand{\mlangle}{{$\langle$}}
\newcommand{\mrangle}{{$\rangle$}}
\newcommand{\mldots}{{$\ldots$}}
\newcommand{\mcdots}{{$\cdots$}}
%\newcommand{\rhd}{\triangleright}
\newcommand{\mrhd}{{$\triangleright$}}
\newcommand{\mRHD}{{$\triangleright_{\!R}\,$}}
\newcommand{\Sequent}[2]{{${#1}\,\vdash{}\,{#2}$}}

\newcommand{\typingRuleThree}[4]{{\begin{tabular}{c}{${#1}$} \hspace{2em} {${#2}$} \hspace{2em} {${#3}$}\\\hline{${#4}$}\end{tabular}}}
\newcommand{\typingRuleTwo}[3]{{\begin{tabular}{c}{${#1}$} \hspace{2em} {${#2}$}\\\hline{${#3}$}\end{tabular}}}
\newcommand{\typingRule}[2]{{\begin{tabular}{c}{${#1}$}{}\\\hline{${#2}$}{}\end{tabular}}}


\newcommand{\newSequentRule}[2]{{\begin{tabular}{c}{${#1}$}{}\\\hline{${#2}$}{}\end{tabular}}}

\newcommand{\sequentRule}[4]{\mbox{\raisebox{-.4ex}[4.25ex]{\begin{tabular}{rcl} \rule{0mm}{2.85ex}{$#1$} & {$\vdash$} & ${#2}$ \\\hline {\rule{0mm}{2.65ex}$#3$} & {$\vdash$} & {$#4$}\end{tabular}\\}}}
\newcommand{\AxiomRule}[3]{{\begin{tabular}{rcl} $\,$ \\\hline {\rule{0mm}{2.65ex}$#1$} & {$\vdash$} & {$#2$}\end{tabular}}(#3)}

\newcommand{\startProof}[1]{{\begin{tabular}{c} $\,$ \\\hline {\rule{0mm}{2.65ex}{#1}} \end{tabular}}}

\newcommand{\SequentRule}[5]{{\begin{tabular}{rcl} \rule{0mm}{2.85ex}{$#3$} & {$\vdash$} & ${#4}$ \\\hline {\rule{0mm}{2.65ex}$#1$} & {$\vdash$} & {$#2$}\end{tabular}}(#5)}

\newcommand{\SequentRuleTwo}[7]{{\begin{tabular}{lr} \Sequent{#3}{#4}\ &\ \Sequent{#5}{#6} \\\hline%
\multicolumn{2}{c}{{\rule{0mm}{2.65ex}{\Sequent{#1}{#2}}}} \end{tabular}}({#7})}


\newcommand{\cSequentRuleTwo}[7]{{\begin{tabular}{lr} \Sequent{${#1}$}{${#2}$}\ &\ \Sequent{${#3}$}{${#4}$} \\\hline%
\multicolumn{2}{c}{{\rule{0mm}{2.65ex}{\Sequent{${#5}$}{${#6}$}}}} \end{tabular}}({#7})}

\newcommand{\false}{\bf{false}}
\newcommand{\definedAs}{\;{\stackrel{\rm def}{=}}\;}
\newcommand{\pair}[1]{{\langle{#1}\rangle}}

\newcommand{\ttrue}{{\bf{T}}}
\newcommand{\ffalse}{{\bf{F}}}


% \renewcommand{\inr}[1]{{\tt{inr}({#1})}}
% \renewcommand{\inl}[1]{{\tt{inl}({#1})}}
% \renewcommand{\spread}[1]{{\tt{spread}}{#1}}
% \renewcommand{\decide}[1]{{\tt{decide}}({#1})}
% \renewcommand{\void}{{\tt{void}}}
% \renewcommand{\any}[1]{{\tt{any}{#1}}}
\newcommand{\Zero}{{\mathit{Zero}}}
\newcommand{\Succ}{{\mathit{Succ\,}}}
% \renewcommand{\ind}[1]{{\tt{ind}}{#1}}
% \renewcommand{\N}{\bf{N}}
% \renewcommand{\xxx}{\mbox{\hspace{.125in}}}
% \renewcommand{\Tpp}{${\bf{T}}^{++}$}
% \renewcommand{\TPP}{{\bf{T}}^{++}}
% \renewcommand{\GT}{{\bf{T}}}
% \newcommand{\abit}{$\,$}

\newcommand{\mysubtitle}[1]{{\ \\*[-.75em]{\bf{{#1}}}}}

{\obeyspaces\global\let =\ }

\newenvironment{bogustabbing}{\begin{tabbing}\={\mbox{\hspace{10em}}}\=\=\=\kill}%
{\end{tabbing}}


\newenvironment{program}{\tt\obeyspaces\begin{bogustabbing}\+\kill}{\end{bogustabbing}}
\newenvironment{program*}{\tt\obeyspaces\begin{bogustabbing}}{\end{bogustabbing}}
\newenvironment{program**}{\it\obeyspaces\begin{bogustabbing}}{\end{bogustabbing}}
\newenvironment{smallprogram*}{\hspace{2.5em}\small \it\obeyspaces\begin{bogustabbing}}{\end{bogustabbing}\vspace{-.0625in}}

\newcommand{\CASE}{\noindent{\bf{Case}}}
\newcommand{\expr}{{{\cal{E}}}}
\newcommand{\comm}{{{\cal{C}}}}
\newcommand{\mylet}[3]{{\tt{let\ }}{#1}{\tt{\ =\ }}{#2}{\tt{\ in\ }}{#3}}
\newcommand{\myfun}[2]{{\tt{fun\ }}{#1}{\tt{\ =\ }}{#2}}
\newcommand{\semwhile}[2]{{\tt{while\ }}{#1}{\tt{\ do\ }}{#2}{\tt{\ od:\,}}comm}
\newcommand{\while}[2]{{\tt{while\ }}{#1}{\tt{\ do\ }}{#2}{\tt{\ od}}}
\newcommand{\Seq}[2]{{#1}{\bf{\,;\,}}{#2}{\tt{:\,}}comm}
\newcommand{\semif}[4]{{\tt{if(}}{\Meaning{#1}({#4})}{\tt{,\ }}{\Meaning{#2}({#4})}{\tt{,\ }}{#3}{\tt{)}}}
\newcommand{\wif}[3]{{\tt{if}}{{#1}}{\tt{\ then\ }}{{#2}}{\tt{\ else\ }}{#3}{\tt{\ fi}}:\,comm}
\newcommand{\wnot}[1]{\neg{#1}}
\newcommand{\wassign}[2]{{\tt{loc}}_{#1} := {#2}}
\newcommand{\wderef}[1]{{\tt{@loc}}_{#1} }
\newcommand{\loc}[1]{${\tt{loc}}_{#1}$}
% \newcommand{\semif1}[4]{{\tt{if(}}{\Meaning{#1}({#4})}{\tt{,\ }}{\Meaning{#2}({#4})}{\tt{,\ }}{\Meaning{#3}({#4})}{\tt{)}}}

\newcommand{\mkleeneopen}{{\boldmath{^{\lceil}}}}
\newcommand{\mkleeneclose}{{\boldmath{^{\rceil}}}}
\newcommand{\kquote}[1]{{\mkleeneopen{}{{#1}}\mkleeneclose}}

\newcommand{\homework}[2]{\ \\\vspace{-1.25in}\\%
{\bf{HW {#1}}} \hfill {\bf{Prof. Caldwell}} \\%
{\bf{Due:}} {#2} 2013 \hfill {\bf{COSC 3015}}\ \\}

\newcommand{\ite}[3]{{\bf{if\ }}{#1}{\bf{\ then\ }}{#2}{\bf{\ else\ }}{#3}{\bf{\ fi}}}
\newcommand{\fun}{{\bf{fun\, }}}
\newcommand{\prog}[3]{{#1}{\bf{\ in \ }}{#2}}



\newcommand{\store}[1]{\langle{}{#1}\rangle}

\newcommand{\alphaeq}{\,{=_\alpha}\,}
\newcommand{\xleaf}{{\rm{Leaf}}}
\newcommand{\xnode}{{\rm{Node}}}
\newcommand{\vbar}{{\;\;{\bf{|}}\;\;}}
\newcommand{\abit}{{\mbox{\hspace{1em}}}}


\newcommand{\assign}[2]{{{#1} {\bf{:=}} {#2}}}
\newcommand{\sequence}[2]{{#1}\,{\bf{;}}\,{#2}}

\newcommand{\F}[1]{{\cal{F}}_{#1}}
\newcommand{\harrow}{{\tt{-\!\!\!>}}}
\newcommand{\smallsection}{{\goodbreak\[ \star {\hspace{.35in}} \star {\hspace{.35in}} \star \] \ \\}}
\newcommand{\append}{\,{\texttt{++}}\,}
\newcommand{\nil}{{\texttt{[{\hspace{.125em}}]}}}

\newcommand{\defof}[1]{\stackrel{<\!\!<{\textrm{def. of }}{#1}>\!\!>}{=}}
\newcommand{\pnote}[1]{{$\langle\!\langle$ {#1} $\rangle\!\rangle$}}

\newcommand{\spread}[3]{{\mathit{spread}({#1};{#2}.{#3})}}


\newcommand{\sfa}{{\sf{a}}}
\newcommand{\sfb}{{\sf{b}}}
\newcommand{\sfc}{{\sf{c}}}
\newcommand{\sfd}{{\sf{d}}}
\newcommand{\sfe}{{\sf{e}}}
\newcommand{\sff}{{\sf{f}}}


\begin{document}
\homework{5}{18 September}


\section{Type Inference - A Table Based Method}


In class I informally presented a method to determine the polymorphic type of a
Haskell expression.  We make the method more precise here by presenting two
rules and a table based method for deriving types.

We will use lower case Latin letters in a {\sf{Serif}} font
$\{\sfa,\sfb,\sfc,\sfd,\cdots\}$ to denote polymorphic type variables.

Polymorphic type variables range over types,{\em{i.e.}} they stand for
any type.  In the same way that an variable declared to be of type
{\tt{int}} in a C++ program can take on the value of any {\tt{int}},
the polymorphic type variable $\sfa$ can take the value of {\it{any}}
type.  This means, if $f::\sfa\rightarrow\sfa$ then any type can be
substituted for $\sfa$ and the function $f$ has that type.  For
example, replacing the polymorphic type variable \sfa~by the type
{\sf{String}} we get $f::{\sf{String}}\rightarrow{\sf{String}}$.
Replacing \sfa~ by the type $\sfa\rightarrow\sfa$ we get
$f::(\sfa\rightarrow\sfa)\rightarrow(\sfa\rightarrow\sfa)$.

Recall, function application associates to the left and so the term $x\;z
(y\;z)$ is parenthesized as $((x\;z)(y\;z))$. Also, recall that the function
type constructor $\rightarrow$ associates to the right so the type
$a\rightarrow{}b\rightarrow{}c$ is parenthesized as
$(a\rightarrow{}(b\rightarrow{}c))$.


\subsection{The Method}

We start by constructing an initial table that has a columns on the left
labeled at the top by the names of the formal parameters and whose first row
entries are labeled by different polymorphic type variables. There is also a
column on the right where the entry in each row is a labeled Haskell
expression.  This expression is the body of the function definition with
sub-expressions tagged with types (known so far). It may be more readable to
fully parenthesize the expression.

For example, consider the function $s$ defined as follows.

\[  s\; x\;y \; z\;=\; x\;z\;(y\;z) \]

The initial table for this function appears as follows:

\begin{center}
\begin{tabular}{lll|l}
$x$ & $y$ & $z$ & expression \\\hline{}
$\sfa$ & $\sfb$ & $\sfc$ &$((\stackrel{\sfa}{x} \; \stackrel{\sfc}{z})\; (\stackrel{\sfb}{y} \; \stackrel{\sfc}{z}))$ \\
\end{tabular}
\end{center}

There are two rules for constructing the next row of the table.  In the table
above, the rule that refines a type to an arrow type ($\rightarrow$) can be
applied.

\begin{quotation}
\noindent{{\bf{[Arrow Introduction Rule]}}} If $\tau$ is a type expression and
$\alpha$ is a polymorphic type variable ($\alpha\in\{\sfa,\sfb,\sfc,\cdots\}$)
and there is an application of labeled expressions $e_1$ and $e_2$ in the right
column having the form $(\stackrel{\alpha}{e_1}\;\stackrel{\tau}{e_2})$, then
make a new row by copying the last row and replacing all occurrences of
$\alpha$ by the type $\tau\rightarrow{}\beta$ where $\beta$ is a new variable
name not appearing anywhere in the row being copied.
\end{quotation}

The justification for the arrow introduction rule goes like this: If
$e_2::\tau$ then the application $(e_1\;e_2)$ is well-typed if and only if
$e_1$ is a function whose domain is $\tau$.  We do not know the range (yet) so
we just choose a fresh polymorphic variable name and wait to figure it out
later.  So, we create a new row from the one above by copying it and changing
all occurrences of the type variable $\alpha$ to the type
$\tau\rightarrow{}\beta$ where $\beta$ is a completely new variable.

There are two places this rule can be applied in the last row of the
example. The pattern of the rule
$(\stackrel{\alpha}{e_1}\;\stackrel{\tau}{e_2})$ matches the expression
$(\stackrel{\sfa}{x}\;\stackrel{\sfc}{z})$ by the following mapping:
\[\{e_1\mapsto{}x, \alpha\mapsto\sfa, e_2\mapsto{}z,\tau\mapsto\sfc\}\]
Also, the pattern $(\stackrel{\alpha}{e_1}\;\stackrel{\tau}{e_2})$ matches the
expression $(\stackrel{\sfb}{y}\;\stackrel{\sfc}{z})$ by the mapping:
\[\{e_1\mapsto{}y,\alpha\mapsto\sfb,e_2\mapsto{}z,  \tau\mapsto\sfc\}\]
Either application of the rule may be chosen; for no particular reason, we
choose the second.

We apply the arrow introduction rule setting
$\sfb\;=\;\sfc\rightarrow\sfd$. The polymorphic type variable \sfd~ is new. We
create a new row in the table by copying the last row and replacing all
occurrences of \sfb~by the type $(\sfc\rightarrow\sfd)$.  This yields the
following table.

\begin{center}
\begin{tabular}{ccc|l}
$x$ & $y$ & $z$ & expression \\\hline{}
$\sfa$ & $\sfb$ & $\sfc$ & $((\stackrel{\sfa}{x} \; \stackrel{\sfc}{z})\; (\stackrel{\sfb}{y} \; \stackrel{\sfc}{z}))$ \\
$\sfa$ & $\sfc\rightarrow{}\sfd$ & $\sfc$ &$((\stackrel{\sfa}{x} \; \stackrel{\sfc}{z})\; (\stackrel{\sfc\rightarrow{}\sfd}{y} \; \stackrel{\sfc}{z}))$ \\
\end{tabular}
\end{center}
Note that {\em{all}} the occurrences of $\sfb$ have been changed.

Now the arrow introduction rule could be applied again to the application
$(x\;z)$.  Instead, we introduce the second rule which is a simplification rule
that eliminates arrow types from the right side.

\begin{quotation}
\noindent{{\bf{[Arrow Elimination Rule]}}} If $\tau$ and $\tau'$ are type
expressions and there is an labeled application of expression $e_1$ of type
$\tau\rightarrow\tau'$ to expression $e_2$ of type $\tau$, then create a new
row in the table by copying the last row and replacing the labeled application
$(\stackrel{\tau\rightarrow\tau'}{e_1}\;\stackrel{\tau}{e_2})$ by
$\stackrel{\tau'}{(e_1\;e_2)}$.
\end{quotation}

\noindent{}The justification for the rule is simply that $(e_1\;e_2)$ must have type
$\tau'$ if $e_1::\tau\rightarrow\tau'$ and $e_2::\tau$.\\

In the running example, the arrow elimination rule has one match in the last
row of the example table.  Here is the matching:
\[\{e_1\mapsto{}y,e_2\mapsto{}z,\tau\mapsto{}\sfc,\tau'\mapsto\sfd\}\]
Applying the arrow elimination rule to the last row in the table above yields
the following table.

\begin{center}
\begin{tabular}{ccc|l}
$x$ & $y$ & $z$ & expression \\\hline{}
$\sfa$ & $\sfb$ & $\sfc$ &$((\stackrel{\sfa}{x} \; \stackrel{\sfc}{z})\; (\stackrel{\sfb}{y} \; \stackrel{\sfc}{z}))$ \\
$\sfa$ & $\sfc\rightarrow{}\sfd$ & $\sfc$ &$((\stackrel{\sfa}{x} \; \stackrel{\sfc}{z})\; (\stackrel{\sfc\rightarrow{}\sfd}{y} \; \stackrel{\sfc}{z}))$ \\
$\sfa$ & $\sfc\rightarrow{}\sfd$ & $\sfc$ & $((\stackrel{\sfa}{x} \; \stackrel{\sfc}{z})\; \stackrel{\sfd}{(y\;z)})$ \\
\end{tabular}
\end{center}

As the next step, we apply the arrow introduction rule.Since $x$ is applied to
an argument of type $\sfc$ it must be a function of type
$\sfc\rightarrow{}\sfe$ where $\sfe$ is a fresh type variable. Thus we use the
arrow introduction rule setting
\[\sfa \;=\; (\sfc\rightarrow\sfe)\]
To create the next row of the table, copy the last row and replace all
occurrences of \sfa~by the type $(\sfc\rightarrow\sfe)$.
\begin{center}
\begin{tabular}{ccc|l}
$x$ & $y$ & $z$ & expression \\\hline{}
$\sfa$ & $\sfb$ & $\sfc$ &$((\stackrel{\sfa}{x} \; \stackrel{\sfc}{z})\; (\stackrel{\sfb}{y} \; \stackrel{\sfc}{z}))$ \\
$\sfa$ & $\sfc\rightarrow{}\sfd$ & $\sfc$ &$((\stackrel{\sfa}{x} \; \stackrel{\sfc}{z})\; (\stackrel{\sfc\rightarrow{}\sfd}{y} \; \stackrel{\sfc}{z}))$ \\
$\sfa$ & $\sfc\rightarrow{}\sfd$ & $\sfc$ &$((\stackrel{\sfa}{x} \; \stackrel{\sfc}{z})\; \stackrel{\sfd}{(y\;z)})$ \\
$\sfc\rightarrow\sfe$ & $\sfc\rightarrow{}\sfd$ & $\sfc$ &$((\stackrel{\sfc\rightarrow\sfe}{x} \; \stackrel{\sfc}{z})\; \stackrel{\sfd}{(y\;z)})$ \\
\end{tabular}
\end{center}

Now, because $x::\sfc\rightarrow{}\sfe$ and $z::\sfc$ we know that the
application $(x\;z)$ has type $\sfe$.  We simplify the table by applying the
arrow elimination rule as follows:

\begin{center}
\begin{tabular}{ccc|l}
$x$ & $y$ & $z$ & expression \\\hline{}
$\sfa$ & $\sfb$ & $\sfc$ &$((\stackrel{\sfa}{x} \; \stackrel{\sfc}{z})\; (\stackrel{\sfb}{y} \; \stackrel{\sfc}{z}))$ \\
$\sfa$ & $\sfc\rightarrow{}\sfd$ & $\sfc$ &$((\stackrel{\sfa}{x} \; \stackrel{\sfc}{z})\; (\stackrel{\sfc\rightarrow{}\sfd}{y} \; \stackrel{\sfc}{z}))$ \\
$\sfa$ & $\sfc\rightarrow{}\sfd$ & $\sfc$ &$((\stackrel{\sfa}{x} \; \stackrel{\sfc}{z})\; \stackrel{\sfd}{(y\;z)})$ \\
$\sfc\rightarrow\sfe$ & $\sfc\rightarrow{}\sfd$ & $\sfc$ &$((\stackrel{\sfc\rightarrow\sfe}{x} \; \stackrel{\sfc}{z})\; \stackrel{\sfd}{(y\;z)})$ \\
$\sfc\rightarrow\sfe$ & $\sfc\rightarrow{}\sfd$ & $\sfc$ &$(\stackrel{\sfe}{(x\;z)} \stackrel{\sfd}{(y\;z)})$ \\
\end{tabular}
\end{center}
But now, $(x\;z)::\sfe$ and $(x\;z)$ is applied to $(y\;z)::\sfd$ so
$\sfe=\sfd\rightarrow\sff$ where $\sff$ is a new type variable. To create the
next line of the table we apply the arrow introduction rule by copying the last
line of the table and replacing all occurrences of $\sfe$ by
$\sfd\rightarrow\sff$.
\begin{center}
\begin{tabular}{ccc|l}
$x$ & $y$ & $z$ & expression \\\hline{}
$\sfa$ & $\sfb$ & $\sfc$ &$((\stackrel{\sfa}{x} \; \stackrel{\sfc}{z})\; (\stackrel{\sfb}{y} \; \stackrel{\sfc}{z}))$ \\
$\sfa$ & $\sfc\rightarrow{}\sfd$ & $\sfc$ &$((\stackrel{\sfa}{x} \; \stackrel{\sfc}{z})\; (\stackrel{\sfc\rightarrow{}\sfd}{y} \; \stackrel{\sfc}{z}))$ \\
$\sfa$ & $\sfc\rightarrow{}\sfd$ & $\sfc$ &$((\stackrel{\sfa}{x} \; \stackrel{\sfc}{z})\; \stackrel{\sfd}{(y\;z)})$ \\
$\sfc\rightarrow\sfe$ & $\sfc\rightarrow{}\sfd$ & $\sfc$ &$((\stackrel{\sfc\rightarrow\sfe}{x} \; \stackrel{\sfc}{z})\; \stackrel{\sfd}{(y\;z)})$ \\
$\sfc\rightarrow\sfe$ & $\sfc\rightarrow{}\sfd$ & $\sfc$ &$(\stackrel{\sfe}{(x\;z)} \stackrel{\sfd}{(y\;z)})$ \\
$\sfc\rightarrow(\sfd\rightarrow{}\sff)$ & $\sfc\rightarrow{}\sfd$ & $\sfc$ &$(\stackrel{\sfd\rightarrow\sff}{(x\;z)} \stackrel{\sfd}{(y\;z)})$ \\
\end{tabular}
\end{center}
But now we see that $(x\;z)::\sfd\rightarrow\sff$ and it is applied to
$(y\;z)::\sfd$ so the term $((x\;z)(y\;z))::\sff$.  We use the arrow
elimination rule to create a new row in the table as follows:
\begin{center}
\begin{tabular}{ccc|l}
$x$ & $y$ & $z$ & expression \\\hline{}
$\sfa$ & $\sfb$ & $\sfc$ &$((\stackrel{\sfa}{x} \; \stackrel{\sfc}{z})\; (\stackrel{\sfb}{y} \; \stackrel{\sfc}{z}))$ \\
$\sfa$ & $\sfc\rightarrow{}\sfd$ & $\sfc$ &$((\stackrel{\sfa}{x} \; \stackrel{\sfc}{z})\; (\stackrel{\sfc\rightarrow{}\sfd}{y} \; \stackrel{\sfc}{z}))$ \\
$\sfa$ & $\sfc\rightarrow{}\sfd$ & $\sfc$ &$((\stackrel{\sfa}{x} \; \stackrel{\sfc}{z})\; \stackrel{\sfd}{(y\;z)})$ \\
$\sfc\rightarrow\sfe$ & $\sfc\rightarrow{}\sfd$ & $\sfc$ &$((\stackrel{\sfc\rightarrow\sfe}{x} \; \stackrel{\sfc}{z})\; \stackrel{\sfd}{(y\;z)})$ \\
$\sfc\rightarrow\sfe$ & $\sfc\rightarrow{}\sfd$ & $\sfc$ &$(\stackrel{\sfe}{(x\;z)} \stackrel{\sfd}{(y\;z)})$ \\
$\sfc\rightarrow(\sfd\rightarrow{}\sff)$ & $\sfc\rightarrow{}\sfd$ & $\sfc$ &$\stackrel{\sff}{((x\;z)(y\;z))}$ \\
\end{tabular}
\end{center}

From this table we know the following:
\[\begin{array}{rcl}
x & :: & \sfc\rightarrow\sfd\rightarrow\sff \\
y & :: & \sfc\rightarrow\sfd\\
z & :: & \sfc \\
((x\,z)(y\,z)) & :: & \sff
\end{array}\]
We can read off the type of $s$ as
\[s::(\sfc\rightarrow{}\sfd\rightarrow{}\sff)\rightarrow(\sfc\rightarrow{}\sfd)\rightarrow \sfc \rightarrow{}\sff\]

Since $\sfc$, $\sfd$ and $\sff$ are type polymorphic type variables, we can
uniformly rename them to make the type more readable (we use the mapping
$\{\sfc\mapsto\sfa,\sfd\mapsto\sfb,\sff\mapsto\sfc\}$.) This gives the following
type.

\[s::(\sfa\rightarrow{}\sfb\rightarrow{}\sfc)\rightarrow(\sfa\rightarrow{}\sfb)\rightarrow \sfa \rightarrow{}\sfc\]


\begin{problem}
Use this method to compute the types for the following Haskell functions.

\[\begin{array}{ll}
1. & k\;x\;y\;=\;x \\
2. & {\mathit{compose}} \;f\; g\; x \;= f\; (g \;x)\\
3. & {\mathit{flip}}\;f\;x\;y\;=\;f\;y\;x
\end{array}\]


\end{problem}

\end{document}
% Local Variables:
% mode:latex
% comment-column:0
% comment-start: "% "
% compile-command: "pdflatex hw5e"
% fill-column:79
% End:




