\documentclass[11pt]{article}

\usepackage{amssymb}
\usepackage{remark}
% \usepackage{html}


\usepackage{/home/faculty/jlc/papers/tex-stuff/bussproofs}



\setlength{\textwidth}{6.5in}
\setlength{\textheight}{8.8in}
\setlength{\oddsidemargin}{0.0in}
\setlength{\evensidemargin}{0.0in}


 \newremark{theorem}{Theorem}[section]
 \newremark{Rule}{Rule}[section]
 \newremark{conjecture}{Conjecture}[section]
 \newremark{corollary}{Corollary}[section]
 \newremark{example}{Example}[section]
 \newremark{fact}{Fact}[section]
 \newremark{lemma}{Lemma}[section]
 \newremark{definition}{Definition}[section]
 \newremark{claim}{Claim}[section]
 \newremark{remark}{Remark}[section]
 \newremark{slogan}{Slogan}[section]
 \newremark{axiom}{Axiom}[section]
 \newremark{note}{Note}[]
 \newremark{problem}{Problem}[section]
 \newremark{exercise}{Exercise}[section]

\newcommand{\Proof}{{\goodbreak\noindent\bf{Proof:\ }}}
\newcommand{\qed}{\goodbreak\noindent$\Box$}

\newcommand{\byDef}[1]{\stackrel{\langle\langle{{\rm{def.\ of\ }}{#1}}\rangle\rangle}{=}}
\newcommand{\byEq}[1]{\stackrel{\langle\langle{{#1}}\rangle\rangle}{=}}
\newcommand{\by}[1]{{\small{\langle\langle{\mathit{by.\;def.\;of\;}}{#1}\rangle\rangle}}}


\newcommand{\dotcup}{{\cup\hspace{-.5em}\cdot}}
\newcommand{\barcup}{{\cup\hspace{-.68em}-}}


\newcommand{\arrow}{$\rightarrow$}
\newcommand{\nat}{{\mathbb{N}}}
\newcommand{\bool}{{\mathbb{B}}}
\newcommand{\Int}{{\mathbb{Z}}}

\newcommand{\nand}{{\overline{\wedge}}}
\newcommand{\nor}{{\overline{\vee}}}


\newcommand{\Meaning}[1]{{[{\hspace{-.13em}}[{#1}]{\hspace{-.13em}}]}}
\newcommand{\Meaningof}[2]{{[{\hspace{-.13em}}[\,{#2:#1}\,]{\hspace{-.13em}}]}}
\newcommand{\Skip}{{\bf{skip}}}
\newcommand{\Let}[2]{{\bf{Let\ }} {#1}{\bf{\ in\ }}{#2}{\bf{\ end}}}
\newcommand{\A}{{\tt{a}}}
% \newcommand{\F}{{\tt{F}}}
\newcommand{\ua}[1]{{\tt{(#1 under a)}}}
\newcommand{\Fua}{\ua{F}}
\newcommand{\aF}{{\tt{(a F)}}}
\newcommand{\Btt}{{\tt{Btt}}}
\newcommand{\Bff}{{\tt{Bff}}}
%\newcommand{\T3}{{\tt{T3}}}
%\newcommand{\F3}{{\tt{F3}}}
%\newcommand{\?3}{{\tt{?3}}}
\newcommand{\V}[1]{{\mkleeneopen{}{\tt{#1}}\mkleeneclose{}}}
\newcommand{\Neg}[1]{{\tt{N(#1)}}}
\newcommand{\I}[2]{{\tt{(#1)I(#2)}}}
\newcommand{\ie}{{\em{i.e.}}}
\newcommand{\eg}{{\em{e.g.}}}
\newcommand{\sat}[2]{{\tt{#1 \mbar{}= #2}}}
\newcommand{\nsat}[2]{{\tt{#1 \mbar{}\mneq{} #2}}}
\newcommand{\Three}{{\mBbbN{}\msubthree{}}}
%\newcommand{\Three}{{\bf{3}}}
\newcommand{\Tzero}{${\tt{0}}_{\tt{3}}$}
\newcommand{\Tone}{${\tt{1}}_{\tt{3}}$}
\newcommand{\Ttwo}{${\tt{2}}_{\tt{3}}$}
\newcommand{\msubst}{{\tt{/}}}
\newcommand{\nsequent}{{\tt{>>}}}

\newcommand{\kleene}[1]{{\mkleeneopen{}#1\mkleeneclose{}}}
\newcommand{\mfvar}[1]{{\kleene{{\tt{#1}}}}}
\newcommand{\mfnot}{{\kleene{\msim}}}
\newcommand{\mfand}{{\kleene{\mwedge}}}
\newcommand{\mfor}{{\kleene{\mvee}}}
\newcommand{\mfimp}{{\kleene{\mRightarrow}}}
\newcommand{\Knot}{$\sim_K\,$}
\newcommand{\Kand}{$\wedge_K$}
\newcommand{\Kor}{$\vee_K$}
\newcommand{\Kimp}{$\Rightarrow_K$}
\newcommand{\miff}{$\Leftrightarrow{}$}
\newcommand{\mlangle}{{$\langle$}}
\newcommand{\mrangle}{{$\rangle$}}
\newcommand{\mldots}{{$\ldots$}}
\newcommand{\mcdots}{{$\cdots$}}
%\newcommand{\rhd}{\triangleright}
\newcommand{\mrhd}{{$\triangleright$}}
\newcommand{\mRHD}{{$\triangleright_{\!R}\,$}}
\newcommand{\Sequent}[2]{{${#1}\,\vdash{}\,{#2}$}}

\newcommand{\typingRuleThree}[4]{{\begin{tabular}{c}{${#1}$} \hspace{2em} {${#2}$} \hspace{2em} {${#3}$}\\\hline{${#4}$}\end{tabular}}}
\newcommand{\typingRuleTwo}[3]{{\begin{tabular}{c}{${#1}$} \hspace{2em} {${#2}$}\\\hline{${#3}$}\end{tabular}}}
\newcommand{\typingRule}[2]{{\begin{tabular}{c}{${#1}$}{}\\\hline{${#2}$}{}\end{tabular}}}


\newcommand{\newSequentRule}[2]{{\begin{tabular}{c}{${#1}$}{}\\\hline{${#2}$}{}\end{tabular}}}

\newcommand{\sequentRule}[4]{\mbox{\raisebox{-.4ex}[4.25ex]{\begin{tabular}{rcl} \rule{0mm}{2.85ex}{$#1$} & {$\vdash$} & ${#2}$ \\\hline {\rule{0mm}{2.65ex}$#3$} & {$\vdash$} & {$#4$}\end{tabular}\\}}}
\newcommand{\AxiomRule}[3]{{\begin{tabular}{rcl} $\,$ \\\hline {\rule{0mm}{2.65ex}$#1$} & {$\vdash$} & {$#2$}\end{tabular}}(#3)}

\newcommand{\startProof}[1]{{\begin{tabular}{c} $\,$ \\\hline {\rule{0mm}{2.65ex}{#1}} \end{tabular}}}

\newcommand{\SequentRule}[5]{{\begin{tabular}{rcl} \rule{0mm}{2.85ex}{$#3$} & {$\vdash$} & ${#4}$ \\\hline {\rule{0mm}{2.65ex}$#1$} & {$\vdash$} & {$#2$}\end{tabular}}(#5)}

\newcommand{\SequentRuleTwo}[7]{{\begin{tabular}{lr} \Sequent{#3}{#4}\ &\ \Sequent{#5}{#6} \\\hline%
\multicolumn{2}{c}{{\rule{0mm}{2.65ex}{\Sequent{#1}{#2}}}} \end{tabular}}({#7})}


\newcommand{\cSequentRuleTwo}[7]{{\begin{tabular}{lr} \Sequent{${#1}$}{${#2}$}\ &\ \Sequent{${#3}$}{${#4}$} \\\hline%
\multicolumn{2}{c}{{\rule{0mm}{2.65ex}{\Sequent{${#5}$}{${#6}$}}}} \end{tabular}}({#7})}

\newcommand{\false}{\bf{false}}
\newcommand{\definedAs}{\;{\stackrel{\rm def}{=}}\;}
\newcommand{\pair}[1]{{\langle{#1}\rangle}}

\newcommand{\ttrue}{{\bf{T}}}
\newcommand{\ffalse}{{\bf{F}}}


% \renewcommand{\inr}[1]{{\tt{inr}({#1})}}
% \renewcommand{\inl}[1]{{\tt{inl}({#1})}}
% \renewcommand{\spread}[1]{{\tt{spread}}{#1}}
% \renewcommand{\decide}[1]{{\tt{decide}}({#1})}
% \renewcommand{\void}{{\tt{void}}}
% \renewcommand{\any}[1]{{\tt{any}{#1}}}
\newcommand{\Zero}{{\mathit{Zero}}}
\newcommand{\Succ}{{\mathit{Succ\,}}}
% \renewcommand{\ind}[1]{{\tt{ind}}{#1}}
% \renewcommand{\N}{\bf{N}}
% \renewcommand{\xxx}{\mbox{\hspace{.125in}}}
% \renewcommand{\Tpp}{${\bf{T}}^{++}$}
% \renewcommand{\TPP}{{\bf{T}}^{++}}
% \renewcommand{\GT}{{\bf{T}}}
% \newcommand{\abit}{$\,$}

\newcommand{\mysubtitle}[1]{{\ \\*[-.75em]{\bf{{#1}}}}}

{\obeyspaces\global\let =\ }

\newenvironment{bogustabbing}{\begin{tabbing}\={\mbox{\hspace{10em}}}\=\=\=\kill}%
{\end{tabbing}}


\newenvironment{program}{\tt\obeyspaces\begin{bogustabbing}\+\kill}{\end{bogustabbing}}
\newenvironment{program*}{\tt\obeyspaces\begin{bogustabbing}}{\end{bogustabbing}}
\newenvironment{program**}{\it\obeyspaces\begin{bogustabbing}}{\end{bogustabbing}}
\newenvironment{smallprogram*}{\hspace{2.5em}\small \it\obeyspaces\begin{bogustabbing}}{\end{bogustabbing}\vspace{-.0625in}}

\newcommand{\CASE}{\noindent{\bf{Case}}}
\newcommand{\expr}{{{\cal{E}}}}
\newcommand{\comm}{{{\cal{C}}}}
\newcommand{\mylet}[3]{{\tt{let\ }}{#1}{\tt{\ =\ }}{#2}{\tt{\ in\ }}{#3}}
\newcommand{\myfun}[2]{{\tt{fun\ }}{#1}{\tt{\ =\ }}{#2}}
\newcommand{\semwhile}[2]{{\tt{while\ }}{#1}{\tt{\ do\ }}{#2}{\tt{\ od:\,}}comm}
\newcommand{\while}[2]{{\tt{while\ }}{#1}{\tt{\ do\ }}{#2}{\tt{\ od}}}
\newcommand{\Seq}[2]{{#1}{\bf{\,;\,}}{#2}{\tt{:\,}}comm}
\newcommand{\semif}[4]{{\tt{if(}}{\Meaning{#1}({#4})}{\tt{,\ }}{\Meaning{#2}({#4})}{\tt{,\ }}{#3}{\tt{)}}}
\newcommand{\wif}[3]{{\tt{if}}{{#1}}{\tt{\ then\ }}{{#2}}{\tt{\ else\ }}{#3}{\tt{\ fi}}:\,comm}
\newcommand{\wnot}[1]{\neg{#1}}
\newcommand{\wassign}[2]{{\tt{loc}}_{#1} := {#2}}
\newcommand{\wderef}[1]{{\tt{@loc}}_{#1} }
\newcommand{\loc}[1]{${\tt{loc}}_{#1}$}
% \newcommand{\semif1}[4]{{\tt{if(}}{\Meaning{#1}({#4})}{\tt{,\ }}{\Meaning{#2}({#4})}{\tt{,\ }}{\Meaning{#3}({#4})}{\tt{)}}}

\newcommand{\mkleeneopen}{{\boldmath{^{\lceil}}}}
\newcommand{\mkleeneclose}{{\boldmath{^{\rceil}}}}
\newcommand{\kquote}[1]{{\mkleeneopen{}{{#1}}\mkleeneclose}}

\newcommand{\homework}[2]{\ \\\vspace{-1.25in}\\%
{\bf{HW {#1}}} \hfill {\bf{Prof. Caldwell}} \\%
{\bf{Due:}} {#2} 2013 \hfill {\bf{COSC 3015}}\ \\}

\newcommand{\ite}[3]{{\bf{if\ }}{#1}{\bf{\ then\ }}{#2}{\bf{\ else\ }}{#3}{\bf{\ fi}}}
\newcommand{\fun}{{\bf{fun\, }}}
\newcommand{\prog}[3]{{#1}{\bf{\ in \ }}{#2}}



\newcommand{\store}[1]{\langle{}{#1}\rangle}

\newcommand{\alphaeq}{\,{=_\alpha}\,}
\newcommand{\xleaf}{{\rm{Leaf}}}
\newcommand{\xnode}{{\rm{Node}}}
\newcommand{\vbar}{{\;\;{\bf{|}}\;\;}}
\newcommand{\abit}{{\mbox{\hspace{1em}}}}


\newcommand{\assign}[2]{{{#1} {\bf{:=}} {#2}}}
\newcommand{\sequence}[2]{{#1}\,{\bf{;}}\,{#2}}

\newcommand{\F}[1]{{\cal{F}}_{#1}}
\newcommand{\harrow}{{\tt{-\!\!\!>}}}
\newcommand{\smallsection}{{\goodbreak\[ \star {\hspace{.35in}} \star {\hspace{.35in}} \star \] \ \\}}
\newcommand{\append}{\,{\texttt{++}}\,}
\newcommand{\nil}{{\texttt{[{\hspace{.125em}}]}}}

\newcommand{\defof}[1]{\stackrel{<\!\!<{\textrm{def. of }}{#1}>\!\!>}{=}}
\newcommand{\pnote}[1]{{$\langle\!\langle$ {#1} $\rangle\!\rangle$}}

\newcommand{\spread}[3]{{\mathit{spread}({#1};{#2}.{#3})}}


\newcommand{\sfa}{{\sf{a}}}
\newcommand{\sfb}{{\sf{b}}}
\newcommand{\sfc}{{\sf{c}}}
\newcommand{\sfd}{{\sf{d}}}
\newcommand{\sfe}{{\sf{e}}}
\newcommand{\sff}{{\sf{f}}}


\begin{document}
\homework{11}{4 October}

\section{Types and Terms}

Any discussion of type inference for a programming language involves
two languages: a language of type expressions and the programming
language itself.

We start with a simple programming language -- the $\lambda$-calculus with
pairs.  Let \[{\cal{V}}=\{x,y,z,w,x_1,y_1,z_1,w_1,\cdots\}\] be an unbounded
set of variables. The, the following grammar presents the language of
$\lambda$-terms.

\[\begin{array}{l}
\Lambda ::= x \mid (M\,N) \mid \lambda{}x.M \mid (M,N) \mid {\mathit{fst}}\;M \mid {\mathit{snd}} \, M \\
\;\;{\mathrm{where\ }} x\in{}{\cal{V}} {\mathrm{\ is\ a \ variable.}}\\
\;\;\;\;\;\;\;\;\;\;\;\;M,N\in\Lambda \; {\mathrm{\ are\  previously\  constructed\ \lambda-terms.}}\\
\;\;\;\;\;\;\;\;\;\;\;\;{\mathit{fst,snd}}\;{\mathrm{\ are\ constant\ symbols.}}
\end{array}
\]
The term $(M\,N)$ is an {\em{application}} (of $M$ to $N$.) The term
$\lambda{}x.M$ is an {\em{abstraction}} and is how functions are defined.  The
term $(M,\,N)$ is a pair and {\it{fst}} and {\it{snd}} are the projections
functions for pairs.  Eventually we will discuss computation in the language,
but for now, we are interested in determining if a term is well typed or not.

We represent the $\lambda$-terms in Haskell as follows:

\begin{smallprogram*}
\>data  Term =   Var String $\mid$ Ap Term Term $\mid$ Abs String Term $\mid$ Pair Term Term $\mid$ Fst Term $\mid$ Snd Term \\
\>        deriving (Eq,Show) \\
\end{smallprogram*}


Let ${\cal{VT}} = \{\alpha,\beta,\gamma,\alpha_1\cdots\}$ be an unbounded set
of type variables. The language of types is give by the following grammar.

\[\begin{array}{l}
T ::= \alpha \mid \tau_1 \rightarrow \tau_2 \mid \tau_1 \times \tau_2  \\
\;\;{\mathrm{where\ }} \alpha\in{}{\cal{VT}} {\mathrm{\ is\ a \ type\ variable.}}\\
\;\;\;\;\;\;\;\;\;\;\;\;\tau_1,\tau_2\in{}T \; {\mathrm{\ are\  previously\  constructed\ type\ expressions.}}\\
\end{array}\]

The type $\tau_1\rightarrow\tau_2$ denotes the type of functions from type
$\tau_1$ to $\tau_2$.  The type $\tau_1\times\tau_2$ denoted the type of pairs
where the first element is of type $\tau_1$ and the second is of type $\tau_2$.

The Haskell representation of type expressions can be given as follows:
\begin{smallprogram*}
\> data Op = Arrow $\mid$ Product     deriving (Eq,Show) \\
\> data  Type =   TVar  String $\mid$  BinType Op  Type Type     deriving Eq \\
\end{smallprogram*}

In this representation the type $\alpha\rightarrow(\beta\times\gamma)$ would
have the representation
\begin{smallprogram*}
\>  BinType Arrow (TVar "a") (BinType Product (TVar "b") (TVar "c"))
\end{smallprogram*}

Note that type variables occur as the leafs in the syntax trees of type
expressions.


We choose to instantiate the Haskell type {\it{Type}} as an instance of the
show type class as follows:

\begin{smallprogram*}
\> instance Show Type where \\
\>    show (TVar x) = x \\
\>    show (BinType Arrow t1 t2) = "(" ++ show t1 ++ " -> " ++ show t2 ++ ")" \\
\>    show (BinType Product t1 t2) = "(" ++ show t1 ++ " X " ++ show t2 ++ ")" \\
\end{smallprogram*}


The list of variables in a type expression can be computed by the following
Haskell function.

\begin{smallprogram*}
\> vars :: Type $\rightarrow$ [String] \\
\> vars ty = nub (v ty)  \\
\>     where v (TVar x) = [x] \\
\>           v (BinType op t1 t2) = v t1 ++ v t2 \\
\end{smallprogram*}

Note that {\it{nub}} is in the library {\em{Data.List}} and eliminates
duplicate entries in a list.


\section{Substitutions}

A {\em{type substitution }} is a function of type ${\cal{VT}}\rightarrow{}T$
mapping type variables to types.  Since the Haskell datatype {\it{Type}} uses
{\tt{String}} to represent variables substitutions can be defined in Haskell by
the following type:
\begin{smallprogram*}
\> Substitution :: String $\rightarrow$ Type \\
\end{smallprogram*}
We define the following function to recursively apply a substitution to a type.
\begin{smallprogram*}
\> subst :: Substitution $\rightarrow$ Type $\rightarrow$ Type\\
\> subst s (TVar x) = s x \\
\> subst s (BinType op t1 t2) = BinType op (subst s t1) (subst s t2) \\
\end{smallprogram*}
Note that applying a substitution to a type can only add structure at the
leaves {\em{i.e.}} applying a substitution can not change the top-level shape of
the syntax tree of the type, it can only change a variable which is a leaf.

The identity substitution is the one that maps strings {\it{x}} to types of the tome {\it{TVar x}}.
\begin{smallprogram*}
\> idSubst :: Substitution\\
\> idSubst x = (TVar x)\\
\end{smallprogram*}
We have the following theorem.
\begin{theorem}
\[\forall{}t:{\mathit{Type}}.\;\; {\mathit{subst}} \; {\mathit{idSubst}} \; t = t\]
\end{theorem}
You could prove this theorem by induction on the structure of the type $t$.

We can write down substitutions by enumerating the points where they differ
from {\it{idSubst}}.  For example the substitution that behaves like
{\it{idSubst}} except on variables $\alpha$ and $\beta$  could be written as
\[s = \{\alpha\mapsto{}\tau_1,\beta\mapsto\tau_2\}\]

We define the pointwise update of a function as follows:
\begin{smallprogram*}
\> update (x,v) f = ($\backslash\,$y $\rightarrow$  if y == x then v else f y)\\
\end{smallprogram*} 

Thus, the substitution $s$ enumerated above could be computed in Haskell by the following expression:
\begin{smallprogram*}
\> update ($\beta$, $\tau_2$) (update ($\alpha$,$\tau_2$) idSubst) \\
\end{smallprogram*} 



\section{Unification}

Given a pair of types $\tau_1$ and $\tau_2$ we say they are {\em{unifiable}} if
there is a substitution (call it $\sigma$) such that
\[{\mathit{subst\ }} \sigma \;\tau_1 = {\mathit{subst\ }} \sigma\; \tau_2\]

For example, the types $\alpha\rightarrow\alpha$ and
$\beta\times\gamma\rightarrow\delta$ is unifiable by a substitution of the
following form:
\[\begin{array}{l}
\{\alpha \mapsto  \beta\times\gamma,\; \delta \mapsto \alpha\}
\end{array}\]
Type expressions that do not share the same top level shape can not be unified.
For example, there is no substitution that unifies $\alpha\rightarrow\beta$
with $\alpha\times\beta$ because $\rightarrow$ and $\times$ do not match.
There is also a problem with a case like $\alpha$ and $\alpha\rightarrow\beta$.
If you try to figure out a way to do this, you see that $\alpha$ must become
something like $\alpha\rightarrow\beta$.  Consider what happens if we apply the
substitution $s = \{\alpha\mapsto(\alpha\rightarrow\beta)\}$.
\[\begin{array}{l}
\> {\mathit{subst}}\;s\;\alpha = (\alpha\rightarrow\beta)\\
\> {\mathit{subst}}\;s\; (\alpha\rightarrow\beta) = (\alpha\rightarrow\beta)\rightarrow\beta\\
\end{array}\]
So this leads to a kind of loop.  We keep expanding $\alpha$ to a term that has
$\alpha$ in it so we'll never get a match on both sides. This is called an
{\em{occurs check failure}}.


The unification algorithm takes two terms and, if they are unifiable, returns a
substitution that will make them identical.  Here is the unification algorithm
described mathematically where $\otimes$ is one of the type constructors
$\{\rightarrow,\times\}$.

\begin{smallprogram*}
\> unify $\alpha$ $\alpha$ = idSubst \\
\> unify $\alpha$ $\beta$ = update ($\alpha, \beta$) idSubst \\
\> unify $\alpha$ $(\tau_1\otimes\tau_2)$ = \\
\>             if $\alpha\in$ vars$(\tau_1\otimes\tau_2)$ then\\
\>                error "Occurs check failure"\\
\>             else\\
\>                update ($\alpha, \tau_1\otimes\tau_2$) idSubst \\
\> unify $(\tau_1\otimes\tau_2)$ $\alpha$ = unify $\alpha$ $(\tau_1\otimes\tau_2)$ \\
\> unify $(\tau_1\otimes_1\tau_2)$ $(\tau_3\otimes_2\tau_4)$ = if $\otimes_1$ == $\otimes_2$ then\\
\>                                       (subst s2) . s1 \\
\>                                     else \\
\>                                       error "not unifiable." \\
\>               where s1 = unify $\tau_1$ $\tau_3$  \\
\>                       s2 = unify (subst s1 $\tau_2$) (subst s1 $\tau_4$) \\
\end{smallprogram*}

\exercise{Using the base code provided on the web-page implement the
{\it{unify}} function in Haskell.}


\end{document}
% Local Variables:
% mode:latex
% comment-column:0
% comment-start: "% "
% compile-command: "pdflatex hw11c"
% fill-column:79
% End:




