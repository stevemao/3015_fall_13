\documentclass[11pt]{article}

\usepackage{amssymb}
\usepackage{remark}
% \usepackage{html}


\usepackage{/home/faculty/jlc/papers/tex-stuff/bussproofs}



\setlength{\textwidth}{6.5in}
\setlength{\textheight}{8.8in}
\setlength{\oddsidemargin}{0.0in}
\setlength{\evensidemargin}{0.0in}


 \newremark{theorem}{Theorem}[section]
 \newremark{Rule}{Rule}[section]
 \newremark{conjecture}{Conjecture}[section]
 \newremark{corollary}{Corollary}[section]
 \newremark{example}{Example}[section]
 \newremark{fact}{Fact}[section]
 \newremark{lemma}{Lemma}[section]
 \newremark{definition}{Definition}[section]
 \newremark{claim}{Claim}[section]
 \newremark{remark}{Remark}[section]
 \newremark{slogan}{Slogan}[section]
 \newremark{axiom}{Axiom}[section]
 \newremark{note}{Note}[]
 \newremark{problem}{Problem}[section]
 \newremark{exercise}{Exercise}[section]

\newcommand{\Proof}{{\goodbreak\noindent\bf{Proof:\ }}}
\newcommand{\qed}{\goodbreak\noindent$\Box$}

\newcommand{\byDef}[1]{\stackrel{\langle\langle{{\rm{def.\ of\ }}{#1}}\rangle\rangle}{=}}
\newcommand{\byEq}[1]{\stackrel{\langle\langle{{#1}}\rangle\rangle}{=}}
\newcommand{\by}[1]{{\small{\langle\langle{\mathit{by.\;def.\;of\;}}{#1}\rangle\rangle}}}


\newcommand{\dotcup}{{\cup\hspace{-.5em}\cdot}}
\newcommand{\barcup}{{\cup\hspace{-.68em}-}}


\newcommand{\arrow}{$\rightarrow$}
\newcommand{\nat}{{\mathbb{N}}}
\newcommand{\bool}{{\mathbb{B}}}
\newcommand{\Int}{{\mathbb{Z}}}

\newcommand{\nand}{{\overline{\wedge}}}
\newcommand{\nor}{{\overline{\vee}}}


\newcommand{\Meaning}[1]{{[{\hspace{-.13em}}[{#1}]{\hspace{-.13em}}]}}
\newcommand{\Meaningof}[2]{{[{\hspace{-.13em}}[\,{#2:#1}\,]{\hspace{-.13em}}]}}
\newcommand{\Skip}{{\bf{skip}}}
\newcommand{\Let}[2]{{\bf{Let\ }} {#1}{\bf{\ in\ }}{#2}{\bf{\ end}}}
\newcommand{\A}{{\tt{a}}}
% \newcommand{\F}{{\tt{F}}}
\newcommand{\ua}[1]{{\tt{(#1 under a)}}}
\newcommand{\Fua}{\ua{F}}
\newcommand{\aF}{{\tt{(a F)}}}
\newcommand{\Btt}{{\tt{Btt}}}
\newcommand{\Bff}{{\tt{Bff}}}
%\newcommand{\T3}{{\tt{T3}}}
%\newcommand{\F3}{{\tt{F3}}}
%\newcommand{\?3}{{\tt{?3}}}
\newcommand{\V}[1]{{\mkleeneopen{}{\tt{#1}}\mkleeneclose{}}}
\newcommand{\Neg}[1]{{\tt{N(#1)}}}
\newcommand{\I}[2]{{\tt{(#1)I(#2)}}}
\newcommand{\ie}{{\em{i.e.}}}
\newcommand{\eg}{{\em{e.g.}}}
\newcommand{\sat}[2]{{\tt{#1 \mbar{}= #2}}}
\newcommand{\nsat}[2]{{\tt{#1 \mbar{}\mneq{} #2}}}
\newcommand{\Three}{{\mBbbN{}\msubthree{}}}
%\newcommand{\Three}{{\bf{3}}}
\newcommand{\Tzero}{${\tt{0}}_{\tt{3}}$}
\newcommand{\Tone}{${\tt{1}}_{\tt{3}}$}
\newcommand{\Ttwo}{${\tt{2}}_{\tt{3}}$}
\newcommand{\msubst}{{\tt{/}}}
\newcommand{\nsequent}{{\tt{>>}}}

\newcommand{\kleene}[1]{{\mkleeneopen{}#1\mkleeneclose{}}}
\newcommand{\mfvar}[1]{{\kleene{{\tt{#1}}}}}
\newcommand{\mfnot}{{\kleene{\msim}}}
\newcommand{\mfand}{{\kleene{\mwedge}}}
\newcommand{\mfor}{{\kleene{\mvee}}}
\newcommand{\mfimp}{{\kleene{\mRightarrow}}}
\newcommand{\Knot}{$\sim_K\,$}
\newcommand{\Kand}{$\wedge_K$}
\newcommand{\Kor}{$\vee_K$}
\newcommand{\Kimp}{$\Rightarrow_K$}
\newcommand{\miff}{$\Leftrightarrow{}$}
\newcommand{\mlangle}{{$\langle$}}
\newcommand{\mrangle}{{$\rangle$}}
\newcommand{\mldots}{{$\ldots$}}
\newcommand{\mcdots}{{$\cdots$}}
%\newcommand{\rhd}{\triangleright}
\newcommand{\mrhd}{{$\triangleright$}}
\newcommand{\mRHD}{{$\triangleright_{\!R}\,$}}
\newcommand{\Sequent}[2]{{${#1}\,\vdash{}\,{#2}$}}

\newcommand{\typingRuleThree}[4]{{\begin{tabular}{c}{${#1}$} \hspace{2em} {${#2}$} \hspace{2em} {${#3}$}\\\hline{${#4}$}\end{tabular}}}
\newcommand{\typingRuleTwo}[3]{{\begin{tabular}{c}{${#1}$} \hspace{2em} {${#2}$}\\\hline{${#3}$}\end{tabular}}}
\newcommand{\typingRule}[2]{{\begin{tabular}{c}{${#1}$}{}\\\hline{${#2}$}{}\end{tabular}}}


\newcommand{\newSequentRule}[2]{{\begin{tabular}{c}{${#1}$}{}\\\hline{${#2}$}{}\end{tabular}}}

\newcommand{\sequentRule}[4]{\mbox{\raisebox{-.4ex}[4.25ex]{\begin{tabular}{rcl} \rule{0mm}{2.85ex}{$#1$} & {$\vdash$} & ${#2}$ \\\hline {\rule{0mm}{2.65ex}$#3$} & {$\vdash$} & {$#4$}\end{tabular}\\}}}
\newcommand{\AxiomRule}[3]{{\begin{tabular}{rcl} $\,$ \\\hline {\rule{0mm}{2.65ex}$#1$} & {$\vdash$} & {$#2$}\end{tabular}}(#3)}

\newcommand{\startProof}[1]{{\begin{tabular}{c} $\,$ \\\hline {\rule{0mm}{2.65ex}{#1}} \end{tabular}}}

\newcommand{\SequentRule}[5]{{\begin{tabular}{rcl} \rule{0mm}{2.85ex}{$#3$} & {$\vdash$} & ${#4}$ \\\hline {\rule{0mm}{2.65ex}$#1$} & {$\vdash$} & {$#2$}\end{tabular}}(#5)}

\newcommand{\SequentRuleTwo}[7]{{\begin{tabular}{lr} \Sequent{#3}{#4}\ &\ \Sequent{#5}{#6} \\\hline%
\multicolumn{2}{c}{{\rule{0mm}{2.65ex}{\Sequent{#1}{#2}}}} \end{tabular}}({#7})}


\newcommand{\cSequentRuleTwo}[7]{{\begin{tabular}{lr} \Sequent{${#1}$}{${#2}$}\ &\ \Sequent{${#3}$}{${#4}$} \\\hline%
\multicolumn{2}{c}{{\rule{0mm}{2.65ex}{\Sequent{${#5}$}{${#6}$}}}} \end{tabular}}({#7})}

\newcommand{\false}{\bf{false}}
\newcommand{\definedAs}{\;{\stackrel{\rm def}{=}}\;}
\newcommand{\pair}[1]{{\langle{#1}\rangle}}

\newcommand{\ttrue}{{\bf{T}}}
\newcommand{\ffalse}{{\bf{F}}}


% \renewcommand{\inr}[1]{{\tt{inr}({#1})}}
% \renewcommand{\inl}[1]{{\tt{inl}({#1})}}
% \renewcommand{\spread}[1]{{\tt{spread}}{#1}}
% \renewcommand{\decide}[1]{{\tt{decide}}({#1})}
% \renewcommand{\void}{{\tt{void}}}
% \renewcommand{\any}[1]{{\tt{any}{#1}}}
\newcommand{\Zero}{{\mathit{Zero}}}
\newcommand{\Succ}{{\mathit{Succ\,}}}
% \renewcommand{\ind}[1]{{\tt{ind}}{#1}}
% \renewcommand{\N}{\bf{N}}
% \renewcommand{\xxx}{\mbox{\hspace{.125in}}}
% \renewcommand{\Tpp}{${\bf{T}}^{++}$}
% \renewcommand{\TPP}{{\bf{T}}^{++}}
% \renewcommand{\GT}{{\bf{T}}}
% \newcommand{\abit}{$\,$}

\newcommand{\mysubtitle}[1]{{\ \\*[-.75em]{\bf{{#1}}}}}

{\obeyspaces\global\let =\ }

\newenvironment{bogustabbing}{\begin{tabbing}\={\mbox{\hspace{10em}}}\=\=\=\kill}%
{\end{tabbing}}


\newenvironment{program}{\tt\obeyspaces\begin{bogustabbing}\+\kill}{\end{bogustabbing}}
\newenvironment{program*}{\tt\obeyspaces\begin{bogustabbing}}{\end{bogustabbing}}
\newenvironment{program**}{\it\obeyspaces\begin{bogustabbing}}{\end{bogustabbing}}
\newenvironment{smallprogram*}{\hspace{2.5em}\small \it\obeyspaces\begin{bogustabbing}}{\end{bogustabbing}\vspace{-.0625in}}

\newcommand{\CASE}{\noindent{\bf{Case}}}
\newcommand{\expr}{{{\cal{E}}}}
\newcommand{\comm}{{{\cal{C}}}}
\newcommand{\mylet}[3]{{\tt{let\ }}{#1}{\tt{\ =\ }}{#2}{\tt{\ in\ }}{#3}}
\newcommand{\myfun}[2]{{\tt{fun\ }}{#1}{\tt{\ =\ }}{#2}}
\newcommand{\semwhile}[2]{{\tt{while\ }}{#1}{\tt{\ do\ }}{#2}{\tt{\ od:\,}}comm}
\newcommand{\while}[2]{{\tt{while\ }}{#1}{\tt{\ do\ }}{#2}{\tt{\ od}}}
\newcommand{\Seq}[2]{{#1}{\bf{\,;\,}}{#2}{\tt{:\,}}comm}
\newcommand{\semif}[4]{{\tt{if(}}{\Meaning{#1}({#4})}{\tt{,\ }}{\Meaning{#2}({#4})}{\tt{,\ }}{#3}{\tt{)}}}
\newcommand{\wif}[3]{{\tt{if}}{{#1}}{\tt{\ then\ }}{{#2}}{\tt{\ else\ }}{#3}{\tt{\ fi}}:\,comm}
\newcommand{\wnot}[1]{\neg{#1}}
\newcommand{\wassign}[2]{{\tt{loc}}_{#1} := {#2}}
\newcommand{\wderef}[1]{{\tt{@loc}}_{#1} }
\newcommand{\loc}[1]{${\tt{loc}}_{#1}$}
% \newcommand{\semif1}[4]{{\tt{if(}}{\Meaning{#1}({#4})}{\tt{,\ }}{\Meaning{#2}({#4})}{\tt{,\ }}{\Meaning{#3}({#4})}{\tt{)}}}

\newcommand{\mkleeneopen}{{\boldmath{^{\lceil}}}}
\newcommand{\mkleeneclose}{{\boldmath{^{\rceil}}}}
\newcommand{\kquote}[1]{{\mkleeneopen{}{{#1}}\mkleeneclose}}

\newcommand{\homework}[2]{\ \\\vspace{-1.25in}\\%
{\bf{HW {#1}}} \hfill {\bf{Prof. Caldwell}} \\%
{\bf{Due:}} {#2} 2013 \hfill {\bf{COSC 3015}}\ \\}

\newcommand{\ite}[3]{{\bf{if\ }}{#1}{\bf{\ then\ }}{#2}{\bf{\ else\ }}{#3}{\bf{\ fi}}}
\newcommand{\fun}{{\bf{fun\, }}}
\newcommand{\prog}[3]{{#1}{\bf{\ in \ }}{#2}}



\newcommand{\store}[1]{\langle{}{#1}\rangle}

\newcommand{\alphaeq}{\,{=_\alpha}\,}
\newcommand{\xleaf}{{\rm{Leaf}}}
\newcommand{\xnode}{{\rm{Node}}}
\newcommand{\vbar}{{\;\;{\bf{|}}\;\;}}
\newcommand{\abit}{{\mbox{\hspace{1em}}}}


\newcommand{\assign}[2]{{{#1} {\bf{:=}} {#2}}}
\newcommand{\sequence}[2]{{#1}\,{\bf{;}}\,{#2}}

\newcommand{\F}[1]{{\cal{F}}_{#1}}
\newcommand{\harrow}{{\tt{-\!\!\!>}}}
\newcommand{\smallsection}{{\goodbreak\[ \star {\hspace{.35in}} \star {\hspace{.35in}} \star \] \ \\}}
\newcommand{\append}{\,{\texttt{++}}\,}
\newcommand{\nil}{{\texttt{[{\hspace{.125em}}]}}}

\newcommand{\defof}[1]{\stackrel{<\!\!<{\textrm{def. of }}{#1}>\!\!>}{=}}
\newcommand{\pnote}[1]{{$\langle\!\langle$ {#1} $\rangle\!\rangle$}}

\newcommand{\spread}[3]{{\mathit{spread}({#1};{#2}.{#3})}}


\newcommand{\sfa}{{\sf{a}}}
\newcommand{\sfb}{{\sf{b}}}
\newcommand{\sfc}{{\sf{c}}}
\newcommand{\sfd}{{\sf{d}}}
\newcommand{\sfe}{{\sf{e}}}
\newcommand{\sff}{{\sf{f}}}


\begin{document}
\homework{21}{22 November}


%% --------------------------------------------------------------------------------
\section{The Monad Type Class}
Here's the definitions of the type classes {\tt{Monad}}.

\begin{verbatim}
    class Monad m where
      return :: a -> ma
      (>>=) :: m a -> (a -> m b) -> mb
      (>>) :: m a -> m b -> mb
      m >> n  =  m >>= \_ -> n
\end{verbatim}

\noindent{}The monad laws are given as follows:\\

\begin{tabular}{ll}
Left Identity: & {\tt{(return x >>= f) = f x}} \\
Right Identity: & {\tt{(m >>= return)  = m}} \\
Associativity: & {\tt{((m >>= f) >>= g) = (m >>= ($\backslash$x -> f x >>= g))}} \\
\end{tabular}

\subsection{Maybe and List as instances of the Monad Type Class}

\begin{verbatim}
    instance Monad Maybe where
      return  = Just
      Nothing >>= f   = Nothing
      (Just x) >>= f  = f x
\end{verbatim}

\begin{verbatim}
    instance Monad [] where
      return x = [x]
      xs >>= f   = concat (map f xs)
\end{verbatim}


\subsection{Relating the instances to the Laws}

Consider the list monad. 

\begin{theorem}[Left Identity for Lists]
{\tt{(return x >>= f) = f x}}
\end{theorem}
\Proof
\[\begin{array}{l}
{\mathtt{return \; x >>= f}}\\
  \stackrel{\pair{\pair{{\mathrm{def}}.(>>=)}}}{=} {\mathtt{concat (map \; f (return \; x))}}\\
  \stackrel{\pair{\pair{{\mathrm{def}}.{\mathtt{return}}}}}{=} {\mathtt{concat (map \; f\; [x])}} \\
  \stackrel{\pair{\pair{{\mathrm{def}}.{\mathtt{map}}}}}{=} {\mathtt{concat [f \; x]}}
\end{array}\]
Now, by the type of {\tt{(>>=)}} we know {\tt{f :: a -> [b]}}.  This means {\tt{f x = [y]}} for some {\tt{y}}. This gives the following.
\[
  {\mathtt{concat [f\; x]}} = {\mathtt{concat \; [[y]]}} = [{\mathtt{y}}] = {\mathtt{f \; x}}
\]
\qed


\begin{exercise}
Use the definition of {\tt{(>>=)}} to show that {\tt{([] >>= f) = []}}.
Recall that concat can be defined primitively as follows:
\begin{verbatim}
    concat [] = []
    concat (xs:xss) = xs ++ (concat xss)
\end{verbatim}
\end{exercise}

%% --------------------------------------------------------------------------------
\section{The Monoid Type Class}

Monoids have an associative operator with a left and right identity. 
The type class {\tt{Monoid}} is give as follows.

\begin{verbatim}
    class Monoid m where 
      mempty  :: m
      mappend :: m -> m -> m
\end{verbatim}


\noindent{}The Monoid laws are as follows:\\


\begin{tabular}{ll}
Left Identity: & {\tt{ mempty `mappend` x =  x}} \\
Right Identity: & {\tt{x `mappend` mempty = x}} \\
Associativity: & {\tt{((x `mappend` y ) `mappend` z) = (x `mappend` (y `mappend` z))}} \\
\end{tabular}


\subsection{Maybe and List as instances of the Monoid Type Class}

\begin{verbatim}
    instance Monoid a => Monoid (Maybe a) where
      mempty                    = Nothing
      Nothing  `mappend` m      = m
      m `mappend` Nothing       = m  
      Just m1 `mappend` Just m2 = Just (m1 `mappend` m2)  
\end{verbatim}

\begin{verbatim}
    instance Monad [a] where
      mempty   = []
      mappend  = (++)
\end{verbatim}

%% --------------------------------------------------------------------------------
\section{The MonadPlus Type Class}

The type class MonadPlus essentially extends monads that have some monadic structure. 
is defined as follows:


\begin{verbatim}
    class (Monad m) => MonadPlus m where
       mzero :: m a
       mplus :: m a -> m a -> m a
\end{verbatim}

\noindent{}The laws relating {\tt{mzero}} and {\tt{mplus}} are the monoid laws.\\

\begin{tabular}{ll}
Left Identity: & {\tt{ mzero `mplus` x =  x}} \\
Right Identity: & {\tt{x `mplus` mzero = x}} \\
Associativity: & {\tt{((x `mplus` y ) `mplus` z) = (x `mplus` (y `mplus` z))}} \\
\end{tabular} \vspace{.125in}\\

But also (and this is not discussed in LYAHFGG) there must be a relationship
between the MonadPlus operators and the operators in the underlying monad.  It
turns out that there is some disagreement in the Haskell community about what
the correct laws relating the two should be.

\noindent{}The following laws relating the {\tt{mzero}} element of an instance
of MonadPlus with the bind operator are universally accepted: \\

\begin{tabular}{ll}
Left Zero : & {\tt{(mzero >>= m)  =  mzero}} \\
Right Zero : & {\tt{(m >> mzero)  =  mzero}} \\
\end{tabular}
\vspace{.125in} \\

\noindent{}The following laws are sometimes accepted:\\

\begin{tabular}{ll}
Left Distribution: & {\tt{(m `mplus` n) >>= k  =  (m >>= k) `mplus` (n >>= k)}}\\
Left Catch: & {\tt{((return a) `mplus` b) = return a}}\\
\end{tabular}
\vspace{.125in} \\
The instantiation of lists as an instance of the MonadPlus type class satisfy
the core laws, Left Zero, Right Zero, and Left Distribution.  Maybe, IO and the
state Monoid satisfy Left and Right Zero, and Left Catch.


\subsection{Maybe and List as instances of the MonadPlus Type Class}

\begin{verbatim}
    instance MonadPlus Maybe where
      mzero                   = Nothing
      Nothing  `mplus` m      = m
      m `mplus` Nothing       = m  
      Just m `mplus` Just n   = Just m
\end{verbatim}

\begin{verbatim}
    instance MonadPlus [] where
      mzero   = []
      mplus  = (++)
\end{verbatim}


\subsection{Something about the laws}

To see that the left distributive law does not hold for Maybe consider the
following Haskell interaction.

\begin{verbatim}
    Prelude> :m + Control.Monad
    Prelude Control.Monad> 
    Prelude Control.Monad> let k b =  if b then Nothing else Just True
    Prelude Control.Monad> (Just True `mplus` Just False) >>= k 
    Nothing
    Prelude Control.Monad> (Just True >>= k ) `mplus` (Just False >>=  k)
    Just True
\end{verbatim}

\begin{exercise}
Find an example that shows that lists do not satisfy the Left Catch rule.
\end{exercise}

%% --------------------------------------------------------------------------------
\section{List Comprehensions, do notation and guards}

\subsection{do notation}

Recall that a sequence of bind operations can be rewritten using the do notion as follows:
\begin{verbatim}
   m1 >>= \x1 ->                    do x1 <- m1		     
   m2 >>= \x2 ->                       x2 <- m2		     
   ...                                  ...		     
   mk >>= \xk ->                       xk <- mk		     
   return (f x1 x2 ... xk)             return (f x1 x2 ... xk)
\end{verbatim}
Code written using bind


\subsection{List comprehensions}
Consider the evaluation of the following list comprehension.
\begin{verbatim}
    Prelude> [(x,y) | x <- [1..3], y <- "ab"]
    [(1,'a'),(1,'b'),(2,'a'),(2,'b'),(3,'a'),(3,'b')]

\end{verbatim}
This code can be rewritten using do notion.

\begin{verbatim}
    Prelude> do {x <- [1..3]; y <- "ab"; return (x,y)}
    [(1,'a'),(1,'b'),(2,'a'),(2,'b'),(3,'a'),(3,'b')]
 
\end{verbatim}

Now we can eliminate the do notation.
\begin{verbatim}
    Prelude> [1..3] >>= \x -> "ab" >>= \y -> return (x,y)
    [(1,'a'),(1,'b'),(2,'a'),(2,'b'),(3,'a'),(3,'b')]
\end{verbatim}

Now, using the definitions of bind {\tt{(>>=)}} and {\tt{return}} for the list
monad we get the following.

\begin{verbatim}
    Prelude> concat (map (\x -> (concat (map  (\y -> [(x,y)]) "ab")))  [1..3])
    [(1,'a'),(1,'b'),(2,'a'),(2,'b'),(3,'a'),(3,'b')]
\end{verbatim}
But what about a list comprehension that contains a guard like the following:\\
\begin{verbatim}
    Prelude> [x | x <- [1..2], even x]
    [2]
\end{verbatim}
Guards can be implemented as follows:
\begin{verbatim}
    guard :: MonadPlus m => Bool -> m ()
    guard True  = return ()
    guard False = mzero
\end{verbatim}
Using the guard, we can rewrite the list comprehension using do notation as follows:
\begin{verbatim}
    Prelude> do {x <- [1..2]; guard (even x); return x}
    [2]
\end{verbatim}

\begin{exercise}
Translate the do notation for the expression {\tt{do \{x <- [1..2]; guard (even
x); return x\}}} into the bind operator and then use the definition of
{\tt{(>>=)), guard,}} and {\tt{return}} to symbolically evaluate the expression
to show how the result {\tt{[2]}} is arrived at.  Show every step.
\end{exercise}





% \noindent{}Consider the following list comprehension -- which returns a list of prime numbers less than 100.
% \begin{tabular}{l}
% {\tt{> [x | x <- [2..100], all ($\backslash$z -> rem x z /= 0) [2..(x-1)]]}} \\
% {\tt{[2,3,5,7,11,13,17,19,23,29,31,37,41,43,47,53,59,61,67,71,73,79,83,89,97]}}
% \end{tabular}

% The expression {\tt{all ($\backslash$z -> rem x z /= 0) [2..(x-1)]}} is a predicate of
% {\tt{x}} and returns {\tt{True}} if there are no divisors of {\tt{x}} between 2
% and {\tt{x-1}} -- in which case {\tt{x}} is prime -- and it returns {\tt{False}}
% otherwise.

% With Haskell's do notation, which is supported for all monads including the
% list monad, we can try to rewrite this as follows:


%% guard :: MonadPlus m => Bool -> m ()
%% guard True  = return ()
%% guard False = mzero



\end{document}


% Local Variables:
% mode:latex
% comment-column:0
% comment-start: "% "
% compile-command: "pdflatex hw21b"
% fill-column:79
% End:




