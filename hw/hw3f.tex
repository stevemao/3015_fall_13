\documentclass[11pt]{article}

\usepackage{amssymb}
\usepackage{remark}
% \usepackage{html}


\usepackage{/home/faculty/jlc/papers/tex-stuff/bussproofs}



\setlength{\textwidth}{6.5in}
\setlength{\textheight}{8.8in}
\setlength{\oddsidemargin}{0.0in}
\setlength{\evensidemargin}{0.0in}


 \newremark{theorem}{Theorem}[section]
 \newremark{Rule}{Rule}[section]
 \newremark{conjecture}{Conjecture}[section]
 \newremark{corollary}{Corollary}[section]
 \newremark{example}{Example}[section]
 \newremark{fact}{Fact}[section]
 \newremark{lemma}{Lemma}[section]
 \newremark{definition}{Definition}[section]
 \newremark{claim}{Claim}[section]
 \newremark{remark}{Remark}[section]
 \newremark{slogan}{Slogan}[section]
 \newremark{axiom}{Axiom}[section]
 \newremark{note}{Note}[]
 \newremark{problem}{Problem}[section]
 \newremark{exercise}{Exercise}[section]

\newcommand{\Proof}{{\goodbreak\noindent\bf{Proof:\ }}}
\newcommand{\qed}{\goodbreak\noindent$\Box$}

\newcommand{\byDef}[1]{\stackrel{\langle\langle{{\rm{def.\ of\ }}{#1}}\rangle\rangle}{=}}
\newcommand{\byEq}[1]{\stackrel{\langle\langle{{#1}}\rangle\rangle}{=}}
\newcommand{\by}[1]{{\small{\langle\langle{\mathit{by.\;def.\;of\;}}{#1}\rangle\rangle}}}


\newcommand{\dotcup}{{\cup\hspace{-.5em}\cdot}}
\newcommand{\barcup}{{\cup\hspace{-.68em}-}}


\newcommand{\arrow}{$\rightarrow$}
\newcommand{\nat}{{\mathbb{N}}}
\newcommand{\bool}{{\mathbb{B}}}
\newcommand{\Int}{{\mathbb{Z}}}

\newcommand{\nand}{{\overline{\wedge}}}
\newcommand{\nor}{{\overline{\vee}}}


\newcommand{\Meaning}[1]{{[{\hspace{-.13em}}[{#1}]{\hspace{-.13em}}]}}
\newcommand{\Meaningof}[2]{{[{\hspace{-.13em}}[\,{#2:#1}\,]{\hspace{-.13em}}]}}
\newcommand{\Skip}{{\bf{skip}}}
\newcommand{\Let}[2]{{\bf{Let\ }} {#1}{\bf{\ in\ }}{#2}{\bf{\ end}}}
\newcommand{\A}{{\tt{a}}}
% \newcommand{\F}{{\tt{F}}}
\newcommand{\ua}[1]{{\tt{(#1 under a)}}}
\newcommand{\Fua}{\ua{F}}
\newcommand{\aF}{{\tt{(a F)}}}
\newcommand{\Btt}{{\tt{Btt}}}
\newcommand{\Bff}{{\tt{Bff}}}
%\newcommand{\T3}{{\tt{T3}}}
%\newcommand{\F3}{{\tt{F3}}}
%\newcommand{\?3}{{\tt{?3}}}
\newcommand{\V}[1]{{\mkleeneopen{}{\tt{#1}}\mkleeneclose{}}}
\newcommand{\Neg}[1]{{\tt{N(#1)}}}
\newcommand{\I}[2]{{\tt{(#1)I(#2)}}}
\newcommand{\ie}{{\em{i.e.}}}
\newcommand{\eg}{{\em{e.g.}}}
\newcommand{\sat}[2]{{\tt{#1 \mbar{}= #2}}}
\newcommand{\nsat}[2]{{\tt{#1 \mbar{}\mneq{} #2}}}
\newcommand{\Three}{{\mBbbN{}\msubthree{}}}
%\newcommand{\Three}{{\bf{3}}}
\newcommand{\Tzero}{${\tt{0}}_{\tt{3}}$}
\newcommand{\Tone}{${\tt{1}}_{\tt{3}}$}
\newcommand{\Ttwo}{${\tt{2}}_{\tt{3}}$}
\newcommand{\msubst}{{\tt{/}}}
\newcommand{\nsequent}{{\tt{>>}}}

\newcommand{\kleene}[1]{{\mkleeneopen{}#1\mkleeneclose{}}}
\newcommand{\mfvar}[1]{{\kleene{{\tt{#1}}}}}
\newcommand{\mfnot}{{\kleene{\msim}}}
\newcommand{\mfand}{{\kleene{\mwedge}}}
\newcommand{\mfor}{{\kleene{\mvee}}}
\newcommand{\mfimp}{{\kleene{\mRightarrow}}}
\newcommand{\Knot}{$\sim_K\,$}
\newcommand{\Kand}{$\wedge_K$}
\newcommand{\Kor}{$\vee_K$}
\newcommand{\Kimp}{$\Rightarrow_K$}
\newcommand{\miff}{$\Leftrightarrow{}$}
\newcommand{\mlangle}{{$\langle$}}
\newcommand{\mrangle}{{$\rangle$}}
\newcommand{\mldots}{{$\ldots$}}
\newcommand{\mcdots}{{$\cdots$}}
%\newcommand{\rhd}{\triangleright}
\newcommand{\mrhd}{{$\triangleright$}}
\newcommand{\mRHD}{{$\triangleright_{\!R}\,$}}
\newcommand{\Sequent}[2]{{${#1}\,\vdash{}\,{#2}$}}

\newcommand{\typingRuleThree}[4]{{\begin{tabular}{c}{${#1}$} \hspace{2em} {${#2}$} \hspace{2em} {${#3}$}\\\hline{${#4}$}\end{tabular}}}
\newcommand{\typingRuleTwo}[3]{{\begin{tabular}{c}{${#1}$} \hspace{2em} {${#2}$}\\\hline{${#3}$}\end{tabular}}}
\newcommand{\typingRule}[2]{{\begin{tabular}{c}{${#1}$}{}\\\hline{${#2}$}{}\end{tabular}}}


\newcommand{\newSequentRule}[2]{{\begin{tabular}{c}{${#1}$}{}\\\hline{${#2}$}{}\end{tabular}}}

\newcommand{\sequentRule}[4]{\mbox{\raisebox{-.4ex}[4.25ex]{\begin{tabular}{rcl} \rule{0mm}{2.85ex}{$#1$} & {$\vdash$} & ${#2}$ \\\hline {\rule{0mm}{2.65ex}$#3$} & {$\vdash$} & {$#4$}\end{tabular}\\}}}
\newcommand{\AxiomRule}[3]{{\begin{tabular}{rcl} $\,$ \\\hline {\rule{0mm}{2.65ex}$#1$} & {$\vdash$} & {$#2$}\end{tabular}}(#3)}

\newcommand{\startProof}[1]{{\begin{tabular}{c} $\,$ \\\hline {\rule{0mm}{2.65ex}{#1}} \end{tabular}}}

\newcommand{\SequentRule}[5]{{\begin{tabular}{rcl} \rule{0mm}{2.85ex}{$#3$} & {$\vdash$} & ${#4}$ \\\hline {\rule{0mm}{2.65ex}$#1$} & {$\vdash$} & {$#2$}\end{tabular}}(#5)}

\newcommand{\SequentRuleTwo}[7]{{\begin{tabular}{lr} \Sequent{#3}{#4}\ &\ \Sequent{#5}{#6} \\\hline%
\multicolumn{2}{c}{{\rule{0mm}{2.65ex}{\Sequent{#1}{#2}}}} \end{tabular}}({#7})}


\newcommand{\cSequentRuleTwo}[7]{{\begin{tabular}{lr} \Sequent{${#1}$}{${#2}$}\ &\ \Sequent{${#3}$}{${#4}$} \\\hline%
\multicolumn{2}{c}{{\rule{0mm}{2.65ex}{\Sequent{${#5}$}{${#6}$}}}} \end{tabular}}({#7})}

\newcommand{\false}{\bf{false}}
\newcommand{\definedAs}{\;{\stackrel{\rm def}{=}}\;}
\newcommand{\pair}[1]{{\langle{#1}\rangle}}

\newcommand{\ttrue}{{\bf{T}}}
\newcommand{\ffalse}{{\bf{F}}}


% \renewcommand{\inr}[1]{{\tt{inr}({#1})}}
% \renewcommand{\inl}[1]{{\tt{inl}({#1})}}
% \renewcommand{\spread}[1]{{\tt{spread}}{#1}}
% \renewcommand{\decide}[1]{{\tt{decide}}({#1})}
% \renewcommand{\void}{{\tt{void}}}
% \renewcommand{\any}[1]{{\tt{any}{#1}}}
\newcommand{\Zero}{{\mathit{Zero}}}
\newcommand{\Succ}{{\mathit{Succ\,}}}
% \renewcommand{\ind}[1]{{\tt{ind}}{#1}}
% \renewcommand{\N}{\bf{N}}
% \renewcommand{\xxx}{\mbox{\hspace{.125in}}}
% \renewcommand{\Tpp}{${\bf{T}}^{++}$}
% \renewcommand{\TPP}{{\bf{T}}^{++}}
% \renewcommand{\GT}{{\bf{T}}}
% \newcommand{\abit}{$\,$}

\newcommand{\mysubtitle}[1]{{\ \\*[-.75em]{\bf{{#1}}}}}

{\obeyspaces\global\let =\ }

\newenvironment{bogustabbing}{\begin{tabbing}\={\mbox{\hspace{10em}}}\=\=\=\kill}%
{\end{tabbing}}


\newenvironment{program}{\tt\obeyspaces\begin{bogustabbing}\+\kill}{\end{bogustabbing}}
\newenvironment{program*}{\tt\obeyspaces\begin{bogustabbing}}{\end{bogustabbing}}
\newenvironment{program**}{\it\obeyspaces\begin{bogustabbing}}{\end{bogustabbing}}
\newenvironment{smallprogram*}{\hspace{2.5em}\small \it\obeyspaces\begin{bogustabbing}}{\end{bogustabbing}\vspace{-.0625in}}

\newcommand{\CASE}{\noindent{\bf{Case}}}
\newcommand{\expr}{{{\cal{E}}}}
\newcommand{\comm}{{{\cal{C}}}}
\newcommand{\mylet}[3]{{\tt{let\ }}{#1}{\tt{\ =\ }}{#2}{\tt{\ in\ }}{#3}}
\newcommand{\myfun}[2]{{\tt{fun\ }}{#1}{\tt{\ =\ }}{#2}}
\newcommand{\semwhile}[2]{{\tt{while\ }}{#1}{\tt{\ do\ }}{#2}{\tt{\ od:\,}}comm}
\newcommand{\while}[2]{{\tt{while\ }}{#1}{\tt{\ do\ }}{#2}{\tt{\ od}}}
\newcommand{\Seq}[2]{{#1}{\bf{\,;\,}}{#2}{\tt{:\,}}comm}
\newcommand{\semif}[4]{{\tt{if(}}{\Meaning{#1}({#4})}{\tt{,\ }}{\Meaning{#2}({#4})}{\tt{,\ }}{#3}{\tt{)}}}
\newcommand{\wif}[3]{{\tt{if}}{{#1}}{\tt{\ then\ }}{{#2}}{\tt{\ else\ }}{#3}{\tt{\ fi}}:\,comm}
\newcommand{\wnot}[1]{\neg{#1}}
\newcommand{\wassign}[2]{{\tt{loc}}_{#1} := {#2}}
\newcommand{\wderef}[1]{{\tt{@loc}}_{#1} }
\newcommand{\loc}[1]{${\tt{loc}}_{#1}$}
% \newcommand{\semif1}[4]{{\tt{if(}}{\Meaning{#1}({#4})}{\tt{,\ }}{\Meaning{#2}({#4})}{\tt{,\ }}{\Meaning{#3}({#4})}{\tt{)}}}

\newcommand{\mkleeneopen}{{\boldmath{^{\lceil}}}}
\newcommand{\mkleeneclose}{{\boldmath{^{\rceil}}}}
\newcommand{\kquote}[1]{{\mkleeneopen{}{{#1}}\mkleeneclose}}

\newcommand{\homework}[2]{\ \\\vspace{-1.25in}\\%
{\bf{HW {#1}}} \hfill {\bf{Prof. Caldwell}} \\%
{\bf{Due:}} {#2} 2013 \hfill {\bf{COSC 3015}}\ \\}

\newcommand{\ite}[3]{{\bf{if\ }}{#1}{\bf{\ then\ }}{#2}{\bf{\ else\ }}{#3}{\bf{\ fi}}}
\newcommand{\fun}{{\bf{fun\, }}}
\newcommand{\prog}[3]{{#1}{\bf{\ in \ }}{#2}}



\newcommand{\store}[1]{\langle{}{#1}\rangle}

\newcommand{\alphaeq}{\,{=_\alpha}\,}
\newcommand{\xleaf}{{\rm{Leaf}}}
\newcommand{\xnode}{{\rm{Node}}}
\newcommand{\vbar}{{\;\;{\bf{|}}\;\;}}
\newcommand{\abit}{{\mbox{\hspace{1em}}}}


\newcommand{\assign}[2]{{{#1} {\bf{:=}} {#2}}}
\newcommand{\sequence}[2]{{#1}\,{\bf{;}}\,{#2}}

\newcommand{\F}[1]{{\cal{F}}_{#1}}
\newcommand{\harrow}{{\tt{-\!\!\!>}}}
\newcommand{\smallsection}{{\goodbreak\[ \star {\hspace{.35in}} \star {\hspace{.35in}} \star \] \ \\}}
\newcommand{\append}{\,{\texttt{++}}\,}
\newcommand{\nil}{{\texttt{[{\hspace{.125em}}]}}}

\newcommand{\defof}[1]{\stackrel{<\!\!<{\textrm{def. of }}{#1}>\!\!>}{=}}
\newcommand{\pnote}[1]{{$\langle\!\langle$ {#1} $\rangle\!\rangle$}}

\newcommand{\spread}[3]{{\mathit{spread}({#1};{#2}.{#3})}}


\newcommand{\sfa}{{\sf{a}}}
\newcommand{\sfb}{{\sf{b}}}
\newcommand{\sfc}{{\sf{c}}}
\newcommand{\sfd}{{\sf{d}}}
\newcommand{\sfe}{{\sf{e}}}
\newcommand{\sff}{{\sf{f}}}


\begin{document}
\homework{3}{1 September}


\section{Today's Lecture}

We noted in class that there are two languages at work when using Haskell.
There is the {\em{computation language}} of executable terms (programs) and the
{\em{type language}} for describing types.  For full Haskell, these languages
are very rich. IN this note we just describe the core of these languages that
include the types of pairs and functions.

\subsection{The Core Language of Types}

If $TVar=\{a,b,c,\cdots\}$ is the collection of type variables, then
the language of types is give by the following grammar:
\[\begin{array}{ll}
{\cal{T}} & ::=  x \;\; \mid \;\; (T,T') \;\; \mid \;\; T \rightarrow T' \\
&{\mathrm{where\ \ }} x\in{}TVar {\mathrm{\ is\ a\ type\ variable,\ and}}\\
&{\hspace{3.35em}} T,T'\in {\cal{T}} {\mathrm{\ are\ previously\ constructed\ type\ expressions.}}\\
\end{array}\]

Thus, standalone type variables are type expressions, and if $T$ and $T'$ are
type expressions, then so are $(T,T')$ (denoting the type of pairs whose first
elements have type $T$ and whose second elements have type $T'$. Also, if $T$
and $T'$ are type expressions, then so is $T\rightarrow{}T'$ (which denoted the
type of functions from type $T$ to type $T'$.)

You can think of $(\_,\_)$ as the constructor that maps two type expressions
$T$ and $T'$ to the expression $(T,T')$ denoting the type of pairs.  Similarly,
$\rightarrow$ is the (infix) constructor for function type expressions mapping
type expressions $T$ and $T'$ to the function tye expression
$T\rightarrow{}T'$.

We adopt the convention that $T_1\rightarrow{}T_2\rightarrow{}T_3$ is to be
interpreted as $T_1\rightarrow{}(T_2\rightarrow{}T_3)$.  We say that the type
constructor $\rightarrow$ {\em{associates to the right.}}.



Of course the full Haskell language has a much richer language of types than
the core presented here, but these are fundamental.

\subsection{The Core Computation Language}

If $Var = \{x,y,z,\cdots\}$ is the set of variable name in the computation
language, then Haskell's core computation language (call it $\cal{H}$) is
defined as follows:

\[\begin{array}{ll}
{\cal{H}} & ::=  x \;\; \mid \;\; (t,t') \;\; \mid \;\; (\backslash x \rightarrow t) \;\;|\;\; t\;t' \\
&{\mathrm{where\ \ }} x\in{}Var {\mathrm{\ is\ a\ computation\ variable,\ and}}\\
&{\hspace{3.35em}} t,t'\in {\cal{H}} {\mathrm{\ are\ previously\ constructed\ Haskell\ terms.}}\\
\end{array}\]

We sometimes call executable Haskell expressions {\em{terms}}.  If $t$ and $t'$
are terms then $(t,t')$ is the pair term whose first element is $t$ and whose
second element if $t'$.  The term $(\backslash x \rightarrow t)$ is called a
lambda abstraction, a lambda term or simply an abstraction.  This bit of syntax
allows us to write a term whose value is a function (without having to assign
it a name.)  If $t$ and $t'$ are terms then the term $t\;t'$ denotes function
application -- apply $t$ to $t'$.  It only makes computational sense if $t$ is
a function.

We adopt the convention that the application $t_1\;{}t_2\;{}t_3$ is to be
interpreted as $(t_1\;{}t_2)\;{}t_3$.  We say that function application
{\em{associates to the left.}}


\subsection{Linking the Computation Language and the Type Language}

If $t$ is a term and $T$ is a type expression we write $t::T$ to mean that the
computational term $t$ has type $T$.  This notion is the basis for type
checking which we will go into in more depth later in the course.


\subsection{Equality of functions}
Recall from class that functions are equal if and only if they return equal
results on all inputs (this equality is called {\em{extensionality}}.)

More formally, we can write:
\begin{definition}{{\bf{(extensionality)}}}
If $f,g::a\rightarrow{}b$ then they are defined to be (extensionally) equal as follows:
\[ f = g \; \definedAs\; \forall{}x\!:\!a.\; f(x) = g(x) \]
\end{definition}

So, we can prove two functions $f$ and $g$ are equal by choosing an arbitrary
$x$ of type $a$ and showing $f(x) = g(x)$.

\ \\ For example, if $f(x) = |x|$ (the absolute value) and $g(x) = x$ then,
$f\not={}g$ when we consider them as functions in the type $\Int \rightarrow
\Int$ since $f(-2) = 2$ and $g(-2)=-2$.  But, if we think of these functions as
elements\footnote{Recall that the natural numbers are
$\nat=\{0,1,2,\cdots\}$.} of $\nat\rightarrow\nat$, they are equal.  To see
this, choose an arbitrary $x\in\nat$ and argue that $f(x) = g(x)$ {\em{i.e.}}
that $|x|=x$.  But this is trivially true when $x \ge 0$, which follows because
$x\in\nat$.  \ \\

Recall the following Haskell definitions.
\begin{definition}{plus}
\[\begin{array}{l}
plus :: (Integer,Integer) \rightarrow Integer \\
plus(x,y) = x + y 
\end{array}\]
\end{definition}


\begin{definition}{plusc}
\[\begin{array}{l}
plusc :: Integer \rightarrow (Integer \rightarrow Integer) \\
plusc \; x \; y = x + y \\
\end{array}\]
\end{definition}

\begin{definition}{curry}
\[\begin{array}{l}
curry :: ((a,b) \rightarrow c) \rightarrow (a \rightarrow (b \rightarrow c))\\
curry\; f\; x\; y = f\; (x,y) 
\end{array}\]
\end{definition}

\begin{definition}{uncurry}
\[\begin{array}{l}
uncurry :: (a \rightarrow (b \rightarrow c)) \rightarrow ((a,b) \rightarrow c) \\
uncurry \;f\; (x,y) = f\; x\; y 
\end{array}\]
\end{definition}

In class we proved the following theorem:

\begin{theorem}{}
\[curry\;\; plus = plusc\]
\Proof Note that both $curry\;plus$ and $plusc$ have the type
$Integer\rightarrow (Integer \rightarrow Integer)$ {\em{i.e.}} they are
functions mapping an $Integers$ to a function of type
$Integer\rightarrow{}Integer$.  This means we can use extensionality to prove
they are equal as functions.  We must show the following.
\[\forall{}x:Integer.\; curry\;\; plus\; x = plusc\; x\]
Assume $x$ is an arbitrary $Integer$. Then we must show
\[curry\;\; plus\; x = plusc\; x\]
But $curry\;\; plus \; x$ and $plusc\; x$ are functions of type
$Integer\rightarrow{}Integer$. To show they are equal we use extensionality a
second time, to show:
\[\forall{}y:Integer.\; curry\;\; plus\; x\; y = plusc\; x\; y\]
We choose an arbitrary $y$ of type $Integer$ and  show the following
\[curry\;\; plus\; x\; y = plusc\; x\; y\]
Starting with the left side of the equality we get the following:
\[curry\;\; plus\; x\; y \stackrel{\pair{\pair{def.\,of\, curry}}}{=} plus\; (x,y)   \stackrel{\pair{\pair{def.\,of\, plus}}}{=} x + y \]
On the right side of the equality, we have the following:
\[plusc\; x\; y \stackrel{def.\,of\, plusc}{=} x + y \]
Since both sides of the equality are equal to $x+y$ we see that the functions
are equal.  \qed
\end{theorem}
\ \\

\section{Homework Problems}
\begin{problem}
Prove the following theorem using extensionality.

\begin{theorem}{\bf{[uncurry-plusc]}} 
\[uncurry\;plusc = plus \]
\end{theorem}

Hint: The functions $(uncurry\;plusc)$ and $plus$ have the type
$(Integer,Integer)\rightarrow{}Integer$.  Extensionality for functions $f$ and
$g$ of this type can most conveniently  be written as
\[\forall{}(x,y)::(Integer,Integer).\; f (x,y) = g (x,y)\]


\end{problem}

\begin{problem}

Consider the following function definitions: 

\begin{definition}{flip}
\[\begin{array}{l}
flip :: (a \rightarrow b \rightarrow c) \rightarrow (b \rightarrow a \rightarrow c) \\
flip  \;f\; x\; y = f\; y\; x 
\end{array}\]
\end{definition}

Prove the following theorem using extensionality.

\begin{theorem}{\bf{[flip plusc]}} 
\[flip\;plusc = plusc \]
\end{theorem}

Hints: What is the types of $flip\;plusc$ and $plusc$. You will need the fact
that addition is commutative {\em{i.e.}}  $x + y = y + x$.


\end{problem}
% \newpage
% \section{Function Composition}


% \goodbreak\noindent{}Now, consider the following two definitions.
% \vspace{.06125in}

% \begin{definition}{Function Composition}
% \[\begin{array}{l}
% compose :: (b\rightarrow{}c)\rightarrow (a\rightarrow{}b) \rightarrow (a \rightarrow c)\\
% compose\; f\; g \; x = f (g\; x)
% \end{array}
% \]
% \end{definition}

% In Haskell, $(compose\; f\; g)$ is written $(f\;.\;g)$, we will write
% $(f\circ{}g)$ here.

% \begin{definition}{Identity function}
% \[\begin{array}{l}
% id :: a \rightarrow {}a\\
% id\; x = x
% \end{array}
% \]
% \end{definition}

% \begin{theorem}{Compose-id-right}
% \[\forall{}f:a\rightarrow{}b.\; (f\circ{} id) = f\]
% \Proof
% Choose an arbitrary function $f::a\rightarrow{}b$ and show that 
% \[ f\circ{}id = f\]
% Since $f$ has type $a\rightarrow{}b$ and $id$ has type $a\rightarrow{}a$ we can
% see that $f\circ{}id$ has the same type. (why?)  We use extensionality to show
% that these two functions are equal, {\em{i.e.}} we must show:
% \[\forall{}x:a.\, (f\circ{}id)\,x =f\; x\]
% Chose an arbitrary $x$ in type $a$ and show
% \[(f\circ{}id)\,x =f\; x\]
% By definition of compose and definition of $id$ we get the following sequence of equalities.
% \[(f\circ{}id)x = compose\;f\;id\;x = f\,(id\;x) = f \; x\]
% This completes the proof.
% \qed

% \end{theorem}


% \begin{problem}
% Prove the following theorem. 
% \begin{theorem}{\bf{[compose-id-left]}} 
% \[\forall{}f:a \rightarrow b.\;id\circ{} f = f\]
% \end{theorem}

% Hint: First argue that $id\circ{}f$ and $f$ have the same type (note that
% $id::b\rightarrow{}b$) and then use extensionality.


% \end{problem}





\end{document}
% Local Variables:
% mode:latex
% comment-column:0
% comment-start: "% "
% compile-command: "pdflatex hw3f"
% fill-column:79
% End:

