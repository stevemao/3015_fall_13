\documentclass[11pt]{article}

\usepackage{amssymb}
\usepackage{remark}
% \usepackage{html}


\usepackage{/home/faculty/jlc/papers/tex-stuff/bussproofs}



\setlength{\textwidth}{6.5in}
\setlength{\textheight}{8.8in}
\setlength{\oddsidemargin}{0.0in}
\setlength{\evensidemargin}{0.0in}


 \newremark{theorem}{Theorem}[section]
 \newremark{Rule}{Rule}[section]
 \newremark{conjecture}{Conjecture}[section]
 \newremark{corollary}{Corollary}[section]
 \newremark{example}{Example}[section]
 \newremark{fact}{Fact}[section]
 \newremark{lemma}{Lemma}[section]
 \newremark{definition}{Definition}[section]
 \newremark{claim}{Claim}[section]
 \newremark{remark}{Remark}[section]
 \newremark{slogan}{Slogan}[section]
 \newremark{axiom}{Axiom}[section]
 \newremark{note}{Note}[]
 \newremark{problem}{Problem}[section]
 \newremark{exercise}{Exercise}[section]

\newcommand{\Proof}{{\goodbreak\noindent\bf{Proof:\ }}}
\newcommand{\qed}{\goodbreak\noindent$\Box$}

\newcommand{\byDef}[1]{\stackrel{\langle\langle{{\rm{def.\ of\ }}{#1}}\rangle\rangle}{=}}
\newcommand{\byEq}[1]{\stackrel{\langle\langle{{#1}}\rangle\rangle}{=}}
\newcommand{\by}[1]{{\small{\langle\langle{\mathit{by.\;def.\;of\;}}{#1}\rangle\rangle}}}


\newcommand{\dotcup}{{\cup\hspace{-.5em}\cdot}}
\newcommand{\barcup}{{\cup\hspace{-.68em}-}}


\newcommand{\arrow}{$\rightarrow$}
\newcommand{\nat}{{\mathbb{N}}}
\newcommand{\bool}{{\mathbb{B}}}
\newcommand{\Int}{{\mathbb{Z}}}

\newcommand{\nand}{{\overline{\wedge}}}
\newcommand{\nor}{{\overline{\vee}}}


\newcommand{\Meaning}[1]{{[{\hspace{-.13em}}[{#1}]{\hspace{-.13em}}]}}
\newcommand{\Meaningof}[2]{{[{\hspace{-.13em}}[\,{#2:#1}\,]{\hspace{-.13em}}]}}
\newcommand{\Skip}{{\bf{skip}}}
\newcommand{\Let}[2]{{\bf{Let\ }} {#1}{\bf{\ in\ }}{#2}{\bf{\ end}}}
\newcommand{\A}{{\tt{a}}}
% \newcommand{\F}{{\tt{F}}}
\newcommand{\ua}[1]{{\tt{(#1 under a)}}}
\newcommand{\Fua}{\ua{F}}
\newcommand{\aF}{{\tt{(a F)}}}
\newcommand{\Btt}{{\tt{Btt}}}
\newcommand{\Bff}{{\tt{Bff}}}
%\newcommand{\T3}{{\tt{T3}}}
%\newcommand{\F3}{{\tt{F3}}}
%\newcommand{\?3}{{\tt{?3}}}
\newcommand{\V}[1]{{\mkleeneopen{}{\tt{#1}}\mkleeneclose{}}}
\newcommand{\Neg}[1]{{\tt{N(#1)}}}
\newcommand{\I}[2]{{\tt{(#1)I(#2)}}}
\newcommand{\ie}{{\em{i.e.}}}
\newcommand{\eg}{{\em{e.g.}}}
\newcommand{\sat}[2]{{\tt{#1 \mbar{}= #2}}}
\newcommand{\nsat}[2]{{\tt{#1 \mbar{}\mneq{} #2}}}
\newcommand{\Three}{{\mBbbN{}\msubthree{}}}
%\newcommand{\Three}{{\bf{3}}}
\newcommand{\Tzero}{${\tt{0}}_{\tt{3}}$}
\newcommand{\Tone}{${\tt{1}}_{\tt{3}}$}
\newcommand{\Ttwo}{${\tt{2}}_{\tt{3}}$}
\newcommand{\msubst}{{\tt{/}}}
\newcommand{\nsequent}{{\tt{>>}}}

\newcommand{\kleene}[1]{{\mkleeneopen{}#1\mkleeneclose{}}}
\newcommand{\mfvar}[1]{{\kleene{{\tt{#1}}}}}
\newcommand{\mfnot}{{\kleene{\msim}}}
\newcommand{\mfand}{{\kleene{\mwedge}}}
\newcommand{\mfor}{{\kleene{\mvee}}}
\newcommand{\mfimp}{{\kleene{\mRightarrow}}}
\newcommand{\Knot}{$\sim_K\,$}
\newcommand{\Kand}{$\wedge_K$}
\newcommand{\Kor}{$\vee_K$}
\newcommand{\Kimp}{$\Rightarrow_K$}
\newcommand{\miff}{$\Leftrightarrow{}$}
\newcommand{\mlangle}{{$\langle$}}
\newcommand{\mrangle}{{$\rangle$}}
\newcommand{\mldots}{{$\ldots$}}
\newcommand{\mcdots}{{$\cdots$}}
%\newcommand{\rhd}{\triangleright}
\newcommand{\mrhd}{{$\triangleright$}}
\newcommand{\mRHD}{{$\triangleright_{\!R}\,$}}
\newcommand{\Sequent}[2]{{${#1}\,\vdash{}\,{#2}$}}

\newcommand{\typingRuleThree}[4]{{\begin{tabular}{c}{${#1}$} \hspace{2em} {${#2}$} \hspace{2em} {${#3}$}\\\hline{${#4}$}\end{tabular}}}
\newcommand{\typingRuleTwo}[3]{{\begin{tabular}{c}{${#1}$} \hspace{2em} {${#2}$}\\\hline{${#3}$}\end{tabular}}}
\newcommand{\typingRule}[2]{{\begin{tabular}{c}{${#1}$}{}\\\hline{${#2}$}{}\end{tabular}}}


\newcommand{\newSequentRule}[2]{{\begin{tabular}{c}{${#1}$}{}\\\hline{${#2}$}{}\end{tabular}}}

\newcommand{\sequentRule}[4]{\mbox{\raisebox{-.4ex}[4.25ex]{\begin{tabular}{rcl} \rule{0mm}{2.85ex}{$#1$} & {$\vdash$} & ${#2}$ \\\hline {\rule{0mm}{2.65ex}$#3$} & {$\vdash$} & {$#4$}\end{tabular}\\}}}
\newcommand{\AxiomRule}[3]{{\begin{tabular}{rcl} $\,$ \\\hline {\rule{0mm}{2.65ex}$#1$} & {$\vdash$} & {$#2$}\end{tabular}}(#3)}

\newcommand{\startProof}[1]{{\begin{tabular}{c} $\,$ \\\hline {\rule{0mm}{2.65ex}{#1}} \end{tabular}}}

\newcommand{\SequentRule}[5]{{\begin{tabular}{rcl} \rule{0mm}{2.85ex}{$#3$} & {$\vdash$} & ${#4}$ \\\hline {\rule{0mm}{2.65ex}$#1$} & {$\vdash$} & {$#2$}\end{tabular}}(#5)}

\newcommand{\SequentRuleTwo}[7]{{\begin{tabular}{lr} \Sequent{#3}{#4}\ &\ \Sequent{#5}{#6} \\\hline%
\multicolumn{2}{c}{{\rule{0mm}{2.65ex}{\Sequent{#1}{#2}}}} \end{tabular}}({#7})}


\newcommand{\cSequentRuleTwo}[7]{{\begin{tabular}{lr} \Sequent{${#1}$}{${#2}$}\ &\ \Sequent{${#3}$}{${#4}$} \\\hline%
\multicolumn{2}{c}{{\rule{0mm}{2.65ex}{\Sequent{${#5}$}{${#6}$}}}} \end{tabular}}({#7})}

\newcommand{\false}{\bf{false}}
\newcommand{\definedAs}{\;{\stackrel{\rm def}{=}}\;}
\newcommand{\pair}[1]{{\langle{#1}\rangle}}

\newcommand{\ttrue}{{\bf{T}}}
\newcommand{\ffalse}{{\bf{F}}}


% \renewcommand{\inr}[1]{{\tt{inr}({#1})}}
% \renewcommand{\inl}[1]{{\tt{inl}({#1})}}
% \renewcommand{\spread}[1]{{\tt{spread}}{#1}}
% \renewcommand{\decide}[1]{{\tt{decide}}({#1})}
% \renewcommand{\void}{{\tt{void}}}
% \renewcommand{\any}[1]{{\tt{any}{#1}}}
\newcommand{\Zero}{{\mathit{Zero}}}
\newcommand{\Succ}{{\mathit{Succ\,}}}
% \renewcommand{\ind}[1]{{\tt{ind}}{#1}}
% \renewcommand{\N}{\bf{N}}
% \renewcommand{\xxx}{\mbox{\hspace{.125in}}}
% \renewcommand{\Tpp}{${\bf{T}}^{++}$}
% \renewcommand{\TPP}{{\bf{T}}^{++}}
% \renewcommand{\GT}{{\bf{T}}}
% \newcommand{\abit}{$\,$}

\newcommand{\mysubtitle}[1]{{\ \\*[-.75em]{\bf{{#1}}}}}

{\obeyspaces\global\let =\ }

\newenvironment{bogustabbing}{\begin{tabbing}\={\mbox{\hspace{10em}}}\=\=\=\kill}%
{\end{tabbing}}


\newenvironment{program}{\tt\obeyspaces\begin{bogustabbing}\+\kill}{\end{bogustabbing}}
\newenvironment{program*}{\tt\obeyspaces\begin{bogustabbing}}{\end{bogustabbing}}
\newenvironment{program**}{\it\obeyspaces\begin{bogustabbing}}{\end{bogustabbing}}
\newenvironment{smallprogram*}{\hspace{2.5em}\small \it\obeyspaces\begin{bogustabbing}}{\end{bogustabbing}\vspace{-.0625in}}

\newcommand{\CASE}{\noindent{\bf{Case}}}
\newcommand{\expr}{{{\cal{E}}}}
\newcommand{\comm}{{{\cal{C}}}}
\newcommand{\mylet}[3]{{\tt{let\ }}{#1}{\tt{\ =\ }}{#2}{\tt{\ in\ }}{#3}}
\newcommand{\myfun}[2]{{\tt{fun\ }}{#1}{\tt{\ =\ }}{#2}}
\newcommand{\semwhile}[2]{{\tt{while\ }}{#1}{\tt{\ do\ }}{#2}{\tt{\ od:\,}}comm}
\newcommand{\while}[2]{{\tt{while\ }}{#1}{\tt{\ do\ }}{#2}{\tt{\ od}}}
\newcommand{\Seq}[2]{{#1}{\bf{\,;\,}}{#2}{\tt{:\,}}comm}
\newcommand{\semif}[4]{{\tt{if(}}{\Meaning{#1}({#4})}{\tt{,\ }}{\Meaning{#2}({#4})}{\tt{,\ }}{#3}{\tt{)}}}
\newcommand{\wif}[3]{{\tt{if}}{{#1}}{\tt{\ then\ }}{{#2}}{\tt{\ else\ }}{#3}{\tt{\ fi}}:\,comm}
\newcommand{\wnot}[1]{\neg{#1}}
\newcommand{\wassign}[2]{{\tt{loc}}_{#1} := {#2}}
\newcommand{\wderef}[1]{{\tt{@loc}}_{#1} }
\newcommand{\loc}[1]{${\tt{loc}}_{#1}$}
% \newcommand{\semif1}[4]{{\tt{if(}}{\Meaning{#1}({#4})}{\tt{,\ }}{\Meaning{#2}({#4})}{\tt{,\ }}{\Meaning{#3}({#4})}{\tt{)}}}

\newcommand{\mkleeneopen}{{\boldmath{^{\lceil}}}}
\newcommand{\mkleeneclose}{{\boldmath{^{\rceil}}}}
\newcommand{\kquote}[1]{{\mkleeneopen{}{{#1}}\mkleeneclose}}

\newcommand{\homework}[2]{\ \\\vspace{-1.25in}\\%
{\bf{HW {#1}}} \hfill {\bf{Prof. Caldwell}} \\%
{\bf{Due:}} {#2} 2013 \hfill {\bf{COSC 3015}}\ \\}

\newcommand{\ite}[3]{{\bf{if\ }}{#1}{\bf{\ then\ }}{#2}{\bf{\ else\ }}{#3}{\bf{\ fi}}}
\newcommand{\fun}{{\bf{fun\, }}}
\newcommand{\prog}[3]{{#1}{\bf{\ in \ }}{#2}}



\newcommand{\store}[1]{\langle{}{#1}\rangle}

\newcommand{\alphaeq}{\,{=_\alpha}\,}
\newcommand{\xleaf}{{\rm{Leaf}}}
\newcommand{\xnode}{{\rm{Node}}}
\newcommand{\vbar}{{\;\;{\bf{|}}\;\;}}
\newcommand{\abit}{{\mbox{\hspace{1em}}}}


\newcommand{\assign}[2]{{{#1} {\bf{:=}} {#2}}}
\newcommand{\sequence}[2]{{#1}\,{\bf{;}}\,{#2}}

\newcommand{\F}[1]{{\cal{F}}_{#1}}
\newcommand{\harrow}{{\tt{-\!\!\!>}}}
\newcommand{\smallsection}{{\goodbreak\[ \star {\hspace{.35in}} \star {\hspace{.35in}} \star \] \ \\}}
\newcommand{\append}{\,{\texttt{++}}\,}
\newcommand{\nil}{{\texttt{[{\hspace{.125em}}]}}}

\newcommand{\defof}[1]{\stackrel{<\!\!<{\textrm{def. of }}{#1}>\!\!>}{=}}
\newcommand{\pnote}[1]{{$\langle\!\langle$ {#1} $\rangle\!\rangle$}}

\newcommand{\spread}[3]{{\mathit{spread}({#1};{#2}.{#3})}}


\newcommand{\sfa}{{\sf{a}}}
\newcommand{\sfb}{{\sf{b}}}
\newcommand{\sfc}{{\sf{c}}}
\newcommand{\sfd}{{\sf{d}}}
\newcommand{\sfe}{{\sf{e}}}
\newcommand{\sff}{{\sf{f}}}


\begin{document}
\homework{18}{3 December}

\section{Type Inference}

Recall the type of terms.

\begin{program*}
\>  data Term = V String \\
\>            | Ap Term Term \\
\>            | Abs String Term \\
\end{program*}

\noindent{}The data-type {\tt{Type}} with products is:

\begin{program*}
\>   data Type = TyVar String | Arrow Type Type deriving Eq \\
\end{program*}

\subsection{Proof Rules}

Sequents in the system (which represent the state of a derivation) are of the form

Sequents in the system (which represent the state of a type derivation) are of
the form:

\[\Gamma,E \vdash M : T\]
In this structure, $\Gamma$ is a {\em{context}} representing a state of
knowledge about the types of some variables.  Contexts have the form:
 \[\Gamma = [x_1:\tau_1,\cdots{},x_k:\tau_k]\]
where the $x_i$'s are variables and $\tau_i$'s are types.

$E$ is a list of constraints between pairs
of types and in the rules is presented as follows:
\[E=\{\tau_{(1,1)} = \tau_{(1,2)}, \cdots , \tau_{(k,1)} = \tau_{(k,2)}\}\]
wher $\tau_{i,j}$'s are types.  

We write $\Gamma\backslash{}x$ to denote the list obtained from $\Gamma$ by
deleting all pairs whose first element is $x$.

As presented in the last homework, The proof rules for Wand's type
inference system are given as follows:


\AxiomRule{\Gamma,\{\alpha=\tau\}}{x:\tau}{Ax} {\hspace{.25in}} if $(x,\alpha)\in\Gamma$.
\vspace{.125in}\\

\SequentRule{\Gamma,E\cup{}\{\tau=\alpha\rightarrow\beta\}}{\lambda{}x.M : \tau}{[x:\alpha]{\tt{++}}(\Gamma\backslash{x}),\,E}{M:\beta}{Abs}
{\hspace{.25in}} where $\alpha$ and $\beta$ are fresh.
\vspace{.125in}\\

\SequentRuleTwo{\Gamma,E_1\cup{}E_2}{MN:\tau}{\Gamma,E_1}{M:\alpha\rightarrow\tau}{\Gamma,E_2}{N:\alpha}{App}
{\hspace{.25in}} where $\alpha$ is fresh.
\vspace{.125in}\\

A derivation in this system is a tree of instances of these rules where the
leaves of the tree are all instances of the (Ax) rule.  To construct a proof
that a closed term (no free variables) (say $M$) has a type, we postulate that
$M$ has some type (say $\alpha$) and proceed by recursion on the structure of
$M$ to show

\[ \exists{}E.[(Type,Type)]. {\rm{\ such \ that\ the \ sequent\ }}[],E\vdash M:\alpha {\rm{\  is \ derivable.}}\]

To find $E$, we use the proof rules above to try to construct a derivation
(leaving the $E$'s blank to start) and then propagate the constraints in the
$E$'s back down through the derivation tree from the leaves.

\begin{example}
Here is an example of a derivation that $\lambda{}x.x$ has a type by starting
with the sequent of the form  $[],\{??\}\vdash (\lambda{}x.x) :\tau$.  The term is an abstraction
so we apply the rule (Abs).

\begin{center}
\AxiomC{$[x:\alpha],E\vdash x : \beta$}
\LeftLabel{}\RightLabel{(Abs)}
\UnaryInfC{$[],\{\tau=\alpha \rightarrow \beta\}\cup{}E\vdash (\lambda{}x.x) :\tau$}
\DisplayProof
\end{center}

But if we fill in the set $E$ with the constraint $\tau=\alpha$, we have an
instance of the Axiom rule.


\begin{center}
\AxiomC{$E=\{\beta=\alpha\}$}
\LeftLabel{}\RightLabel{(Ax)}
\UnaryInfC{$[x:\alpha],E\vdash x : \beta$}
\LeftLabel{}\RightLabel{(Abs)}
\UnaryInfC{$[],\{\tau=\alpha \rightarrow \beta\}\cup{}E\vdash (\lambda{}x.x) :\tau$}
\DisplayProof
\end{center}

\noindent{}If we completely instantiate the sets $E$ we get the following complete derivation.

\begin{center}
\AxiomC{}
\LeftLabel{}\RightLabel{(Ax)}
\UnaryInfC{$[x:\alpha],\{\beta=\alpha\}\vdash x : \beta$}
\LeftLabel{}\RightLabel{(Abs)}
\UnaryInfC{$[],\{\tau=\alpha \rightarrow \beta,\beta=\alpha\}\vdash (\lambda{}x.x) :\tau$}
\DisplayProof
\end{center}

\end{example}


The fact that there is a derivation indicates that the term $(\lambda{}x.x)$
does have a type. We use the constraint set $E$ to actually determine the type of
$\lambda{}x.x$.  To do this, we unify the set $E$ and apply the resulting
substitution to the type $\tau$.  For this case, when we unify $E$ we get the
substitution $[\tau := \alpha\rightarrow\alpha, \beta:=\alpha]$.  Applying this
substitution to $\tau$ we determine that
$(\lambda{}x.x):\alpha\rightarrow\alpha$.



We can also do type derivations for terms containing free variables if we
assume those free variables do have types.

\begin{example}
Consider the term $y(\lambda{}x.x)$. This should have a type if
$y:(\alpha\rightarrow\alpha)\rightarrow\beta$.

We start by trying to show there is some $E$ such that there is a derivation of the sequent

\[[y:(\alpha\rightarrow\alpha)\rightarrow\beta],E\vdash y(\lambda{}x.x):\tau\]

\noindent{}Since the term is an application, we use the (Ap) rule.

\begin{center}
\AxiomC{$[y:(\alpha\rightarrow\alpha)\rightarrow\beta],E_1\vdash y :\alpha'\rightarrow\tau$}
\AxiomC{$[y:(\alpha\rightarrow\alpha)\rightarrow\beta],E_2\vdash (\lambda{}x.x):\alpha'$}
\LeftLabel{}\RightLabel{(Abs)}
\BinaryInfC{$[y:(\alpha\rightarrow\alpha)\rightarrow\beta],E_1\cup{}E_2\vdash y(\lambda{}x.x):\tau$}
\DisplayProof
\end{center}

\noindent{}The left branch is an instance of an axiom because there is an entry for the variable $y$ in the context.

\small{
\begin{center}
\AxiomC{$E_1 = \{\alpha'\rightarrow\tau=(\alpha\rightarrow\alpha)\rightarrow\beta\}$}
\LeftLabel{}\RightLabel{(Ax)}
\UnaryInfC{$[y:(\alpha\rightarrow\alpha)\rightarrow\beta],E_1\vdash y :\alpha'\rightarrow\tau$}
\AxiomC{$[y:(\alpha\rightarrow\alpha)\rightarrow\beta],E_2\vdash (\lambda{}x.x):\alpha'$}
\LeftLabel{}\RightLabel{(Abs)}
\BinaryInfC{$[y:(\alpha\rightarrow\alpha)\rightarrow\beta],E_1\cup{}E_2\vdash y(\lambda{}x.x):\tau$}
\DisplayProof
\end{center}
}

\noindent{}On the right branch we rebuild the proof given above.

\small{
\begin{center}
\AxiomC{$E_1 = \{\alpha'\rightarrow\tau=(\alpha\rightarrow\alpha)\rightarrow\beta\}$}
\LeftLabel{}\RightLabel{(Ax)}
\UnaryInfC{$[y:(\alpha\rightarrow\alpha)\rightarrow\beta],E_1\vdash y :\alpha'\rightarrow\tau$}
% \AxiomC{$[y:(\alpha\rightarrow\alpha)\rightarrow\beta],E_2\vdash (\lambda{}x.x):\alpha'$}
\AxiomC{$E_3=\{\beta'=\alpha''\}$}
\LeftLabel{}\RightLabel{(Ax)}
\UnaryInfC{$[x:\alpha'',y:(\alpha\rightarrow\alpha)\rightarrow\beta],E_3\vdash x : \beta'$}
\LeftLabel{}\RightLabel{(Abs)}
\UnaryInfC{$[y:(\alpha\rightarrow\alpha)\rightarrow\beta],E_2=(\{\alpha'=\alpha'' \rightarrow \beta'\}\cup{}E_3)\vdash (\lambda{}x.x) :\alpha'$}
\LeftLabel{}\RightLabel{(Abs)}
\BinaryInfC{$[y:(\alpha\rightarrow\alpha)\rightarrow\beta],E=(E_1\cup{}E_2)\vdash y(\lambda{}x.x):\tau$}
\DisplayProof
\end{center}
}

Putting together the constraints, we get the following set:

\[\begin{array}{lcl}
E & = & E_1 \cup E_2 \\
  & = & \{\alpha'\rightarrow\tau=(\alpha\rightarrow\alpha)\rightarrow\beta\} \cup (\{\alpha'=\alpha'' \rightarrow \beta'\}\cup{}E_3)\\
  & = & \{\alpha'\rightarrow\tau=(\alpha\rightarrow\alpha)\rightarrow\beta\} \cup (\{\alpha'=\alpha'' \rightarrow \beta'\}\cup{}\{\beta'=\alpha''\})\\
  & = & \{\alpha'\rightarrow\tau=(\alpha\rightarrow\alpha)\rightarrow\beta,\alpha'=\alpha'' \rightarrow \beta', \beta'=\alpha''\})\\
\end{array}\]

Unification of this results in the substitution:

\[s = [a' := (b' \rightarrow b'),t := b,a := b',a'' := b']\]

When $s$ is applied to $\tau$ we get the type $\beta$, as expected.

\end{example}



\subsection{Implementation}


In Haskell we encode contexts as list of type
{\tt{[(String,Type)]}}. Constraint sets are represented in the Haskell
implementaiton as a list of type {\tt{[(Type,Type)]}}. $M$ denotes a
lambda-term, and in Haskell is represented by elements of the data-type
{\tt{Term}}.  $T$ denotes a type and is represented in Haskell by elements of
the data-type {\tt{Type}}.


The implementation Here is the type of the {\tt{infer\_type}} function:
\begin{program*}
\> infer\_type :: [(String, Type)] \\ 
\>               -> Term \\ 
\>               -> Type \\ 
\>               -> [String] \\ 
\>               -> ([(Type, Type)], [String]) \\ 

\end{program*}
This function takes a context (denoted $\Gamma$ in the rules above and
represented by a list of {\tt{String}}, {\tt{Type}} pairs.), a term to infer
the type of, a type (denoted $\tau$ in the rules above and initially a type
variable not occurring anywhere in the context), and a string list containing
the names of all variables used so far.

\begin{program*}
\>  infer\_type context trm typ vars = \\ 
\>   case trm of \\ 
\>     (V x) ->  \\ 
\>        case (lookup x context) of  \\ 
\>          (Just a) -> ([(typ,a)],vars) \\ 
\>          Nothing ->  error ("infer\_type: " ++ x ++ " not in context!") \\ 
               
\>     (Ap m n) ->  \\ 
\>        let a = fresh "a" vars in \\ 
\>        let (e1,vars1) = infer\_type context m (Arrow (TyVar a) typ)(a:vars) in \\ 
\>        let (e2,vars2) = infer\_type context n (TyVar a) vars1 in \\ 
\>          (e1 ++ e2, vars2) \\ 

\>     (Abs x m) ->  \\ 
\>        let a = fresh "a" vars in \\ 
\>        let b = fresh "b" (a : vars) in \\ 
\>        let (e1,vars1) = infer\_type ((x,(TyVar a)):context) m (TyVar b) (a:b:vars) in \\ 
\>          ( [(typ, Arrow (TyVar a) (TyVar b))] ++ e1 , vars1) \\ 
\end{program*}

The case {\tt{V x}} implements the Axiom rule, the case labeled {\tt{(Ap m
n)}} implements the (Ap) rule and the case labeled {\tt{(Abs x m)}} implements
the (Abs) rule.


\subsection{Adding product types.}

To add product types  we extend the data-types {\tt{Type}} and {\tt{term}} as follows:

\begin{program*}
\> data Op = Arrow | Product  \\
\>         deriving (Eq,Show) \\
\> \\
\> data  Type =   TVar  String |  BinType Op  Type   Type \\
\>         deriving (Eq) \\
\> \\
\> data  Term =   Var String \\ 
\>              | Abs  String Term   \\
\>              | Ap Term Term   \\
\>              | Pair Term Term  \\
\>              | Fst Term  \\
\>              | Snd Term \\
\>        deriving (Eq) \\
\end{program*}

Mathematically we write $M\times{}N$ for the Haskell term {\tt{Prod A B}} and
render the Haskell term {\tt{(Pair M N)}} as $\langle{}M,N,\rangle$.
\ \\
Here are the additional proof rules:
\vspace{.125in}\\

\SequentRuleTwo{\Gamma,E_1\cup{}E_2\cup{}\{\tau=\alpha\times\beta\}}{\pair{M,N} : \tau}{\Gamma,E_1}{M:\alpha}{\Gamma,E_2}{N:\beta}{Pair}
{\hspace{.25in}} where $\alpha$ and $\beta$ are fresh.
\vspace{.125in}\\

\SequentRule{\Gamma,E}{{\mathit{fst}}\, M : \tau}{\Gamma,E}{M:\tau\times\alpha}{Fst}
{\hspace{.25in}} where $\alpha$ is fresh.
\vspace{.125in}\\

\SequentRule{\Gamma,E}{{\mathit{snd}}\, M : \tau}{\Gamma,E}{M:\alpha\times\tau}{Snd}
{\hspace{.25in}} where $\alpha$ is fresh.
\vspace{.125in}\\


\begin{exercise}
Using the base code provided (which includes unification for products) extend
the {\tt{infer\_type}} function to implement these additional type inference
rules.
\end{exercise}


\begin{exercise}
Design a number of test cases to show that your extension works.  You should at least include the test  examples in the file {\tt{hw18\_expected.txt}}.  Include at least six more.
\end{exercise}


\end{document}


% Local Variables:
% mode:latex
% comment-column:0
% comment-start: "\\> "
% comment-end: "\\\\ "
% compile-command: "pdflatex hw18f"
% fill-column:79
% End:




