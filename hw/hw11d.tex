\documentclass[11pt]{article}

\usepackage{amssymb}
\usepackage{remark}
% \usepackage{html}


\usepackage{/home/faculty/jlc/papers/tex-stuff/bussproofs}



\setlength{\textwidth}{6.5in}
\setlength{\textheight}{8.8in}
\setlength{\oddsidemargin}{0.0in}
\setlength{\evensidemargin}{0.0in}


 \newremark{theorem}{Theorem}[section]
 \newremark{Rule}{Rule}[section]
 \newremark{conjecture}{Conjecture}[section]
 \newremark{corollary}{Corollary}[section]
 \newremark{example}{Example}[section]
 \newremark{fact}{Fact}[section]
 \newremark{lemma}{Lemma}[section]
 \newremark{definition}{Definition}[section]
 \newremark{claim}{Claim}[section]
 \newremark{remark}{Remark}[section]
 \newremark{slogan}{Slogan}[section]
 \newremark{axiom}{Axiom}[section]
 \newremark{note}{Note}[]
 \newremark{problem}{Problem}[section]
 \newremark{exercise}{Exercise}[section]

\newcommand{\Proof}{{\goodbreak\noindent\bf{Proof:\ }}}
\newcommand{\qed}{\goodbreak\noindent$\Box$}

\newcommand{\byDef}[1]{\stackrel{\langle\langle{{\rm{def.\ of\ }}{#1}}\rangle\rangle}{=}}
\newcommand{\byEq}[1]{\stackrel{\langle\langle{{#1}}\rangle\rangle}{=}}
\newcommand{\by}[1]{{\small{\langle\langle{\mathit{by.\;def.\;of\;}}{#1}\rangle\rangle}}}


\newcommand{\dotcup}{{\cup\hspace{-.5em}\cdot}}
\newcommand{\barcup}{{\cup\hspace{-.68em}-}}


\newcommand{\arrow}{$\rightarrow$}
\newcommand{\nat}{{\mathbb{N}}}
\newcommand{\bool}{{\mathbb{B}}}
\newcommand{\Int}{{\mathbb{Z}}}

\newcommand{\nand}{{\overline{\wedge}}}
\newcommand{\nor}{{\overline{\vee}}}


\newcommand{\Meaning}[1]{{[{\hspace{-.13em}}[{#1}]{\hspace{-.13em}}]}}
\newcommand{\Meaningof}[2]{{[{\hspace{-.13em}}[\,{#2:#1}\,]{\hspace{-.13em}}]}}
\newcommand{\Skip}{{\bf{skip}}}
\newcommand{\Let}[2]{{\bf{Let\ }} {#1}{\bf{\ in\ }}{#2}{\bf{\ end}}}
\newcommand{\A}{{\tt{a}}}
% \newcommand{\F}{{\tt{F}}}
\newcommand{\ua}[1]{{\tt{(#1 under a)}}}
\newcommand{\Fua}{\ua{F}}
\newcommand{\aF}{{\tt{(a F)}}}
\newcommand{\Btt}{{\tt{Btt}}}
\newcommand{\Bff}{{\tt{Bff}}}
%\newcommand{\T3}{{\tt{T3}}}
%\newcommand{\F3}{{\tt{F3}}}
%\newcommand{\?3}{{\tt{?3}}}
\newcommand{\V}[1]{{\mkleeneopen{}{\tt{#1}}\mkleeneclose{}}}
\newcommand{\Neg}[1]{{\tt{N(#1)}}}
\newcommand{\I}[2]{{\tt{(#1)I(#2)}}}
\newcommand{\ie}{{\em{i.e.}}}
\newcommand{\eg}{{\em{e.g.}}}
\newcommand{\sat}[2]{{\tt{#1 \mbar{}= #2}}}
\newcommand{\nsat}[2]{{\tt{#1 \mbar{}\mneq{} #2}}}
\newcommand{\Three}{{\mBbbN{}\msubthree{}}}
%\newcommand{\Three}{{\bf{3}}}
\newcommand{\Tzero}{${\tt{0}}_{\tt{3}}$}
\newcommand{\Tone}{${\tt{1}}_{\tt{3}}$}
\newcommand{\Ttwo}{${\tt{2}}_{\tt{3}}$}
\newcommand{\msubst}{{\tt{/}}}
\newcommand{\nsequent}{{\tt{>>}}}

\newcommand{\kleene}[1]{{\mkleeneopen{}#1\mkleeneclose{}}}
\newcommand{\mfvar}[1]{{\kleene{{\tt{#1}}}}}
\newcommand{\mfnot}{{\kleene{\msim}}}
\newcommand{\mfand}{{\kleene{\mwedge}}}
\newcommand{\mfor}{{\kleene{\mvee}}}
\newcommand{\mfimp}{{\kleene{\mRightarrow}}}
\newcommand{\Knot}{$\sim_K\,$}
\newcommand{\Kand}{$\wedge_K$}
\newcommand{\Kor}{$\vee_K$}
\newcommand{\Kimp}{$\Rightarrow_K$}
\newcommand{\miff}{$\Leftrightarrow{}$}
\newcommand{\mlangle}{{$\langle$}}
\newcommand{\mrangle}{{$\rangle$}}
\newcommand{\mldots}{{$\ldots$}}
\newcommand{\mcdots}{{$\cdots$}}
%\newcommand{\rhd}{\triangleright}
\newcommand{\mrhd}{{$\triangleright$}}
\newcommand{\mRHD}{{$\triangleright_{\!R}\,$}}
\newcommand{\Sequent}[2]{{${#1}\,\vdash{}\,{#2}$}}

\newcommand{\typingRuleThree}[4]{{\begin{tabular}{c}{${#1}$} \hspace{2em} {${#2}$} \hspace{2em} {${#3}$}\\\hline{${#4}$}\end{tabular}}}
\newcommand{\typingRuleTwo}[3]{{\begin{tabular}{c}{${#1}$} \hspace{2em} {${#2}$}\\\hline{${#3}$}\end{tabular}}}
\newcommand{\typingRule}[2]{{\begin{tabular}{c}{${#1}$}{}\\\hline{${#2}$}{}\end{tabular}}}


\newcommand{\newSequentRule}[2]{{\begin{tabular}{c}{${#1}$}{}\\\hline{${#2}$}{}\end{tabular}}}

\newcommand{\sequentRule}[4]{\mbox{\raisebox{-.4ex}[4.25ex]{\begin{tabular}{rcl} \rule{0mm}{2.85ex}{$#1$} & {$\vdash$} & ${#2}$ \\\hline {\rule{0mm}{2.65ex}$#3$} & {$\vdash$} & {$#4$}\end{tabular}\\}}}
\newcommand{\AxiomRule}[3]{{\begin{tabular}{rcl} $\,$ \\\hline {\rule{0mm}{2.65ex}$#1$} & {$\vdash$} & {$#2$}\end{tabular}}(#3)}

\newcommand{\startProof}[1]{{\begin{tabular}{c} $\,$ \\\hline {\rule{0mm}{2.65ex}{#1}} \end{tabular}}}

\newcommand{\SequentRule}[5]{{\begin{tabular}{rcl} \rule{0mm}{2.85ex}{$#3$} & {$\vdash$} & ${#4}$ \\\hline {\rule{0mm}{2.65ex}$#1$} & {$\vdash$} & {$#2$}\end{tabular}}(#5)}

\newcommand{\SequentRuleTwo}[7]{{\begin{tabular}{lr} \Sequent{#3}{#4}\ &\ \Sequent{#5}{#6} \\\hline%
\multicolumn{2}{c}{{\rule{0mm}{2.65ex}{\Sequent{#1}{#2}}}} \end{tabular}}({#7})}


\newcommand{\cSequentRuleTwo}[7]{{\begin{tabular}{lr} \Sequent{${#1}$}{${#2}$}\ &\ \Sequent{${#3}$}{${#4}$} \\\hline%
\multicolumn{2}{c}{{\rule{0mm}{2.65ex}{\Sequent{${#5}$}{${#6}$}}}} \end{tabular}}({#7})}

\newcommand{\false}{\bf{false}}
\newcommand{\definedAs}{\;{\stackrel{\rm def}{=}}\;}
\newcommand{\pair}[1]{{\langle{#1}\rangle}}

\newcommand{\ttrue}{{\bf{T}}}
\newcommand{\ffalse}{{\bf{F}}}


% \renewcommand{\inr}[1]{{\tt{inr}({#1})}}
% \renewcommand{\inl}[1]{{\tt{inl}({#1})}}
% \renewcommand{\spread}[1]{{\tt{spread}}{#1}}
% \renewcommand{\decide}[1]{{\tt{decide}}({#1})}
% \renewcommand{\void}{{\tt{void}}}
% \renewcommand{\any}[1]{{\tt{any}{#1}}}
\newcommand{\Zero}{{\mathit{Zero}}}
\newcommand{\Succ}{{\mathit{Succ\,}}}
% \renewcommand{\ind}[1]{{\tt{ind}}{#1}}
% \renewcommand{\N}{\bf{N}}
% \renewcommand{\xxx}{\mbox{\hspace{.125in}}}
% \renewcommand{\Tpp}{${\bf{T}}^{++}$}
% \renewcommand{\TPP}{{\bf{T}}^{++}}
% \renewcommand{\GT}{{\bf{T}}}
% \newcommand{\abit}{$\,$}

\newcommand{\mysubtitle}[1]{{\ \\*[-.75em]{\bf{{#1}}}}}

{\obeyspaces\global\let =\ }

\newenvironment{bogustabbing}{\begin{tabbing}\={\mbox{\hspace{10em}}}\=\=\=\kill}%
{\end{tabbing}}


\newenvironment{program}{\tt\obeyspaces\begin{bogustabbing}\+\kill}{\end{bogustabbing}}
\newenvironment{program*}{\tt\obeyspaces\begin{bogustabbing}}{\end{bogustabbing}}
\newenvironment{program**}{\it\obeyspaces\begin{bogustabbing}}{\end{bogustabbing}}
\newenvironment{smallprogram*}{\hspace{2.5em}\small \it\obeyspaces\begin{bogustabbing}}{\end{bogustabbing}\vspace{-.0625in}}

\newcommand{\CASE}{\noindent{\bf{Case}}}
\newcommand{\expr}{{{\cal{E}}}}
\newcommand{\comm}{{{\cal{C}}}}
\newcommand{\mylet}[3]{{\tt{let\ }}{#1}{\tt{\ =\ }}{#2}{\tt{\ in\ }}{#3}}
\newcommand{\myfun}[2]{{\tt{fun\ }}{#1}{\tt{\ =\ }}{#2}}
\newcommand{\semwhile}[2]{{\tt{while\ }}{#1}{\tt{\ do\ }}{#2}{\tt{\ od:\,}}comm}
\newcommand{\while}[2]{{\tt{while\ }}{#1}{\tt{\ do\ }}{#2}{\tt{\ od}}}
\newcommand{\Seq}[2]{{#1}{\bf{\,;\,}}{#2}{\tt{:\,}}comm}
\newcommand{\semif}[4]{{\tt{if(}}{\Meaning{#1}({#4})}{\tt{,\ }}{\Meaning{#2}({#4})}{\tt{,\ }}{#3}{\tt{)}}}
\newcommand{\wif}[3]{{\tt{if}}{{#1}}{\tt{\ then\ }}{{#2}}{\tt{\ else\ }}{#3}{\tt{\ fi}}:\,comm}
\newcommand{\wnot}[1]{\neg{#1}}
\newcommand{\wassign}[2]{{\tt{loc}}_{#1} := {#2}}
\newcommand{\wderef}[1]{{\tt{@loc}}_{#1} }
\newcommand{\loc}[1]{${\tt{loc}}_{#1}$}
% \newcommand{\semif1}[4]{{\tt{if(}}{\Meaning{#1}({#4})}{\tt{,\ }}{\Meaning{#2}({#4})}{\tt{,\ }}{\Meaning{#3}({#4})}{\tt{)}}}

\newcommand{\mkleeneopen}{{\boldmath{^{\lceil}}}}
\newcommand{\mkleeneclose}{{\boldmath{^{\rceil}}}}
\newcommand{\kquote}[1]{{\mkleeneopen{}{{#1}}\mkleeneclose}}

\newcommand{\homework}[2]{\ \\\vspace{-1.25in}\\%
{\bf{HW {#1}}} \hfill {\bf{Prof. Caldwell}} \\%
{\bf{Due:}} {#2} 2013 \hfill {\bf{COSC 3015}}\ \\}

\newcommand{\ite}[3]{{\bf{if\ }}{#1}{\bf{\ then\ }}{#2}{\bf{\ else\ }}{#3}{\bf{\ fi}}}
\newcommand{\fun}{{\bf{fun\, }}}
\newcommand{\prog}[3]{{#1}{\bf{\ in \ }}{#2}}



\newcommand{\store}[1]{\langle{}{#1}\rangle}

\newcommand{\alphaeq}{\,{=_\alpha}\,}
\newcommand{\xleaf}{{\rm{Leaf}}}
\newcommand{\xnode}{{\rm{Node}}}
\newcommand{\vbar}{{\;\;{\bf{|}}\;\;}}
\newcommand{\abit}{{\mbox{\hspace{1em}}}}


\newcommand{\assign}[2]{{{#1} {\bf{:=}} {#2}}}
\newcommand{\sequence}[2]{{#1}\,{\bf{;}}\,{#2}}

\newcommand{\F}[1]{{\cal{F}}_{#1}}
\newcommand{\harrow}{{\tt{-\!\!\!>}}}
\newcommand{\smallsection}{{\goodbreak\[ \star {\hspace{.35in}} \star {\hspace{.35in}} \star \] \ \\}}
\newcommand{\append}{\,{\texttt{++}}\,}
\newcommand{\nil}{{\texttt{[{\hspace{.125em}}]}}}

\newcommand{\defof}[1]{\stackrel{<\!\!<{\textrm{def. of }}{#1}>\!\!>}{=}}
\newcommand{\pnote}[1]{{$\langle\!\langle$ {#1} $\rangle\!\rangle$}}

\newcommand{\spread}[3]{{\mathit{spread}({#1};{#2}.{#3})}}


\newcommand{\sfa}{{\sf{a}}}
\newcommand{\sfb}{{\sf{b}}}
\newcommand{\sfc}{{\sf{c}}}
\newcommand{\sfd}{{\sf{d}}}
\newcommand{\sfe}{{\sf{e}}}
\newcommand{\sff}{{\sf{f}}}


\begin{document}
\homework{11}{16 October}
g\section{Avl Trees}

In class we presented an implementation of Avl trees.  I have included the code
on the web-page together with this assignment.  For completeness I have included it here as well.


\begin{smallprogram*}
\>module Avl (Avl, show, fmap, foldr, all,  find, fork, isEmpty, empty, ht, insert, delete, join) where \\
\> \\
\>  import Prelude hiding (foldr,all) \\
\>  import Data.Foldable \\
\>  import Data.Functor \\
\> \\
\>  data Avl a = Null $\mid$ Fork Int (Avl a) a (Avl a)  \\
\> \\
\>  instance (Show a) $\Rightarrow$ Show (Avl a) where \\
\>    show Null = "[]" \\
\>    show (Fork h Null y Null) = "["  ++ show y ++ "]" \\
\>    show (Fork h xt y zt) = "[" ++ show xt ++ show y ++ show zt ++ "]" \\
\> \\
\>  empty = Null \\
\> \\
\>  instance Functor Avl where \\
\>    fmap f Null = Null \\
\>    fmap f (Fork h xt y zt) = Fork h (fmap f xt) (f y) (fmap f zt)   \\
\> \\
\>  instance Foldable Avl where \\
\>    foldr f id  Null = id \\
\>    foldr f id (Fork h xt y zt) = foldr f (f y (foldr f id zt)) xt \\
\> \\
\>  fork :: (Avl a) $\rightarrow$ a $\rightarrow$ (Avl a) $\rightarrow$ (Avl a) \\
\>  fork xt y zt = Fork h xt y zt \\
\>     where h = 1 + max (ht xt) (ht zt) \\
\> \\
\>  isEmpty Null = True \\
\>  isEmpty \_ = False \\
\> \\
\>  ht :: (Avl a) $\rightarrow$ Int \\
\>  ht Null = 0 \\
\>  ht (Fork h \_ \_ \_) = h \\
\> \\
\>  insert :: (Ord a) $\Rightarrow$ a $\rightarrow$ Avl a $\rightarrow$ Avl a \\
\>  insert x Null = fork Null x Null \\
\>  insert x (Fork h xt y zt)  \\
\>    $\mid$ (x $<$ y) = rebalance (insert x xt) y zt \\
\>    $\mid$ (x == y) = Fork h xt x zt \\
\>    $\mid$ (x $>$ y) = rebalance xt y (insert x zt) \\
\end{smallprogram*}
\newpage
\begin{smallprogram*}
\>  delete :: (Ord a) $\Rightarrow$ a $\rightarrow$ Avl a $\rightarrow$ Avl a \\
\>  delete x Null = Null \\
\>  delete x (Fork h xt y zt)  \\
\>    $\mid$ (x $<$ y) = rebalance (delete x xt) y zt \\
\>    $\mid$ (x == y) = join xt zt \\
\>    $\mid$ (x $>$ y) = rebalance xt y (delete x zt) \\
\> \\
\>  join :: Avl a $\rightarrow$ Avl a $\rightarrow$ Avl a \\
\>  join xt yt = if isEmpty yt then xt else rebalance xt y zt \\
\>                   where (y,zt) = splitTree yt \\
\> \\
\>  splitTree :: Avl a $\rightarrow$ (a, Avl a) \\
\>  splitTree (Fork h xt y zt) = \\
\>      if isEmpty xt then (y,zt) else (u,rebalance vt y zt) \\
\>         where (u,vt) = splitTree xt \\
\> \\
\>  bias :: Avl a $\rightarrow$ Int \\
\>  bias (Fork h xt y zt) = ht xt - ht zt \\
\> \\
\>  rotr :: Avl a $\rightarrow$ Avl a \\
\>  rotr (Fork m (Fork n ut v wt) y zt) = fork ut v (fork wt y zt) \\
\> \\
\>  rotl :: Avl a $\rightarrow$ Avl a \\
\>  rotl (Fork m ut v (Fork n rt s tt)) = fork (fork ut v rt) s tt \\
\> \\
\>  rebalance :: Avl a $\rightarrow$ a $\rightarrow$ Avl a $\rightarrow$ Avl a \\
\>  rebalance xt y zt  \\
\>    $\mid$ (hz+1 $<$ hx) \&\& (bias xt $<$ 0) = rotr (fork (rotl xt) y zt) \\
\>    $\mid$ (hz+1 $<$ hx)                       = rotr (fork xt y zt) \\
\>    $\mid$ (hx+1 $<$ hz) \&\& (0 $<$ bias zt) = rotl (fork xt y (rotr zt)) \\
\>    $\mid$ (hx + 1 $<$ hz)                     = rotl (fork xt y zt) \\
\>    $\mid$ otherwise                           = fork xt y zt \\
\>        where hx = ht xt \\
\>                hz = ht zt \\
\end{smallprogram*}



The assignment here is to build
a type of finite functions where the underlying representation is the {\it{Avl}} tree.
In class we discussed how the {\it{Avl}} trees are like sets - if you add an element to the set
that is already there, it does not chance the set.
 
A finite function is a set of functional pairs.  Recall the definition of
functionality for a function $f$ from $a$ to $b$.

\[
\forall{}x:a.\;\forall{}y,z:b.\; (\pair{x,y}\in{}f \wedge \pair{x,z}\in{}f)
\Rightarrow y = z\] 

That is, no two distinct elements can be paired with the same first
element. Thus, to model functions with {\it{Avl}} trees, you will need to be careful
never to add a pair $(x,y)$ to the tree when there is already a pair in the
tree with $x$ as its first element.  Also, you will want to order the pairs in
the tree by their first elements.  There is an elegant way to do this in
Haskell by creating a special type of {\it{FPair}}. You can instantiate the
type {\it{FPair}} in the {\it{Eq}} type class by ignoring the second element
and you can instantiate {\it{FPair}} in the {\it{Ord}} type class by ordering
on the first element.  Then, an {\it{Avl}} tree of FPairs will satisfy the constraints
just mentioned.  Here is a module of {\it{FPair}}.

\begin{smallprogram*}
\>   module FPair where \vspace{1em}\\
\>   data FPair a b = P a b   \vspace{1em}\\
\>   instance (Show a, Show b) $\Rightarrow$ Show (FPair a b) where\\
\>      show (P x y) = show x ++ " := " ++ show \vspace{1em}\\
\>   instance (Eq a) $\Rightarrow$ Eq (FPair a b) \\
\>      (P x y) == (P z w) = x == \vspace{1em}\\
\>   instance (Ord a) $\Rightarrow$ Ord (FPair a b) where\\
\>     (P x y) $le$ (P z w) = x $<$= \\
\end{smallprogram*}

Now a type {\it{FinFun}} of finite functions can be defined where the
implementation is defined as follows:

\begin{smallprogram*}
\>   data FinFun a b = FinFun (Avl (FPair a b)) \vspace{1em} \\
\>   instance (Show a, Show b) $\Rightarrow$ Show (FinFun a b) where\\
\>      show (FinFun f) = show (foldr (:) [] f) \\
\end{smallprogram*}

\begin{exercise}
  Based on this implementation idea - you need to implement the following
  constants and operations on finite functions.  I have included some test
  cases with this assignment.

\begin{smallprogram*}
\>  empty :: FinFun a b \\
\>  apply :: (Eq a) $\Rightarrow$ (FinFun a b) $\rightarrow$ a $\rightarrow$ b \\
\>  update :: (Ord a) $\Rightarrow$ FinFun a b $\rightarrow$ (a, b) $\rightarrow$ FinFun a b \\
\>  dom :: FinFun a b $\rightarrow$ [a] \\
\>  range :: FinFun a b $\rightarrow$ [b] \\
\>  injection :: FinFun a b $\rightarrow$ Bool
\end{smallprogram*}

We briefly describe each.

\begin{smallprogram*}
\>  empty :: FinFun a b \\
\end{smallprogram*}
The empty {\it{FinFun}} behaves as a function with an empty domain and range.


\begin{smallprogram*}
\>  apply :: (Eq a) $\Rightarrow$ (FinFun a b) $\rightarrow$ a $\rightarrow$ b \\
\end{smallprogram*} {\it{apply f x}} applies the {\it{FinFun}} $f$ to the
argument $x$.  If there is no {\it{FPair}} in $f$ with first element $x$, raise
an exception. If $x$ is the first element of some element {\it{P x y}} return
$y$.

\begin{smallprogram*}
\>  update :: (Ord a) $\Rightarrow$ FinFun a b $\rightarrow$ (a, b) $\rightarrow$ FinFun a b \\
\end{smallprogram*} 
A call of the form {\it{update f (x,y)}} inserts the {\it{FPair}}, {\it{P x y}}
into the {\it{FinFun}} f. You might study the {\it{Avl}} tree code for insert
to see why this might work. Consider the equality and order relation for
{\tt{FPairs}}.

\begin{smallprogram*}
\>  dom :: FinFun a b $\rightarrow$ [a] \\
\>  range :: FinFun a b $\rightarrow$ [b] \\
\end{smallprogram*} 
The functions return lists of the domain elements and the range elements for a
{\it{FinFun}}.  Note that there is a {\it{foldr}} operator defined on
{\it{Avl}} trees which you may find useful.

\begin{smallprogram*}
\>  injection :: (Eq a, Show a, Eq b) => FinFun a b -> Bool\\
\end{smallprogram*} 
A function is an injection of every distinct element of the domain gets mapped
to a distinct element of the range {\em{i.e.}} no two elements $x$ and $y$
($x\not=y$) of the domain get mapped to the same element of the range.  Write a
predicate to check if a {\it{FinFun}} is an injection. 


% \subsection
% \subsection{{\tt{update :: Eq a $\Rightarrow$ (a,b) $\rightarrow$ FinFun a b $\rightarrow$ FinFun a b}}}

% This function should be called as {\tt{update(i,v) f}} where {\tt{f}} is a
% finite function with domain {\tt{a}} and codomain {\tt{b}}.  If {\tt{i}} is the
% first element of any pair in {\tt{f}} return the finite function that is just
% like {\tt{f}} except that it maps {\tt{i}} to {\tt{v}}.  If {\tt{i}} is not the
% first element of any pair in {\tt{f}}, return the finite function the behaves
% just like {\tt{f}} but also contains the pair {\tt(i,v)}. You might find your
% {\tt{remove\_all}} function from the previous homework useful.

% \subsection{\tt{{dom :: (Eq a, Eq b) $\Rightarrow$ FinFun a b -> [a]}}}
% This function returns a list of values in the domain {\tt{a}} that the function
% is actually defined for.  Hint - recall that the function {\tt{map}} applies a
% function to every element of a list and that {\tt{fst}} and {\tt{snd}} project
% the first and second elements from a pair.

% \subsection{\tt{range :: (Eq a, Eq b) $\Rightarrow$ FinFun a b -> [b]}}
% This function returns a list of values in the codomain {\tt{b}} that is the
% range of the finite function. See hint above.

% \subsection{\tt{{apply :: (Eq a, Eq b) $\Rightarrow$ FinFun a b -> a -> Maybe b}}}
% {\tt{apply f x}} models function application. Look up {\tt{x}} in the finite
% function {\tt{f}} and return {\tt{Nothing}} if {\tt{x $\not\in$ (domain f)}}
% and return {\tt{Just y}} when the pair {\tt{(x,y) $\in$ f}}.  


% % \exercise{Instantiate the type of finite functions {\tt{FinFun}} as an instance
% %   of the {\tt{Eq}} type class by defining {\tt{==}} to be the extensional
% %   equality on functions.

% % Extensional equality for finite functions means the list representing the
% % function is functional {\em{and}} they have the same elements.  }


\end{exercise}


\end{document}
% Local Variables:
% mode:latex
% comment-column:0
% comment-start: "% "
% compile-command: "pdflatex hw11d"
% fill-column:79
% End:




